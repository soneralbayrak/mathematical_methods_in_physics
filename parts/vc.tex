\chapter{Definition and fundamentals of Vector Spaces}
\section{Preliminaries: some basic terminology}
\subsection{Primer: groups, fields, and algebras}
Groups, rings, skew fields,fields, linear spaces, linear algebras, lie algebras, exponential map, lie groups
\subsection{Properties of Linear spaces}
Bases, standard basis, coordinate systems, relations to tuples and matrices
\section{Inner product spaces}
Concept of \emph{vector spaces with additional structure}, dual vector spaces, inner product, normed vector spaces, topological vector spaces, Hilbert spaces
\section{Algebras over fields}
\subsection{Tensor algebras}
Tensor algebras and tensors
\subsection{Exterior and alternating algebras}
Wedge product and multivectors, oriented area and volume, hodge duality

\subsection{Division algebras}
Normed division algebras and Hurwitz's theorem; finite-dimensional associative division algebras over the real numbers and Frobenius theorem; more about quaternions, their usage in spatial rotation (hence applications in engineering), and Feza Gursey's work with quaternions


\chapter{Differentiation in Vector Calculus}
\section{Vector fields on Euclidean Spaces}
Concept of vector fields, use of partial derivatives to describe vector fields, bivector fields ($F_{\mu\nu}$ in electromagnetism being a bivector), tensor fields
\section{Forms and duals of vector fields}
Forms as covectors, tangent and cotangent bundles, musical isomorphism, index notation of tensors, covariant vs contravariant indices, metric tensor field
\section{Grad, div, curl, and Laplacian}
\subsection{Exterior derivative and hodge star operator}
\subsection{Invariant formulation of differentiation in vector calculus}
The standard vector calculus operators in terms of exterior derivative, hodge star operator, and musical isomorphism.

\section{Helmholtz decomposition of vector fields}
Helmholtz decomposition and a brief mention of its generalization Hodge decomposition

\chapter{Integration in Vector Calculus}
\section{Curves and Line Integrals}
Curves, parametrization, arc length, Frenet–Serret formulas, line integral
\section{Generalized approach to integration}
Chains, boundaries of chains, Poincare lemma
\section{Integral theorems}
Discussion of generalized stokes theorem in forms, and its implication as \emph{Gradient theorem}, \emph{Divergence theorem}, \emph{Curl theorem}, and \emph{Green's theorem}.

\chapter{Curvilinear Coordinate Transformations}
Concept of coordinate charts, transformation of tensor fields under coordinate transformations, common orthogonal curvilinear coordinates (polar, cylindrical, spherical)
