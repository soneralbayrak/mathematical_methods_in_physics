\chapter{Definition and fundamentals of Vector Spaces}
\section{Preliminaries: some basic terminology}
\subsection{Primer: groups, fields, and algebras}
The vector spaces are some structures build on fields, which are themselves built on groups. The groups are built on sets. So we start with sets.

Our practical reason to start with sets is that we need to properly understand them to understand vectors, but one can even say that set theory is the basis of all mathematics! Indeed, historically, 20th century Mathematicians did try to build the foundations of math on sets, but Russell's paradox squeezed in and ruined the whole program, at least to the degree it was envisioned back then. Today, even more fundamental approaches (such as category theory or type theory) try to explain whole of math, but this is beyond our scope: sets are good enough to be the foundation of everything we will care in this book, so we will stick with them to prepare the background for vector spaces.

\paragraph{Sets:} A set is a collection of objects, and its elements determine what the set is. Importantly, we know all of the elements of a set!\footnote{This is actually a rather important point for deep mathematical reasons that we will not visit, so what follows in this footnote is beyond the scope of this book (rather disappointing for me).\\\\ The fact that we cannot know everything in a quantum system suggests that we should probably not use sets for quantum information, and this is actually true! The right mathematical object to use in quantum system does not have the type \texttt{Set} but has the type \texttt{Hilb}, or more explicitly, quantum systems belong to a different \emph{category} then the category of sets! An explicit discussion of these would require us to cover category theory, but an interested reader can consult here \draftnote{add sources}.}

Sets can be denoted as explicit list of elements within curly brackets;\footnote{Note that sets are orderless\footnotemark and that duplication of elements do not change a set, e.g.
\bea 
A,B,C::{}&\texttt{Set}\\
A={}&\{1,2,3,4\}\\
B={}&\{1,4,3,2\}\\
C={}&\{1,2,3,4,3,2,1\}
\eea 
implies $A=B=C$.
}
\footnotetext{
Ordered sets are denoted with \emph{round} brackets, e.g.
\bea 
A::{}&\texttt{Ordered Set}\\
A={}&(1,2,3)
\eea 
where $(1,2,3)\ne(1,3,2)$. Ordered sets can be constructed via usual sets, e.g.
\be 
(1,2,3)\leftrightarrow\{\{1\},\{1,2\},\{1,2,3\}\}\\
(1,3,2)\leftrightarrow\{\{1\},\{1,3\},\{1,2,3\}\}\\
\ee 
} for instance
\bea 
A::{}&\texttt{Set}\label{eq: set example type}\\
A={}&\{1,2,3\}\label{eq: set example}
\eea 
Equation~\eqref{eq: set example} shows that the set $A$ consists of the elements $1$, $2$, and $3$. Equation~\eqref{eq: set example type} on the other hand tells us that \emph{$A$ is of type Set}, i.e. $A$ is a set: for more details about this notation, please check out Section~\ref{sec: type notation and functions}.

Sets can have infinitely many elements and we can use dots to infer the rest from the given; for instance, the set $\{\dots,-1,1,3,5,\dots\}$ can denote the set of all odd integers. Although rather convenient, this method is inherently ambiguous; a more proper way to denote sets of infinite elements is through \emph{set comprehensions}.

A set comprehension is a way to build a set, and it is usually used with \emph{predicates}.\footnote{Predicates are functions with the codomain \texttt{Boolean}.\footnotemark In other words, a function is a predicate if it only yields two outputs: \textsf{True}, or \textsf{False}. For instance, the function ``isOddInteger'' defined as
\bea 
\text{isOddInteger}::{}&\texttt{Boolean}\\
\text{isOddInteger}={}&x\to \left(\frac{x+1}{2}\in\Z\right)
\eea 
is a \emph{predicate}, e.g. \mbox{$\text{isOddInteger}(1)=$\textsf{True}}, \mbox{$\text{isOddInteger}(2)=$\textsf{False}}, \mbox{$\text{isOddInteger}(5/2)=$\textsf{False}}, etc.
}
\footnotetext{This is the definition due to \emph{Gottlob Frege}, the father of axiomatic first-order logic. You may find other definitions in different branches of math: an interested reader can check Wikipedia for more.} There are two conventional ways to write down a set $A$ this way:
\begin{itemize}
	\item $A=\{x\in\text{ Domain}\;|\;\text{conditions}\}$
	\item $A=\{f(x)\;|\;\text{conditions}\}$ which is just an abbreviated version of the method above in the sense that this form actually means
\end{itemize}
\be 
A={}&{}\{y\in\text{ Domain inferred from the conditions}\\&\quad|\;\exists x\text{ s.t. }y=f(x)\text{ \& }\text{conditions}\}
\ee 
For instance, the rigorous way to denote the odd integers, compared to a convenient casual notation $\{\dots,-3,-1,1,3,\dots\}$, is\linebreak $\{a\in\Z\;|\;(\exists n\in\Z)\;a=2n+1\}$ by the first approach.\footnote{This would read as \emph{``the set of integer $a$'s for which there exists an integer $n$ such that $a=2n+1$ is true''}.\footnotemark}\footnotetext{
We can understand this more deeply as follows. Consider the \emph{predicate} $P$ defined as 
\be 
P::{}&{}\Z\to\texttt{Boolean}\\
P={}&{}a\to (\exists n\in\Z)\;a=2n+1
\ee 
The set of odd integers are then constructed by using this predicate with the list comprehension, i.e.
\be 
\text{odd integers}=\{a\in\Z\;|\; P(a)\}
\ee 
This also shows how the list comprehension works: $S=\{T\;|\; C\}$ means that the set $S$ consists of all elements of $T$ for which the condition $C$ is true.
}${}^{\;}$\footnote{We remind the reader that $\exists$ reads as \emph{``there exists$\dots$''}, and it is one of the \emph{quantifiers} in logic. The other two important ones that we will make use of are $\forall$ and $\exists !$, which denote \emph{``for all$\dots$''} and \emph{``there exists a unique$\dots$''} respectively. For instance, $(\exists!\;n\in\Z)\;n^2=n$ is the statement \emph{``there exists a unique integer $n$ such that $n^2=n$''}, and it is \textsf{False} as both $0$ and $1$ satisfy $n^2=n$: $\boxed{\left((\exists!\;n\in\Z)\;n^2=n\right)=\textsf{False}}$.\footnotemark\\\\
It is no coincidence that I put the dots to the right of the expressions for the quantifiers; one should read them as in a sentence, hence their order matters. For instance
\bea 
\Big((\forall a\in\Z)(\exists b\in\Z)\;b=2a\Big)=\textsf{True}\\
\Big((\exists b\in\Z)(\forall a\in\Z)\;b=2a\Big)=\textsf{False}
\eea 
where the first expression is \emph{``For all integers, there exists an integer which is twice the former''} and the second one is \emph{``There exists an integer which is twice all integers''}.
}\footnotetext{
It may seem weird at first that there is a variable in the left hand side of the equality $\left((\exists!\;n\in\Z)\;n^2=n\right)=\textsf{False}$, which does not appear on the right hand side. The variable $n$ here is a \emph{dummy/scooping/bound} variable hence the right hand side should be independent of it, just like the function $g(y)$ in $\int f(x,y)dx=g(y)$ being independent of $x$. For more details about this, see the comprehensive footnote~\ref{footnote:dummy variables}.} In comparison, the second approach makes it simpler: $\{2n+1\;|\; n\in\Z\}$.

I will assume that the readers are familiar with the comparisons of sets\footnote{$A\subseteq B$ (or $B\supseteq A$) means that all elements of $A$ are also elements of $B$ ($A$ is called a \emph{subset} of $B$), and $A\subset B$ (or $B\supset A$) means the same thing with the additional condition that $A\ne B$ ($A$ is called a \emph{proper subset} of $B$).} and miscellaneous details of sets.\footnote{For instance, $\abs{S}$ denotes the number of elements in the set $S$, e.g. $\abs{\{1,2,4\}}=3$. If the set $S$ is infinite, we write $\abs{S}=\infty$. As expected, $S\subseteq T$ implies $\abs{S}\le\abs{T}$. I would also like to introduce \emph{power sets}: the power set of a set $S$ is denoted by $\mathcal{P}(S)$ and it is the set whose elements are the subsets of the set $S$. For example,
\bea
S,\mathcal{P}(S)::{}&{}\texttt{Set}\\
S={}&{}\{a,b,1\}\\
\mathcal{P}(S)={}&{}\Big\{\{\},\{a\},\{b\},\{1\},\{a,b\}\\
{}&{}\{a,1\},\{b,1\},\{a,b,1\}\Big\}
\eea 
We note that if the set $S$ is finite, we have $\abs{\cP(S)}=2^{\abs{S}}$, which is indeed the case in the above example ($8=2^3$). This relation is why the power set of $S$ is also loosely denoted as $2^S$.
} I will also skip discussing the details of operations between sets.\footnote{Six important operations to know are \emph{union}, \emph{intersection}, \emph{difference} (also called \emph{complement}), \emph{Cartesian product}, \emph{disjoint union}, and \emph{quotient by an equivalence relation}. I expect the reader to know the first four operations: the last two will not be relevant for this book, but I advice the interested reader to learn more about them.} Lastly, I would like to mention the relevant concepts such as \emph{relations}, \emph{partitions}, and \emph{quotients}: one should know about them for a good general mathematical background, but they will unfortunately not be covered in this book.

\paragraph{Groups:} Consider a set $S$ and impose the existence of two functions $o$ and $i$ such that 
\bea 
S::{}&{}\texttt{Set}\\
o::{}&{}(S,S)\to S\\
i::{}&{}S\to S
\eea 
where $S$ is of type set, and the function $o$ has the domain ``tuple of $S$'' and the codomain $S$ ---check out section~\ref{sec: partial differential equations} for more details about tuples. In simpler terms, the function $o$ takes two inputs (both being elements of the set $S$) and yields an element of the set $S$; in comparison, the function $i$ takes an element of $S$ to another (not necessarily different) element of $S$.

The set $S$ with these two functions $o$ and $i$ form a group if the following statements are true:
\begin{enumerate}
	\item $\boxed{(\exists e\in S)(\forall s\in S)\; o(e,s)=o(s,e)=s}$. In words, this says that\linebreak \emph{``There exists an element $e$ of the set $S$ and $o(e,s)=o(s,e)=e$ for any element $s$ of this set''}. The element $e$ is called \emph{the identity element} with respect to the \emph{group operation} $o$.\footnote{Suppose for a second that there are two identity elements: $e_1,e_2$. Since $e_1$ is identity element, we need $o(e_1,e_2)=e_1$; but the same argument for $e_2$ dictates $o(e_1,e_2)=e_2$: we need $e_1=e_2$. In conclusion, if identity element exists, it is \emph{unique}!}
	\item  $\boxed{(\forall s \in S)\; o(s,i(s))=o(i(s),s)=e}$. In words, this says that \emph{``The group operation $o$ acting on the tuple $(s,i(s))$ or $(i(s),s)$ yields the identity element $e$, for any element $s$ of the set $S$''}. In practice, this condition specifies the action of $i$ on any element $s$, and $i(s)$ is called the \emph{inverse} of the element $s$.
	\item $\boxed{(\forall a,b,c \in S)\; o(a,o(b,c))=o(o(a,b),c)}$. This is called \emph{associative law}, and it is best visualized when we use \emph{infix notation}: if we denote $o(a,b)$ as $a*b$, then this condition simply reads as $a*(b*c)=(a*b)*c$. Therefore, the parentheses are not important, and we can even loosely denote without parentheses: $a*(b*c)=(a*b)*c=a*b*c$.
\end{enumerate}

We will \emph{define} a group to be the pair $(S,o)$, i.e.
\be 
\texttt{Group}=\Big(\texttt{Set},\;(\texttt{Set},\texttt{Set})\to\texttt{Set}\Big)
\ee 
such that the above statements are true (satisfaction of those rules will in turn determine the inverse function $i::S\to S$ as well). For instance, the pair $(\Z,+)$ is a group, where $+$ denotes the arithmetic addition, and the inverse function turns out to be multiplication by $-1$: we say that \emph{integers form a group under arithmetic addition}. We can indeed check that the above statements are true:
\begin{enumerate}
	\item $(\forall a\in \Z)\; a+0=0+a=a$ (addition by $0$ does not change any integer)
	\item $(\forall a\in \Z)\; a+(-a)=(-a)+a=0$ (addition by the inverse element takes an integer to zero)
	\item $(\forall a,b,c \in \Z)\; a+(b+c)=(a+b)+c$ (the order of addition does not change the result)
\end{enumerate}

As another example for groups, consider the set\linebreak$\textsf{classrooms}=\{P1,P2,P3\}$ and the group operation denoted as $<>$ in infix notation. If we are given the information
\be 
\label{ex: classrooms}
P1<>P1={}&{}P3\\
P2<>P2={}&{}P2\\
P3<>P3={}&{}P1\\
P1<>P2=P2<>P1={}&{}P1\\
P1<>P3=P3<>P1={}&{}P2\\
P2<>P3=P3<>P2={}&{}P3
\ee 
we can immediately deduce that $P2$ is the identity element, and $P1^{-1}=P3$, $P2^{-1}=P2$, \& $P3^{-1}=P1$ (inverse operations). One can check that the necessary conditions are satisfied, hence we state
\be 
(\textsf{classrooms},<>)::\texttt{Group}
\ee 
We will leave further details of groups to the interested reader to research! One important exception will be the concept of \emph{commutativity}: if the group operation is commutative, i.e. $a*b=b*a$, then that group is called a \emph{commutative group}. In commutative groups, one conventionally use the infix operator $+$ instead of $*$ for the group operation, $-s$ instead of $s^{-1}$ to denote inverse operation, and $0$ instead of $1$ to denote identity element. For instance, we can rewrite the example in \equref{ex: classrooms} as $\textsf{classrooms}=\{P1,P2,P3\}=\{P1,0,-P1\}$. Note that the minus ($-$) sign and the symbol $(0)$ has nothing to do with the arithmetic negation and the number $0$ in general: this is simply a conventional notation.

\paragraph{Rings:} Consider a set $S$ and two functions $+$ and $\.$ such that 
\bea 
S::{}&{}\texttt{Set}\\
+::{}&{}(S,S)\to S\\
\.::{}&{}(S,S)\to S
\eea 
where we will denote the functions in the infix form, e.g. $a+b$ instead of $+(a,b)$. We say that the triplet $(S,+,\.)$ is a ring, i.e.
\be 
(S,+,\.)::\texttt{Ring}
\ee
if the following statements are true:
\begin{enumerate}
	\item $\boxed{(S,+)\text{ is a commutative group}}$
	\item $\boxed{(\forall a,b,c\in S)\; a\.(b+c)=a\.b+a\.c}$
	\item $\boxed{(\forall a,b,c\in S)\; (b+c)\.a=b\.a+c\.a}$
\end{enumerate}
In other words, rings are simply commutative groups with an additional operation (call multiplication) that distributes over group addition.

The most familiar rings are those where the ring addition and multiplication operations are literally arithmetic addition and multiplication; for instance, $(\Z,+,\.)$ forms a ring and so does $(\R,+,\.)$. Likewise, rectangular matrices with complex entries form a ring under matrix addition and matrix multiplication, i.e. $(\cM_{n\x n}(\C),+,\.)::\texttt{Ring}$.

\paragraph{Skew field:} We say that a triplet $(S,+,\.)$ is a skew field, i.e.
\be 
(S,+,\.)::\texttt{Skew Field}
\ee
if the following statements are true:
\begin{enumerate}
	\item $\boxed{(S,+,\.)\text{ is a ring}}$
	\item $\boxed{(S\backslash\{0\},\.)\text{ is a group}}$
\end{enumerate}
In other words, skew fields are rings where multiplication operation also forms a group if we remove the identity element of the addition operation. The prototypical example of the skew fields is the \emph{quaternions}.\footnote{\draftnote{Put some discussion of quaternions}.}

\paragraph{Fields:}  We say that a triplet $(S,+,\.)$ is a field, i.e.
\be 
(S,+,\.)::\texttt{Field}
\ee
if the following statements are true:
\begin{enumerate}
	\item $\boxed{(S,+,\.)\text{ is a ring}}$
	\item $\boxed{(S\backslash\{0\},\.)\text{ is a commutative group}}$
\end{enumerate}

Fields are ubiquitous and super important for physicists as we use them everwhere! The rational, real, and complex numbers all form fields under ordinary arithmetic addition and multiplication:\footnote{
For a more exotic example, consider the following set:
\be 
A=\Big\{(x\to x+a)\;\Big|\; a\in \R\Big\}
\ee 
for the ordinary arithmetic addition. This is a set of \emph{functions}, i.e. elements of the set are functions; for instance, one of the elements of this set is $x\to x+1$: this is a function that adds one to its input.
\\\\
Let us rewrite this set in a more familiar way. For this, we define the \emph{higher order function} ``\textsf{plus}'':
\be 
\textsf{plus}::{}&{}\R\to(\R\to\R)\\
\textsf{plus}={}&{}a\to\big(x\to(x+a)\big)
\ee 
This higher order function becomes an ordinary function if given an input, i.e. $\textsf{plus}(1)::\R\to\R$, which can then be given one more input that converts it into a real number: $\textsf{plus}(1)(5)=6::\R$. Indeed, this higher order function simply adds its inputs, i.e. $\textsf{plus}(a)(b)=a+b$. In terms of this higher order function, we can write down the set $A$ above as
\be 
A=\Big\{\textsf{plus}(a)\;\Big|\; a\in \R\Big\}
\ee 
Let us now introduce two higher order functions:
\be
\oplus,\otimes ::{}\big((\R\to\R),(\R\to\R)\big)\to (\R\to\R)
\ee
These functions take a pair of functions, and then give another function! Let us use these guys in infix form, e.g. $\textsf{plus}(a) \oplus\textsf{plus}(b)$.
\\\\
If we are given the information that 
\be 
\textsf{plus}(a) \oplus\textsf{plus}(b)=\textsf{plus}(a+b)
\ee 
we can immediately say $(A,\oplus)$ forms a commutative group, with the identity element $\textsf{plus}(0)$. If we are given the further information
\bea 
{}&\big(\textsf{plus}(a) \oplus\textsf{plus}(b)\big) \otimes\textsf{plus}(c)=\nn\\
{}&\big(\textsf{plus}(a)\otimes\textsf{plus}(c)\big) \oplus\big(\textsf{plus}(b)\otimes\textsf{plus}(c)\big)
\\
{}& \textsf{plus}(c)\otimes\big(\textsf{plus}(a) \oplus\textsf{plus}(b)\big)=\nn\\
{}&\big(\textsf{plus}(c)\otimes\textsf{plus}(a)\big) \oplus\big(\textsf{plus}(c)\otimes\textsf{plus}(b)\big)
\eea 
we can then say that $(A,\oplus,\otimes)$ forms a ring! Finally, if we are also given
\be 
\textsf{plus}(a) \otimes\textsf{plus}(b)=\textsf{plus}(a.b)
\ee 
then we can prove that $(A,\oplus,\otimes)$ is actually a field!
}
\be 
(\Q,+,\.)::\texttt{Field}\;,\qquad (\R,+,\.)::\texttt{Field}\;,\qquad (\C,+,\.)::\texttt{Field}
\ee  
\draftnote{To be added: linear spaces, linear algebras, lie algebras, exponential map, lie groups}
\subsection{Properties of Linear spaces}
Bases, standard basis, coordinate systems, relations to tuples and matrices
\section{Inner product spaces}
Concept of \emph{vector spaces with additional structure}, dual vector spaces, inner product, normed vector spaces, topological vector spaces, Hilbert spaces
\section{Algebras over fields}
\subsection{Tensor algebras}
Tensor algebras and tensors
\subsection{Exterior and alternating algebras}
Wedge product and multivectors, oriented area and volume, hodge duality

\subsection{Division algebras}
Normed division algebras and Hurwitz's theorem; finite-dimensional associative division algebras over the real numbers and Frobenius theorem; more about quaternions, their usage in spatial rotation (hence applications in engineering), and Feza Gursey's work with quaternions


\chapter{Differentiation in Vector Calculus}
\section{Vector fields on Euclidean Spaces}
Concept of vector fields, use of partial derivatives to describe vector fields, bivector fields ($F_{\mu\nu}$ in electromagnetism being a bivector), tensor fields
\section{Forms and duals of vector fields}
Forms as covectors, tangent and cotangent bundles, musical isomorphism, index notation of tensors, covariant vs contravariant indices, metric tensor field
\section{Grad, div, curl, and Laplacian}
\subsection{Exterior derivative and hodge star operator}
\subsection{Invariant formulation of differentiation in vector calculus}
The standard vector calculus operators in terms of exterior derivative, hodge star operator, and musical isomorphism.

\section{Helmholtz decomposition of vector fields}
Helmholtz decomposition and a brief mention of its generalization Hodge decomposition

\chapter{Integration in Vector Calculus}
\section{Curves and Line Integrals}
Curves, parametrization, arc length, Frenet–Serret formulas, line integral
\section{Generalized approach to integration}
Chains, boundaries of chains, Poincare lemma
\section{Integral theorems}
Discussion of generalized stokes theorem in forms, and its implication as \emph{Gradient theorem}, \emph{Divergence theorem}, \emph{Curl theorem}, and \emph{Green's theorem}.

\chapter{Curvilinear Coordinate Transformations}
Concept of coordinate charts, transformation of tensor fields under coordinate transformations, common orthogonal curvilinear coordinates (polar, cylindrical, spherical)
