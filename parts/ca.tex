\chapter{Basics of complex calculus}
\section{Structures of complex numbers}
\subsection{Complex numbers as a Euclidean vector space}
With $\{1,i\}$ as the standard basis, complex numbers are simply a real vector space of dimension two: complex plane as cartesian plane, geometric interpretation of complex numbers, complex absolute value as Euclidean norm, complex cojugation, polar form
\subsection{Further properties of complex numbers}
Complex numbers as an algebraically closed field, complex numbers as a commutative algebra over the reals, complex numbers (endowed with the metric $d(z_1,z_2)=\abs{z_1-z_2}$) as a complete metric space
\section{Complex differentiation}
Cauchy–Riemann equations, holomorphicity, Cauchy's integral formula

\chapter{Riemann Surfaces}
\section{Riemann sphere}
Extended complex numbers, projective maps
\section{Analysis of elementary functions via Riemann surfaces}
The qualitative analysis of the analytic structure of elementary functions via their Riemann surfaces (square-root, log, etc.),  

\chapter{Analytic Structure of Multi-valued ``functions''}
\section{Branch points, branch cuts, and singularities}
Branch analysis, different sheets, essential singularities, etc.
\section{Series expansion of complex functions}
Taylor and Laurent series, relations to Cauchy's integral formula, poles and residues, residue theorem
\section{Analytic continuation and monodromy theorem}