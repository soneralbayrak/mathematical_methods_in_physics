%%% METADATA
\def\publishnote{Originally prepared for \textit{METU - Phys209 \& Phys210}}
\def\titlee{Mathematical Methods in Physics}
\def\authorr{Soner Albayrak}
\def\datee{\today}
\def\abstractt{These are the notes of the lectures prepared for the ``PHYS209 Mathematical Methods in Physics I'' and ``PHYS210 Mathematical Methods in Physics  II'' courses at \emph{Middle East Technical University}, 2023-2024 Fall \& Spring Semesters ---see\linebreak \hyperref{https://soneralbayrak.com/teaching}{}{}{https://soneralbayrak.com/teaching/}. These notes are mostly based on other sources and I provided the sources in relevant places. Whenever I get the chance, I will keep updating the notes to keep it up-to-date and self-contained; there are also reminders \draftnote{in blue} for me to add further discussion/comments.
\par For contact: \href{mailto:contact@soneralbayrak.com}{contact@soneralbayrak.com}
}
%
%%% EDITING PREFERENCES
\def\showtrimss{no}
%
%%% BOOK SPECIFIC CHOICES
\def\lang{english}
\def\numberwithinn{chapter} % equation numbering
\def\chapterstylechoice{23} % any value between 1 and 29, default 1
\def\chapterpagesidenotesmargin{-.5} % fix the vertical position of sidenotes in chapter pages
%
%%% BOOK SPECIFIC FILES
\def\bibliographystylefile{utphysModified}
\def\bibliographyfile{collectiveReferenceLibrary}
%
%%% BOOK SPECIFIC PACKAGES
\def\Packages{\usepackage{musicography,slashed,multicol,pifont}}
\RequirePackage[nointegrals]{wasysym} % for astronomical symbols
\newcommand{\cmark}{\ding{51}}%
\newcommand{\xmark}{\ding{55}}%
%
%%% BOOK SPECIFIC DEFINITIONS
\def\Flat{{\musFlat{}}}
\def\Sharp{{\musSharp{}}}
\newcommand\draftnote[1]{{\color{blue} #1}}
\newcommand\dplus{\mathbin{+\mkern-2mu+}}
\def\naive{naïve }
\def\naively{naïvely }
\newcommand{\equref}[1]{eqn.~\eqref{#1}}
\newcommand{\secref}[1]{section~#1}
\newcommand{\figref}[1]{Fig.~(#1)}
\newcommand{\tabref}[1]{Table~(#1)}
\def\tA{\texttt{A}}
\def\tB{\texttt{B}}
\def\tC{\texttt{C}}
\def\tD{\texttt{D}}
%
%%% THE BOOK
%%% CLASS OPTIONS
\def\yes{yes}
\ifx\showtrimss\yes%
\documentclass[11pt,oneside,final,openany,onecolumn,showtrims]{memoir}%
\else%
\documentclass[11pt,oneside,final,openany,onecolumn]{memoir}%
\fi
%%% CHAPTER STYLE
\ifnum\chapterstylechoice=2 		\chapterstyle{section}
\else\ifnum\chapterstylechoice=3 	\chapterstyle{hangnum}
\else\ifnum\chapterstylechoice=4 	\chapterstyle{companion}
\else\ifnum\chapterstylechoice=5 	\chapterstyle{article}
\else\ifnum\chapterstylechoice=6 	\chapterstyle{reparticle}
\else\ifnum\chapterstylechoice=7 	\chapterstyle{bianchi}
\else\ifnum\chapterstylechoice=8 	\chapterstyle{bringhurst}
\else\ifnum\chapterstylechoice=9 	\chapterstyle{brotherton}
\else\ifnum\chapterstylechoice=10 	\chapterstyle{chappell}
\else\ifnum\chapterstylechoice=11	\chapterstyle{crosshead}
\else\ifnum\chapterstylechoice=12	\chapterstyle{culver}
\else\ifnum\chapterstylechoice=13	\chapterstyle{dash}
\else\ifnum\chapterstylechoice=14	\chapterstyle{demo2}
\else\ifnum\chapterstylechoice=15 	\chapterstyle{demo3}
\else\ifnum\chapterstylechoice=16	\chapterstyle{dowding}
\else\ifnum\chapterstylechoice=17	\chapterstyle{ell}
\else\ifnum\chapterstylechoice=18	\chapterstyle{ger}
\else\ifnum\chapterstylechoice=19	\chapterstyle{komalike}
\else\ifnum\chapterstylechoice=20	\chapterstyle{lyhne}
\else\ifnum\chapterstylechoice=21	\chapterstyle{madsen}
\else\ifnum\chapterstylechoice=22	\chapterstyle{ntglike}
\else\ifnum\chapterstylechoice=23	\chapterstyle{pedersen}
\else\ifnum\chapterstylechoice=24	\chapterstyle{southall}
\else\ifnum\chapterstylechoice=25	\chapterstyle{tandh}
\else\ifnum\chapterstylechoice=26	\chapterstyle{thatcher}
\else\ifnum\chapterstylechoice=27	\chapterstyle{veelo}
\else\ifnum\chapterstylechoice=28	\chapterstyle{verville}
\else\ifnum\chapterstylechoice=29	\chapterstyle{wilsondob}
\fi\fi\fi\fi\fi\fi\fi\fi\fi\fi\fi\fi\fi\fi\fi\fi\fi\fi\fi\fi\fi\fi\fi\fi\fi\fi\fi\fi
%
%%% FONT OPTIONS
\usepackage[utf8]{inputenc}
\usepackage[T1]{fontenc}
\usepackage[oldstyle]{libertine}
\usepackage{libertinust1math}
%
%%% ADDITIONAL PACKAGES
\Packages
%
%%% NUMBERING OPTIONS
\setsecnumdepth{subsection}
%
%%% SET PAPER SIZE
\setstocksize{297mm}{210mm} %A4 paper
\settrims{18mm}{18mm} % we trim 18mm from every direction. We chose this value at the very end, once we established the lengths of text block and side margin
\settrimmedsize{261mm}{174mm}{*} % we trim 18mm from every direction
%% Length of the text-block
\settypeblocksize{233mm}{26pc}{*} % the height is maximized at the end
%133pt, for 66pts, it is ~26pc
% This is determined by the length of alphabet and Table 2.2 of https://mirrors.evoluso.com/CTAN/macros/latex/contrib/memoir/memman.pdf. The length of alphabet is determined by running the following command after begin document:
%\newlength{\mylen} % a length
%\newcommand{\alphabet}{qwertyuiopasdfghjklzxcvbnm} % the lowercase alphabet
%\begingroup % keep font change local
% font specification e.g., \Large\sffamily
%\settowidth{\mylen}{\alphabet}
%The length of this alphabet is \the\mylen. % print in document
%\typeout{The length of the Large sans alphabet is \the\mylen} % put in log file
%\endgroup % end the grouping
%
% By the same method, we also determined sidenotes should be around ~14pc (for ~44pt)
\setsidebars{2\marginparsep}{14pc}{\onelineskip}{0pt}{\footnotesize\normalfont}{\textheight}
%
%%% MARGIN SETUP
\setlrmargins{0mm}{*}{*} % no additional margin from left
\setulmargins{*}{*}{1} % the top and bottom margins are devided equally
\setheadfoot{\onelineskip}{2\onelineskip}
\setheaderspaces{*}{2\onelineskip}{*}
%Make sidenotes use hyphenation
\usepackage{ragged2e}
%\renewcommand*{\sidebarform}{\ifmemtortm\RaggedRight\else\RaggedLeft\fi}
\renewcommand*{\sidebarform}{\justifying}
%
%%% FINALIZE PAGE SETUP
\checkandfixthelayout 
%
%%% FULL WIDTH PAGE FORMAT
\newlength{\fullwidthlen}
\setlength{\fullwidthlen}{\marginparwidth}
\addtolength{\fullwidthlen}{\marginparsep}

\def\beginfullwidth#1\endfullwidth{%
\checkoddpage\ifoddpage\strictpagecheck
\begin{adjustwidth*}{0pt}{-\fullwidthlen}%
	#1
\end{adjustwidth*}%
\else\strictpagecheck
\begin{adjustwidth*}{-\fullwidthlen}{0pt}%
	#1
\end{adjustwidth*}%
\fi
}

\def\begincenter#1\endcenter{%
\calccentering{\unitlength}
\checkoddpage\ifoddpage\strictpagecheck
\begin{adjustwidth*}{\unitlength}{-\unitlength}%
	#1
\end{adjustwidth*}%
\else\strictpagecheck
\begin{adjustwidth*}{-\unitlength}{\unitlength}%
	#1
\end{adjustwidth*}%
\fi
}



%
%%% ENSURE PARAGRAPHS ARE LEAST SPLIT BETWEEN PAGES
\clubpenalty=10000
\widowpenalty=10000
\raggedbottom
%
%%% SETUP SIDE NOTES
\usepackage{alphalph}
\makeatletter
\newalphalph{\alphmult}[alph]{\@alph}{26}
\makeatother
\newcounter{myfootnote}
\setcounter{myfootnote}{0}
\renewcommand{\themyfootnote}{\alphmult{\value{myfootnote}}}
%
\newcommand{\note}[1]{%
\refstepcounter{myfootnote}%
\hyperref[footnoteinternal:{\themyfootnote}]{${}^{\themyfootnote}$}%
\sidebar{\label{footnoteinternal:{\themyfootnote}}${}^{\themyfootnote}$#1}%
}
%
\let\oldfootnote\footnote
\renewcommand{\footnote}{\note}
%
\let\oldfootnotemark\footnotemark
\renewcommand{\footnotemark}{%
\refstepcounter{myfootnote}%
\hyperref[footnoteinternal:{\themyfootnote}]{${}^{\themyfootnote}$}}
%
\newcommand{\notetext}[1]{%
\sidebar{\label{footnoteinternal:{\themyfootnote}}${}^{\themyfootnote}$#1}%
}
\let\oldfootnotetext\footnotetext
\renewcommand{\footnotetext}{\notetext}
%
\newcommand{\sidefigure}[3]{%
\sidebar{\centering #3\captionof{figure}[#1]{#2}
}}
%
\newcommand{\sidetable}[3]{%
	\sidebar{\centering \captionof{table}[#1]{#2} #3
}}
%
%%% BOTTOM NOTES
\newcommand{\bottomnote}[1]{%
	\refstepcounter{myfootnote}%
	\renewcommand{\thefootnote}{\themyfootnote}%
	\oldfootnote{#1}%
}
%
\newcommand{\bottomnotemark}{%
	\refstepcounter{myfootnote}%
	\renewcommand{\thefootnote}{\themyfootnote}%
	\oldfootnotemark}
%
\newcommand{\bottomnotetext}[1]{%
	\renewcommand{\thefootnote}{\themyfootnote}%
	\oldfootnotetext{#1}}
%
%%% ARRANGE SIDEBAR MARGINS ON CHAPTER PAGES
\let\chapterTemporary\chapter
\renewcommand{\chapter}[1]{%
\chapterTemporary{#1}%
\sidebar{\vspace*{\chapterpagesidenotesmargin\baselineskip}}%
}
%
%%% MAKE SURE THERE IS NO FULL-PAGE FLOATS
\renewcommand{\topfraction}{1}
\renewcommand{\textfraction}{0}
\setfloatlocations{figure}{t}
\setfloatlocations{table}{t}
\suppressfloats[p]
%
%%% TITLE SETUP
\pretitle{\begin{flushleft}\Huge\sffamily}
\posttitle{\par\end{flushleft}\vskip 3em}
\preauthor{\begin{center}\Large}
\postauthor{\end{center}\vskip 3em}
\predate{\begin{flushright}\large\scshape}
\postdate{\par\end{flushright}\vskip 3em}	
%
\newcommand{\published}[1]{%
	\gdef\puB{#1}}
\newcommand{\puB}{}
\renewcommand{\maketitlehooka}{%
	\par\noindent \puB}
%
\published{\publishnote}
\title{\titlee}
\author{\authorr}
\date{\datee}
%
%%% SET LANGUAGE SPECIFIC DATA
\usepackage[\lang]{babel}
% One can change the titles as follows if so wishes:
\addto\captionsenglish{%
\renewcommand{\abstractname}{Summary}%
%\renewcommand{\prefacename}{Preface}%
%\renewcommand{\refname}{References}%
%\renewcommand{\bibname}{Bibliography}%
%\renewcommand\contentsname{Contents}%
%\renewcommand{\listfigurename}{LIST OF FIGURES}%
%\renewcommand{\chaptername}{Chapter}%
%\renewcommand{\appendixname}{Appendix}%
}
%
%\addto\captionsturkish{%
%	\renewcommand{\abstractname}{Summary}%
%	\renewcommand{\prefacename}{Preface}%
%	\renewcommand{\refname}{Refereces}%
%	\renewcommand\contentsname{Contents}%
%	\renewcommand{\listfigurename}{LIST OF FIGURES}%
%}
%
%%% USEFUL PACKAGES
\usepackage{amsmath,amssymb,amsfonts,mathtools,bm} % Default math packages
\numberwithin{equation}{\numberwithinn}
\usepackage{physics} % Default physics package
\usepackage[margin=10pt,font=small,labelfont=bf]{caption}
\usepackage[square, comma, sort&compress,numbers]{natbib} 
\usepackage[pdftex,dvipsnames]{xcolor}
\definecolor{darkblue}{rgb}{0.1,0.1,.7}
\usepackage[colorlinks, linkcolor=darkblue, citecolor=darkblue, urlcolor=darkblue, linktocpage]{hyperref} 
\usepackage{graphicx}  % Allows including images
\graphicspath{ {images/} }
%
%%% CURLY L IN MATH MODE
\mathcode`l="8000
\begingroup
\makeatletter
\lccode`\~=`\l
\DeclareMathSymbol{\lsb@l}{\mathalpha}{letters}{`l}
\lowercase{\gdef~{\ifnum\the\mathgroup=\m@ne \ell \else \lsb@l \fi}}%
\endgroup
%
%%% SEVERAL MACROS
\def\<#1\>{\expval{#1}}
\newcommand   \rdr  [2] {\frac{\dd #1}{\dd #2}}
\newcommand   \pdr  [2] {\frac{\partial #1}{\partial #2}}
\newcommand   \tl  [1] {\widetilde{#1}}
\newcommand{\undertl}[1]{\mathord{\vtop{\ialign{##\crcr
				$\hfil\displaystyle{#1}\hfil$\crcr\noalign{\kern1.5pt\nointerlineskip}
				$\hfil\tilde{}\hfil$\crcr\noalign{\kern1.5pt}}}}} % Puts tilde under given variable
% \def\blfootnote{\gdef\@thefnmark{}\@footnotetext} %FOOTNOTE WITHOUT NUMBER OR SYMBOL
\renewcommand \.   {\cdot}
\newcommand   \x   {\times}
\newcommand   \aeq {\thickapprox}
\newcommand   \mto {\mapsto}
\newcommand   \half{\frac 1 2}
\newcommand   \lptl{\raise .8ex\hbox{$^\leftarrow$} \hspace{-9pt} \partial} % derivatives with arrows
\newcommand   \lrptl{\raise .8ex\hbox{$^\leftrightarrow$} \hspace{-9pt} \partial} % derivatives with arrows
%
%%% EQUATION COMMANDS
\newcommand \nn {\nonumber}
\def\be#1\ee{\begin{equation}\begin{aligned}#1\end{aligned}\end{equation}}
\makeatletter
\DeclareRobustCommand\bea{\@ifnextchar[{\@@bea}{\@bea}}
\def\@@bea[#1]#2\eea{\begin{subequations}\begin{align}#2\end{align}\label{#1}\end{subequations}}
\def\@bea#1\eea{\begin{subequations}\begin{align}#1\end{align}\end{subequations}}
\makeatother
%
%%% RENDERING HYPERGEOMETRIC FUNCTION 
 \newcommand*\pFqskip{8mu}
 \catcode`,\active
 \newcommand*\pFq{\begingroup
		\catcode`\,\active
		\def ,{\mskip\pFqskip\relax}%
		\dopFq
	 }
 \catcode`\,12
 \def\dopFq#1#2#3#4#5{%
		{}_{#1}F_{#2}\biggl[\genfrac..{0pt}{}{#3}{#4};#5\biggr]%
		\endgroup
	 }
%
%%% GREEK LETTERS
\renewcommand \a  {\alpha}
\renewcommand \b  {\beta}
\newcommand   \g  {\gamma}
\newcommand   \G  {\Gamma}
\newcommand   \de {\delta}
\newcommand   \De {\Delta}
\newcommand   \e  {\epsilon}
\renewcommand \l  {\lambda}
\renewcommand \L  {\Lambda}
\renewcommand \r  {\rho}
\newcommand   \f  {\phi}
\newcommand   \vf {\varphi}
\newcommand   \s  {\sigma}
\newcommand   \w  {\omega}
\renewcommand \O  {\Omega}
\renewcommand \th {\theta}
\newcommand   \ka {\kappa}
\newcommand   \z  {\zeta}
%
%%% BOLD LETTERS
% some letters are commented out to not conflict with the usual TeX commands
%% lowercase
\newcommand   \ba {\mathbf{a}}
\newcommand   \bb {\mathbf{b}}
\newcommand   \bc {\mathbf{c}}
\newcommand   \bd {\mathbf{d}}
%\newcommand   \be {\mathbf{e}}
%\newcommand   \bf {\mathbf{f}}
\newcommand   \bg {\mathbf{g}}
\newcommand   \bh {\mathbf{h}}
\newcommand   \bi {\mathbf{i}}
\newcommand   \bj {\mathbf{j}}
\newcommand   \bk {\mathbf{k}}
\newcommand   \bl {\mathbf{l}}
%\newcommand   \bm {\mathbf{m}}
\newcommand   \bn {\mathbf{n}}
\newcommand   \bo {\mathbf{o}}
\newcommand   \bp {\mathbf{p}}
\newcommand   \bq {\mathbf{q}}
\newcommand   \br {\mathbf{r}}
\renewcommand   \bs {\mathbf{s}}
\newcommand   \bt {\mathbf{t}}
\newcommand   \bu {\mathbf{u}}
\newcommand   \bv {\mathbf{v}}
\newcommand   \bw {\mathbf{w}}
\newcommand   \bx {\mathbf{x}}
\newcommand   \by {\mathbf{y}}
\newcommand   \bz {\mathbf{z}}
%% Uppercase
\newcommand   \bA {\mathbf{A}}
\newcommand   \bB {\mathbf{B}}
\newcommand   \bC {\mathbf{C}}
\newcommand   \bD {\mathbf{D}}
\newcommand   \bE {\mathbf{E}}
\newcommand   \bF {\mathbf{F}}
\newcommand   \bG {\mathbf{G}}
\newcommand   \bH {\mathbf{H}}
\newcommand   \bI {\mathbf{I}}
\newcommand   \bJ {\mathbf{J}}
\newcommand   \bK {\mathbf{K}}
\newcommand   \bL {\mathbf{L}}
\newcommand   \bM {\mathbf{M}}
\newcommand   \bN {\mathbf{N}}
\newcommand   \bO {\mathbf{O}}
\newcommand   \bP {\mathbf{P}}
\newcommand   \bQ {\mathbf{Q}}
\newcommand   \bR {\mathbf{R}}
\newcommand   \bS {\mathbf{S}}
\newcommand   \bT {\mathbf{T}}
\newcommand   \bU {\mathbf{U}}
\newcommand   \bV {\mathbf{V}}
\newcommand   \bW {\mathbf{W}}
\newcommand   \bX {\mathbf{X}}
\newcommand   \bY {\mathbf{Y}}
\newcommand   \bZ {\mathbf{Z}}
%
%%% SPACES AND FIELDS
\newcommand   \R  {\mathbb{R}}
\newcommand   \Z  {\mathbb{Z}}
\newcommand   \N  {\mathbb{N}}
\newcommand   \C  {\mathbb{C}}
\newcommand   \Q  {\mathbb{Q}}
\newcommand   \Qp {\mathbb{Q}_p}
\renewcommand \P  {\mathbb{P}}
\newcommand   \A  {\mathbb{A}}
%
%%% MULTI-LETTER FUNCTORS/FUNCTIONS
\newcommand   \Ad    {\mathrm{Ad}}
\newcommand   \SU    {\mathrm{SU}}
\newcommand   \SO    {\mathrm{SO}}
\newcommand   \ISO    {\mathrm{ISO}}
\newcommand   \GL    {\mathrm{GL}}
\newcommand   \SL    {\mathrm{SL}}
\newcommand   \PSL   {\mathrm{PSL}}
\newcommand   \PGL   {\mathrm{PGL}}
\newcommand   \Sp    {\mathrm{Sp}}
\newcommand   \Span  {\mathrm{Span}}
\newcommand   \Aut   {\mathrm{Aut}}
\renewcommand \Re    {\mathop{\mathrm{Re}}}
\renewcommand \Im    {\mathop{\mathrm{Im}}}
\newcommand   \Pf    {\mathrm{Pf}}
\newcommand   \Ker   {\mathop{\mathrm{Ker}}}
\newcommand   \Coker {\mathop{\mathrm{Coker}}}
\newcommand   \Hom   {\mathrm{Hom}}
\newcommand   \Mor   {\mathrm{Mor}}
\newcommand   \Gal   {\mathrm{Gal}}
\newcommand   \Sym   {\mathrm{Sym}}
\newcommand   \codim {\mathop{\mathrm{codim}}}
\newcommand   \ind   {\mathop{\mathrm{ind}}}
\newcommand   \Vol   {\mathrm{Vol}}
\newcommand   \vol   {\mathrm{vol}}
\newcommand   \diag  {\mathrm{diag}}
%
%%% CALIGRAPHIC CAPITALS
\newcommand   \cA {\mathcal{A}}
\newcommand   \cB {\mathcal{B}}
\newcommand   \cC {\mathcal{C}}
\newcommand   \cD {\mathcal{D}}
\newcommand   \cE {\mathcal{E}}
\newcommand   \cF {\mathcal{F}}
\newcommand   \cG {\mathcal{G}}
\newcommand   \cH {\mathcal{H}}
\newcommand   \cI {\mathcal{I}}
\newcommand   \cJ {\mathcal{J}}
\newcommand   \cK {\mathcal{K}}
\newcommand   \cL {\mathcal{L}}
\newcommand   \cM {\mathcal{M}}
\newcommand   \cN {\mathcal{N}}
\newcommand   \cO {\mathcal{O}}
\newcommand   \cP {\mathcal{P}}
\newcommand   \cQ {\mathcal{Q}}
\newcommand   \cR {\mathcal{R}}
\newcommand   \cS {\mathcal{S}}
\newcommand   \cT {\mathcal{T}}
\newcommand   \cU {\mathcal{U}}
\newcommand   \cV {\mathcal{V}}
\newcommand   \cW {\mathcal{W}}
\newcommand   \cX {\mathcal{X}}
\newcommand   \cY {\mathcal{Y}}
\newcommand   \cZ {\mathcal{Z}}
%

%%%%%%% Fractur Capitals %%%%%%%
\newcommand   \fA {\mathfrak{A}}
\newcommand   \fB {\mathfrak{B}}
\newcommand   \fC {\mathfrak{C}}
\newcommand   \fD {\mathfrak{D}}
\newcommand   \fE {\mathfrak{E}}
\newcommand   \fF {\mathfrak{F}}
\newcommand   \fG {\mathfrak{G}}
\newcommand   \fH {\mathfrak{H}}
\newcommand   \fI {\mathfrak{I}}
\newcommand   \fJ {\mathfrak{J}}
\newcommand   \fK {\mathfrak{K}}
\newcommand   \fL {\mathfrak{L}}
\newcommand   \fM {\mathfrak{M}}
\newcommand   \fN {\mathfrak{N}}
\newcommand   \fO {\mathfrak{O}}
\newcommand   \fP {\mathfrak{P}}
\newcommand   \fQ {\mathfrak{Q}}
\newcommand   \fR {\mathfrak{R}}
\newcommand   \fS {\mathfrak{S}}
\newcommand   \fT {\mathfrak{T}}
\newcommand   \fU {\mathfrak{U}}
\newcommand   \fV {\mathfrak{V}}
\newcommand   \fW {\mathfrak{W}}
\newcommand   \fX {\mathfrak{X}}
\newcommand   \fY {\mathfrak{Y}}
\newcommand   \fZ {\mathfrak{Z}}
%%% START THE DOCUMENT
\bibliographystyle{\bibliographystylefile}
\begin{document}
\setlength{\droptitle}{40pt}
\begin{titlingpage}
\calccentering{\unitlength}
\begin{adjustwidth*}{\unitlength}{-\unitlength}
\maketitle
\begin{abstract}
\abstractt
\end{abstract}
\end{adjustwidth*}
\end{titlingpage}
\addtocontents{toc}{\protect\setcounter{tocdepth}{2}}
\frontmatter
\chapter{Preface}
\sidebar{\vspace*{\baselineskip}}
\section{Remarks to the reader}
I would like to make a few points crystal clear:
\begin{enumerate}
	\item I do not update these notes on a regular basis.
	\item The contents are correct to the best of my knowledge, but I have not put the extra effort to make sure that everything is book-level correct; nevertheless, I try to put as many references as possible when relevant, so please make the most of it!
	\item I believe that the level of this book is appropriate for an average sophomore, but I would dare say that it should be quite useful even for graduate students of theoretical physics. 
	\item Lastly, I provide several links as references here and there: I agree that this is a bad practice in academia and one should instead convert them to proper references in the bibliography. Nevertheless, it is faster for me to write this way and faster for the reader to just click on them where they appear, so I'll keep this practice. My upfront apologies if the links get broken in time: hopefully a snapshot will have been available here \href{https://web.archive.org/}{https://\\web.archive.org/} (it would be rather amusing if this site itself becomes unavailable).
\end{enumerate}
\section{About the format of the book}
This book is publicly available online, and anyone can simply download and read it on their computer. Nevertheless, it is a lot easier on eyes to read a book on paper (or epaper), so I suspect many readers will simply print this book. To make this book convenient for all types of readers, I need to make a few design choices\footnote{For instance, I provide all links in their explicit form (not like \href{https://www.google.com/}{google} but as \href{https://www.google.com/}{https://www.google.com/}) so that the book on an ereader without a browser or its printed version is as useful as its digital version.} and the most important of it is the format of this book.

The most convenient paper that most students have easy access to is A4 (or letterpaper which is very similar in size): so if they choose to print this book, their easiest and cheapest option would be to use A4 paper. However that paper has been historically designed for typewriters which use large monospaced fonts, hence is not really appropriate for a digitally prepared book. Indeed, if you check your favorite-to-read book, you will most likely see that it uses a smaller paper size, with proportionally spaced small fonts.

\beginfullwidth
\begin{SingleSpace}
\hphantom{\indent} What is wrong with using a large paper? It is empirically known that a document is properly legible if there are around 60-75 characters on a line: if there are more characters, it becomes harder to read and the reader may end up re-reading same line over and over again (doubling). When used with typewriter, A4 paper indeed has appropriate number of characters one a line, but as stated previously, this is not the optimal setup with the digital fonts, hence making A4 paper \emph{too large for digital books}.
\\\\
{\ttfamily \hphantom{\indent} How to solve the problem that A4 is too large for a digital book?\newline Obviously, we can use a typewriter font for which A4 is historically\newline intended in the first place! However, such monospaced fonts (Courier\newline being another example) are not aesthetically pleasing and do not\newline belong to modern texts!}
\\\\
{\LARGE \hphantom{\indent} Of course, we can go with a modern proportionally spaced fonts but make the font size large enough such that a line has few enough characters for it to be easily legible. Although a better one than using monospaced fonts, this is still a suboptimal solution to the problem at hand...}
\end{SingleSpace}
%
\begin{DoubleSpace}
	Another solution which is somehow popular around the institutions is to use double-spacing among the lines. Indeed, regulations for master and doctorate thesis of various universities include compulsory large spacing among the lines, such as one and a half spacing or double spacing: you can also see this in my master and doctorate thesis: \href{https://arxiv.org/abs/1602.07676}{https://arxiv.org/abs/1602.07676}, and \href{https://arxiv.org/abs/2107.13601}{https://arxiv.org/abs/2107.13601}. Although this method can indeed prevent doubling to a degree, it is neither an aesthetic nor an efficient solution.
\end{DoubleSpace}
\vspace*{-\baselineskip}
	\begin{multicols}{2}
	A somehow better solution than those listed above is to use multiple columns in the document. Indeed, this is the traditional approach in magazines and newspapers, and is immediately applicable in academic papers and manuscripts as well. However, A4 paper is not really big enough to have two columns of text with around 60-75 characters (let alone three or more columns), and although one can go with smaller font sizes to make it more legible digitally, the printed version would still be hard to read either way (proper number of small font characters, or few number of normal font characters).
\end{multicols}
\endfullwidth 
\begincenter
\qquad Common text editors such as Microsoft Word either go with one of the solutions above or do not solve the problem at all. On the contrary, \LaTeX\;templates default to choose another approach: they stick to a modern font with an appropriate spacing and a single column, but they also increase the margins such that a line has proper number of characters. This is an ideal solution if the resultant text will be read digitally, however it leads to a waste of paper when printed.
\endcenter
\vspace*{\baselineskip}
\sidebar{\vspace*{50\baselineskip}}
\hphantom{\indent} In this book, we will not follow any of these design choices. Instead, we will go with the rather unorthodox \emph{Tufte style},\footnote{See \href{https://www.ctan.org/tex-archive/macros/latex/contrib/tufte-latex/}{https://www.ctan.org/tex-archive\\/macros/latex/contrib/tufte-latex/}} an asymmetric allocation of the text in the paper. Indeed, the main text will be in the left of the paper, whereas we have another block of text on the right dedicated to the \emph{sidenotes},\footnote{In the traditional layout, one usually uses endnotes, margin notes, or footnotes; in this paper, most ``notes'' will be sidenotes with occasional footnotes.\bottomnotemark}\bottomnotetext{Such as this one.} margin figures, and margin tables. We choose a rather narrow font family (\emph{libertine}) and arrange the margins such that the main text is of 26 pica width and side text is of 14 pica width: for the 11pt and 9pt font sizes, this corresponds to roughly 66 and 44 characters of the libertine font for main and side text blocks respectively.\footnote{For a nice discussion of these points along with the tools to compute approximate expected number of characters per line, see \href{https://ftp.cc.uoc.gr/mirrors/CTAN/macros/latex/contrib/memoir/memman.pdf}{https://ftp.cc.uoc.gr/mirrors/CTAN/macros\\/latex/contrib/memoir/memman.pdf}} Thus we have an ideally-sized main text block and acceptably-sized side text block for a modern proportionally spaced font in an A4 paper, and we do this without unnecessarily wasting the paper.\footnote{I would like to acknowledge the following nice discussion with which I started to learn more about these \emph{typographical} issues: \href{https://tex.stackexchange.com/questions/71172/why-are-default-latex-margins-so-big}{https://tex.stackexchange.com/questions/71172\\/why-are-default-latex-margins-so-big}.}

\section{About the courses Phys209 \& Phys210}
As stated in the front page, this ``book'' is collection of notes prepared for the courses Phys209 \& Phys210, with the syllabus provided here:\\ \hyperref{https://soneralbayrak.com/teaching}{}{}{https://soneralbayrak.com/teaching}. Although the title is somewhat generic, the contents of this book is severely restricted and arranged so as to be an appropriate two one-semester-long courses for \emph{an average sophomore at the Physics program of Middle East Technical University}. I would also like to acknowledge that the level and approach of this book is based solely on my expectations and projections for such a student, and thus it may not be really appropriate in real life; nevertheless, it is what it is.

I left several sources in the syllabus, including the textbooks of the courses on which these notes are somewhat based on. I'll also make use of other sources; yet, any error or incorrect information on these notes are entirely my fault and not of any of these sources.
\beginfullwidth
\clearpage
\tableofcontents
\clearpage
\listoftables
\clearpage
\listoffigures
\endfullwidth
\mainmatter
\part{Differential Equations in Physics}
\chapter{Definition and Classification of Differential Equations}
\section{Preliminaries: some basic terminology}

\subsection{Type notation and functions}
\label{sec: type notation and functions}
Consider the number $2$. It is an integer, which mathematicians denote as $2\in \Z$, where $\Z$ denotes the set of integers.\footnote{$\Z$ is actually an integral domain.} Computer scientists on the other hand would denote this as
\be 
2::\texttt{Integer}
\ee 
where $a::b$ reads as \emph{``a is of type b''}. Further examples would be
\bea 
3/2::{}&\texttt{Rational}\\
1.56::{}&\texttt{Real}\\
1+i::{}&\texttt{Complex}
\eea 
Note that an object may be of multiple types. Mathematically, $3\in\Z$ and $3\in\R$, meaning
\bea 
3::{}&\texttt{Integer}\\
3::{}&\texttt{Real}
\eea 
A somewhat hybrid notation between mathematicians and computer scientists would be 
\bea 
3::{}&\Z\\
3::{}&\R
\eea 
We shall use this notation in the rest of the book.\footnote{I personally find this notation clearer when we use it with higher order functions such as derivatives.} 

Just as we do with the explicit numbers above, we can \emph{define} variables with explicit types; for instance,
\be 
x::\Z
\ee 
It is up to us to choose what we want for the type, we can even left the type unknown; for instance,
\be 
y::\texttt{A}
\ee 
means that $y$ is a variable of the type \texttt{A},\footnote{This is called a \emph{type variable}.} where \texttt{A} can be anything.\footnote{It could be a simple field such as $\Z$ or $\N$, or it could be a more complex object such as $\mathfrak{M}_{2\x2}(\C)$ which denotes two by two matrices with complex entries.}

Unlike the numbers or the variables above, the functions have an input and an output, hence their type actually reads differently.\footnote{
	Physicists tend to refer to multi-valued relations as functions as well: this is a justifiable habit as such relations can always be treated as genuine functions by appropriately restricting their domains.\footnotemark We will stick to this convention in the rest of the book and  refer all multi-valued relations (such as an arctan) as functions.
}
\footnotetext{
	Mathematically, a function yields a unique output for a given input, therefore so-called multi-valued ``functions'' are not really functions in their full analytic domain. For instance, the relation \mbox{$\texttt{sqrt}=\lambda\to\sqrt{\lambda}$} is not a function in the complete complex plane, as \mbox{$\texttt{sqrt}(4)=\pm 2$}. One solution is to choose a \emph{restricted domain} so that the relation actually yields a unique solution for a given input from the domain, hence making the relation a genuine function, e.g. choosing the domain $\R^+$ for \texttt{sqrt}. In principle, we do not need to make an arbitrary restriction: the strategy would be to analyze the \emph{Riemann surface} of the relation, and then determine the codomain in which the relation yields a unique result; in the case of \texttt{sqrt}, we can state
	\mbox{$\texttt{sqrt}::\C\to \texttt{A}$} where $x\in\texttt{A}$ if and only if\; $0\le\arg(x)<\pi$. This means that $\texttt{sqrt}\left(re^{i\theta}\right)=\sqrt{r}e^{i\theta/2}$ for $\theta$ chosen in the range $0\le\theta< 2\pi$ with $r>0$; hence, $\texttt{sqrt}(4)=2$.
	
	The regions of the codomain in which the multi-valued relation becomes a genuine function are called \emph{sheets}; in the example above, we choose a principle sheet (or a first sheet) for the relation \texttt{sqrt}: we can move on to the \emph{other sheets} by removing the restriction on $\theta$. Indeed, we have on the second sheet $\texttt{sqrt}(4)=\texttt{sqrt}\left(4e^{i2\pi}\right)=\sqrt{4}e^{i\pi}=-2$, the other solution! One could go on to higher sheets to find even more solutions; in the case of \texttt{sqrt}, $n-$th sheet is actually identified with $(n-2)-$th sheet, hence we have only two solutions (as expected from a square root operation).
	
	Somewhat more traditional approach to the Riemann surfaces is the \emph{analysis of branch cuts}. We \textbf{(1)} take one of the solutions of the relation as the output (called \emph{principal value}), \textbf{(2)} determine some lines on the complex plane (branch cuts), \textbf{(3)} impose discontinuity on the cuts such that the relation is a true function in the rest of the complex plane! With the insight from Riemann surfaces, we know that moving across such lines actually takes us from one sheet to another ---previous (next) sheet if we pass the branch cut (counter)clockwise. For \texttt{sqrt}, the conventionally chosen principle value is $\sqrt{r^2}=r$ for $r\in\R^+$, and branch cut is the line $(-\infty,0)$: $\texttt{sqrt}(z)$ for any other $z\in\C$ can then be uniquely determined to be consistent with these; for instance $\texttt{sqrt}\left(-1\pm i 10^{-100}\right)\sim 6\x 10^{-17}\pm i$ ---note the jump!
} For instance,
\be 
f::\Z\to\Z
\ee 
denotes \emph{``a function that acts on integers and produces another integer}. An example would be
\be 
f = \l \to \l^2
\ee 
which gives the integer $f(x)=x^2$ when acted on the integer $x$.\footnote{I'd like to note that there is a common misconception (especially) in the physics community. $f(x)$ is \emph{not} the function, the function is $f$. $f$ acts on the input $x$, and produces the output $f(x)$.} Another example would be 
\be 
g(y)=y/3
\ee 
for which we can write down\footnote{If you couldn't remember, $\Q$ denotes the set of rational numbers.}
\bea 
g::{}&\Z\to\Q\\
y::{}&\Z\\
g(y)::{}&\Q
\eea 
Of course, we can also extend our interested regime for the input $y$ and simply state
\bea 
g::{}&\C\to\C\\
y::{}&\C\\
g(y)::{}&\C
\eea 
which is still true for $g(y)=y/3$. In fact, we can even write $g::{}\texttt{A}\to\texttt{B}$ if we do not care for the explicit types of input and output.\footnote{We use different letters for the type variables (\texttt{A} and \texttt{B}) so that the input and output are not necessarily of the same type. On the contrary, the function $h::{}\texttt{A}\to\texttt{A}$ can only produce integers when acted on integers, reals when acted on reals, and so on.}

\subsection{Higher order functions and the derivative}
Consider the operation $T$ of \emph{``doubling the output of a function''}. If we apply this operation $T$ to a function $f$, then it yields a function $g$ such that $g(x)=2 f(x)$. For instance,
\bea 
f={}&\l\to\l+5\\
g={}&\l\to2\l+10
\eea 
The question now is this: what is the type of the operation $T$?

Clearly, $T=f\to g$ as it takes the function $f$ as an input and produces the function $g$ as the output. Thus, we can write it as 
\be 
T::{}&\left(\texttt{A}\to\texttt{B}\right)\to\left(\texttt{C}\to\texttt{D}\right)
\ee 
which means if 
\be 
f::{}&\texttt{A}\to\texttt{B}
\ee 
then 
\be 
\left(g=T\.f\right)::{}&\texttt{C}\to\texttt{D}
\ee 
$T$ is called \emph{a higher order function}: it acts on a function and produces another function.

The derivative operator \emph{is} a higher order function, i.e.
\be 
\label{eq: type of derivative operator}
\rdr{}{x}::{}&\left(\texttt{A}\to\texttt{B}\right)\to\left(\texttt{A}\to\texttt{C}\right)
\ee 
which means\footnote{We are using the common convention $f'\coloneqq \rdr{f}{x}$ and $f'(x)\coloneqq \rdr{f}{x}(x)$ for brevity.}
\bea 
x::{}&\texttt{A}\\
f::{}&\texttt{A}\to\texttt{B}\\
f(x)::{}&\texttt{B}\\
f'::{}&\texttt{A}\to\texttt{C}\\
f'(x)::{}&\texttt{C}
\eea  
For example, 
\bea
f::{}&\R\to\C\\
f={}&\l\to\l^2+2i
\eea
leads to
\bea
f'::{}&\R\to\R\\
f'={}&\l\to 2\l
\eea
where the type variables in \equref{eq: type of derivative operator} are $\texttt{A}=\texttt{C}=\R$ and $\texttt{B}=\C$.

The derivatives can shrink the codomain of a function;\footnote{Reminder: if $f=\texttt{A}\to\texttt{B}$, we call $A$ ($B$) the (co)domain of $f$.} in the above example, the original codomain (that of $f$) was $\C$ whereas the new codomain (that of $f'$) is $\R$. Nevertheless, we can always \emph{embed} the smaller codomain into a larger one (e.g. all real numbers can be considered as complex numbers as well), hence we can always take \mbox{$\rdr{}{x}::{}\left(\texttt{A}\to\texttt{B}\right)\to\left(\texttt{A}\to\texttt{B}\right)$}. This shows that the derivative is a higher order function that can be \emph{repeatedly applied}; thus, we say\footnote{We will use the notation such that $f^{(n)}$ is the $n-$th derivative of the function $f$.}
\bea 
\label{eq: type of nth order derivatives}
\rdr{{}^n}{x^n}::{}&\left(\texttt{A}\to\texttt{B}\right)\to\left(\texttt{A}\to\texttt{B}\right)\\
f::{}&\texttt{A}\to\texttt{B}\\
f^{(n)}::{}&\texttt{A}\to\texttt{B}
\eea  

\subsection{Functionals and the integral}
In the previous section, we have seen that the derivative is a higher-order function, i.e. it takes a function to another function. Naturally, its inverse is also a higher-order function:\footnote{In principle, the anti-derivative can \emph{extend} the codomain of a function, just as derivative shrinks it. We can see this via \emph{integration constant}, which can be anything as long as it is $x-$independent. We put this subtlety aside as we can always extend the original codomain such that it matches the new one, hence \eqref{eq: anti-derivative}.}
\bea 
\rdr{{}^{-1}}{x^{-1}}::{}&\left(\texttt{A}\to\texttt{B}\right)\to\left(\texttt{A}\to\texttt{B}\right)\label{eq: anti-derivative}\\
g::{}&\texttt{A}\to\texttt{B}\\
g^{(-1)}::{}&\texttt{A}\to\texttt{B}
\eea  
where $g^{(-n)}=\left(\rdr{{}^{-1}}{x^{-1}}\right)\.g^{(1-n)}$ with $g^{(0)}=g$, in line with the notation for derivatives. Fundamental theorem of calculus then tells us that the output \mbox{$g^{(-1)}(x)::\texttt{B}$} can be written as
\be 
g^{(-1)}(x)=\int\limits_0^x g(t) dt
\ee 
which is compatible with $\rdr{}{x}g^{(-1)}(x)=g(x)$.

We have shown above that the \emph{the indefinite integral} is a higher order function, but how about a definite integral? How do we determine its type?

We can start by writing down a generic definite integral:
\be 
\int\limits_{0}^{\pi/2} \cos(x) dx= 1
\ee 
Clearly, we take a function ($\cos$) and a range over which we do the integration (between $0$ and $\pi/2$). We can always specify the integration range via the domain of the function,\footnote{
	For instance, if we would like to do the integration from $0$ to $1$, we can restrict the function $f(x)::\texttt{A}\to\texttt{B}$ to $f(x)::\texttt{UnitReal}\to\texttt{B}$ where $x::\texttt{UnitReal}$ means $x\in[0,1]$.
} thus
\be 
\int\limits :: (\texttt{A}\to\texttt{B})\to\texttt{C}
\ee 
as the integration turns the function cosine into a number $1$.

Operations that turn functions into numbers are called \emph{functionals}, and definite integration is a functional.  For instance, the operation to compute the area under a curve is a functional: if we call that operation $T$, we then have
\bea 
T::{}&\left(\R\to\R\right)\to\R\\
f::{}&\R\to\R\\
\left(T\.f = \int\limits_{-\infty}^\infty f(x)dx\right)::{}&\R
\eea  

Note that the parentheses in the type definition is important; for instance, 
\bea 
T::{}&\left(\R\to\R\right)\to\R\\
F::{}&\R\to\left(\R\to\R\right)
\eea 
denote different objects: $T$ is a functional, which produces a number if given a function as input. $F$ on the other hand produces a function when fed a number, i.e. $F(x)=f$ is a function, whose output can be written as $F(x)(y)=f(y)$. This show that we can actually interpret $F$ as a function of two variables!\footnote{
	This property can be generalized. A higher order function
	\be 
	f::\texttt{A}_1\to\left(
	\texttt{A}_2\to\left(\dots\to\left(
	\texttt{A}_{n-1}\to\left(\texttt{A}_{n}\to\texttt{B}
	\right)\right)\right)\right)
	\ee 
	produces a function of $n-1$ variable once given a variable as input. That function then produces another function of $n-2$ variable once given a variable as input, and so on. Indeed, it means
	\bea 
	x_i::{}&\texttt{A}_i\\
	f(x_1)(x_2)\dots(x_n)::{}&\texttt{B}
	\eea 
	which can easily be re-interpreted as $f(x_1,\dots,x_n)::\texttt{B}$.
	
	This concept of turning higher-order functions into functions of multiple variables (and vise versa) is called \emph{currying}, see bla bla bla. \draftnote{Put some sources here.}
}

\section{Differential equations}
\subsection{Basics}
\label{section: basics of diff eqn}
Very broadly, we could define any relation that contains the derivative higher order function $\rdr{}{x}$ and an unknown function $f$ as a differential equation. For instance,
\be 
\cos\left(\exp(\rdr{}{x})f(x)+\frac{1}{f(x)}\right)=0
\ee 
is a differential equation: but it is neither real-world motivated nor easy-to-solve, so let's skip it and focus on more relevant and simpler cases.\footnote{\label{footnote: exponentiated differential}
	You may be surprised with the expression $\exp(\rdr{}{x})$. To understand it, let's first view the taking-the-$n^{\text{th}}$-power operation as a higher order function:
	\be 
	P_n::{}&(\texttt{A}\to\texttt{B})\to(\texttt{A}\to\texttt{B})
	\ee 
	and
	\bea
	\left(P_0\.f=f\right)::{}&\texttt{A}\to\texttt{B}\\
	\bigg(P_n\.f=f\.(P_{n-1}\.f)\bigg)::{}&\texttt{A}\to\texttt{B}
	\eea
	meaning 
	\bea
	x::{}&\texttt{A}\\
	(P_n\.f)(x)=f(f(\dots f(x)))::{}&\texttt{B}
	\eea
	For instance, $P_2\.\cos= \l\to\cos(\cos(\l))$.
	
	We can now define \emph{exponentiation} as a higher order operation:
	\bea
	\exp::{}&(\texttt{A}\to\texttt{B})\to(\texttt{A}\to\texttt{B})\\
	\exp={}&\sum\limits_{n=0}^\infty\frac{1}{n!}P_n
	\eea
	One can then immediately compute, say,
	\be
	\exp(\rdr{}{x})x^3={}&x^3+3x^2+3x+1\\
	\exp(\rdr{}{x})e^{k x}={}&e^ke^{k x}
	\ee 
	and so on.
}

The simplest differential equation is
\bea 
x::{}&\R\\
\left(\rdr{}{x}\.f\right)::{}&\tA\to\tB\\
\label{eq:simplest dif equation}\rdr{}{x}\.f={}&0
\eea
which states that \emph{there is an unknown function $f$ such that ``the derivative higher order function acting on it'' leads to the zero function}.\footnote{
	We use the convention such that $0$ can be of any type that yields the ordinary number zero (\mbox{$0::\C$}) as the output. In \equref{eq:simplest dif equation}, $0$ has the type \mbox{$\tA\to(0::\C)$}, which we call \emph{the zero function}.
} You may hope to formally solve this equation by applying $\rdr{^{-1}}{x^{-1}}$ to the both sides and use $\rdr{^{-1}}{x^{-1}}\.\rdr{}{x}\.f=f$, but this actually leads to a circular argument.\footnote{
	Naively applying $\rdr{^{-1}}{x^{-1}}$ would lead to the equation $f=\rdr{^{-1}}{x^{-1}}\.0$ but this equation is not necessarily equivalent to the original one. Indeed, both  $f=\rdr{^{-1}}{x^{-1}}\.0$ and  $f=g+\rdr{^{-1}}{x^{-1}}\.0$ would lead to the original equation if $\rdr{}{x}\.g=0$ as well.
	\draftnote{burada kernel kavramindan, homojen ve hetetojen denklemlerden bahset.}
	
} Instead, let us proceed to apply this function to a real variable and write
\be
\left(\rdr{}{x}\.f\right)(x)\equiv\rdr{f}{x}(x)\equiv f'(x)=0
\ee
for which one usually writes down the result as
\be 
f(x)=\textrm{constant}
\ee 
immediately. This makes sense, as the derivative of a constant is always zero.

The next simplest example would be the following differential equation
\be 
\rdr{}{x}\.f=f
\ee 
for the unknown function $f$. Solving this equation is equivalent to answering this question: \emph{what function is equal to its derivative?}

Even though what we know and what we try to solve for are all \emph{functions}, the traditional way of writing down such equations is in terms of \emph{the values of functions}; in other words, we say
\be 
\label{eq: exponential diff}
f'(x)=f(x)
\ee 
is the differential equation, and we are trying to find the output $f(x)$ that satisfies this. Indeed, in the rest of the notes, we will mostly stick to this more traditional form.

Let us ask the question again: what is the function that is equal to its derivative? We will provide three equivalent answer.
\begin{enumerate}
	\item We \emph{define} a function as solution of this equation. Indeed, most of the famous mathematical functions (Hypergeometric, Bessel, Hankel, Gegenbauer, etc.) are \emph{defined} as solutions to various differential equations. Analogously, we define
	\bea 
	\exp::{}&\C\to\C\\
	\exp={}&x\to \exp(x)\text{ such that } \rdr{\exp(x)}{x}=\exp(x)
	\eea 
	We call this function \emph{exponential} and usually denote it as \mbox{$\exp(x)=e^x$}.\footnote{By using various numerical methods, we can compute the value of this function for arbitrary complex numbers, e.g. $e^{0}=1,\;e^1\sim 2.72,\;e^{1+i}\sim1.5+2.3i$, and so on.}
	
	\item We first assume that $f(x)\ne 0$, with which we can rewrite \equref{eq: exponential diff} as
	\be 
	\frac{1}{f'(x)}=\frac{1}{f(x)}
	\ee 
	By chain rule, we have
	\be 
	\rdr{f(x)}{x}\rdr{x}{f(x)}=1
	\ee 
	hence the above equation becomes
	\be 
	\rdr{x}{f(x)}=\frac{1}{f(x)}
	\ee 
	If we now replace $x=f^{-1}(y)$ where $f^{-1}$ is the inverse of the function $f$,\footnote{This means
		\bea 
		f::{}&\C\to\C\\
		f^{-1}::{}&\C\to\C\\
		f={}&x\to f(x)\\
		f^{-1}={}&f(x)\to x
		\eea 
	} we get
	\be 
	\rdr{f^{-1}(y)}{dy}=\frac{1}{y}
	\ee 
	By integrating this function, we get
	\be 
	f^{-1}(y)=\int\frac{dy}{y}
	\ee 
	If we now \emph{define} the function \emph{logarithm} as the right hand side, we arrive at the solution that \emph{the function whose derivative is equal to itself is the inverse of the logarithm function}, which we call the exponential function.\footnote{We can now check that our very initial assumption $f(x)\ne 0$ is indeed satisfied.}
\end{enumerate}

\paragraph{Summary} In the first approach, we \emph{defined} the exponential function as the solution of the differential equation $f'(x)=f(x)$. We can then \emph{derive} that its inverse (logarithmic function) can be given as the integral of $1/x$.\footnote{We can show this by using the fundamental theorem of calculus.} In the second approach, we \emph{defined} the logarithmic function as the integral of $1/x$, and then \emph{derived} that its inverse (exponential function) solves the differential equation. Which one we choose is purely conventional.

\paragraph{What did we learn?} In math, we \emph{define} many objects as our initial data, and then \emph{derive} other quantities based on those. What we \emph{choose to define} is purely conventional; however, we cannot afford to define too many things and still remain consistent. For instance, in the example above, we actually show that if we give two of the following three statements, the third one is already fixed by the other two: (1) \emph{exponential and logarithm functions are inverse of each other}, (2) \emph{exponential function is the solution of the differential equation $f'(x)=f(x)$}, and (3) \emph{logarithm function is the integration of $1/x$}.

\subsection{Classification}
In the beginning of the section above, we defined differential equations as any relation that contains the derivative operator $\rdr{}{x}$ and an unknown function $f(x)$.\footnote{As stated earlier, $f(x)$ is actually \emph{not} the function but the \emph{output} of the function $f$. Nevertheless, I'll abuse terminology here and there to remain more familiar to physicists.} There is nothing that stops us from generalizing this to multiple variables;\footnote{Alternatively, we can generalize to multiple \emph{functions}; for instance,
	\be 
	\rdr{f(x)}{x}=g(x)\;,\quad\rdr{g(x)}{x}=-f(x)\;.
	\ee 
	Such relations are called \emph{systems of differential equations}. We will see more about such systems in \S~\ref{chapter: Linear nonhomogeneous equations with functional coefficients}.
} indeed, an expression that contains the partial derivatives $\frac{\partial}{\partial x}$ and  $\frac{\partial}{\partial y}$ (along with an unknown function $f(x,y)$) is \emph{also} a differential equation. We then divide all differential equations into two categories:
\be 
\text{A differential equation is called}\left\{\begin{aligned}
	\text{ordinary}\\\text{partial}
\end{aligned}\right\}\text{ if there are}
\\
\text{derivatives with respect to}\left\{\begin{aligned}
	\text{one}\\\text{more than one}
\end{aligned}\right\}\text{variables.}
\ee 
For instance,
\be 
\frac{\partial f(x,y)}{\partial x}+\frac{\partial f(x,y)}{\partial y}+ f(x,y)=0
\ee 
is a partial differential equation. Until the last chapter, we will only focus on \emph{ordinary} differential equations!

We also define the \emph{order} of a differential equation to be the highest number of derivatives in it; for instance,
\be 
\rdr{^{3}}{x^{3}}f(x)=0
\ee 
is a third order differential equation,\footnote{
	See if you can convince yourself that 
	\be 
	f(x)=c_0+c_1x+c_2x^2
	\ee 
	for the coefficients $c_i$ is the solution to this equation.
} whereas
\be 
\rdr{^{3}}{x^{3}}f(x)+f(x)\rdr{^{4}}{x^{4}}f(x)+x\rdr{}{x}f(x)=0
\ee 
is a fourth order one. Note that not all differential equations have to have a finite order.\footnote{
	It is perfectly possible to define the differential equation
	\be 
	\exp(\rdr{}{x})f(x)={}& f(x)+3x^2+3x+1
	\ee 
	for which 
	\be 
	f(x)=x^3
	\ee 
	is a solution (see the footnote~\ref{footnote: exponentiated differential}). However, clearly, this differential equation has arbitrarily high numbers of derivatives, hence it is of infinite order.
}

The differential equations are also grouped according to the \emph{linearity} of the unknown function $f$. For instance, the differential equation
\be 
\rdr{^{2}}{x^2}f(x)+f(x)=1
\ee 
is called a \emph{linear differential equation}, whereas
\be 
f(x)\rdr{}{x}f(x)=x^3
\ee 
is a \emph{nonlinear} differential equation.\footnote{\label{footnote:superposition}
	One important feature of linear differential equations is that their solutions obey \emph{the principle of supersposition}; that is, if $f(x)$ and $g(x)$ are two solutions to the linear differential equation, then $c_1f(x)+c_2g(x)$ is also a solution for arbitrary constants $c_{1,2}$.
} An easy way to check if a differential equation is linear or nonlinear is to apply the transformation $f(x)\rightarrow \lambda f(x)$ for the constant $\lambda$: if the differential equation is linear in $\lambda$ (i.e. it can be written as $\lambda(\dots)+(\dots)=0$), then the differential equation is a linear differential equation; otherwise, it is a nonlinear differential equation.

Nonlinear equations are way harder to solve than the linear equations; in fact, we actually do not know how to solve most of the nonlinear equations! In practice, one usually handles them numerically, which is beyond of the scope of this course. If you are only interested in a particular regime, you can also \emph{linearize} a nonlinear equation around that regime, which is what most physicists do in practice. For instance, consider the nonlinear differential equation
\be 
\label{eq: nonlinear example}
\rdr{}{x}f(x)+\sin(f(x))=0
\ee 
If we say that we are only interested in the results $f(x)\ll1$, then we can linearize this equation as 
\be 
\rdr{}{x}f(x)+f(x)=0
\ee 
which has the solution
\be 
f(x)=c e^{-x}
\ee 
which satisfies our necessary condition for $x\gg 1$.\footnote{
	We can actually solve the full nonlinear differential equation \equref{eq: nonlinear example}; the result is
	\be 
	f(x)=2\arccot(\frac{2e^x}{c})
	\ee 
	which matches the linearized result in the regime it is valid, i.e. 
	\be 
	\lim\limits_{x\rightarrow\infty}2\arccot(\frac{2e^x}{c})=\lim\limits_{x\rightarrow\infty} c e^{-x}
	\ee 
}

A last classification we can do with our differential equations is their \emph{homogeneity}: a differential equation is said to be \emph{homogeneous} if it is invariant under the scaling of the unknown function. This is just a fancy way of saying that the differential equation does not change even if we replace $f(x)$ with $\lambda f(x)$ for an unknown constant $\lambda$.  

We can summarize the classification of all differential equations with examples as given in Table~\ref{table: Illustration of various differential equations}
\begin{table}
	\caption{\label{table: Illustration of various differential equations}Illustration of various differential equations}
	\centering
	\footnotesize
	\begin{tabular}{llll}
		\textbf{Example differential equation}&\textbf{ordinary?}&\textbf{linear?}&\textbf{homogeneous?}\\
		$\displaystyle\rdr{^2f(x)}{x^2}+f(x)=0$&\cmark&\cmark&\cmark
		\\\\
		$\displaystyle\rdr{^2f(x)}{x^2}+f(x)=x^2$&\cmark&\cmark&\xmark
		\\\\
		$\displaystyle f(x)\rdr{^3f(x)}{x^3}+\left(\rdr{f(x)}{x}\right)^2=0$&\cmark&\xmark&\cmark
		\\\\
		$\displaystyle\rdr{^2f(x)}{x^2}+\sin(f(x))=0$&\cmark&\xmark&\xmark
		\\\\
		$\displaystyle\pdr{^2f(x,y)}{x\partial y}+f(x,y)=0$&\xmark&\cmark&\cmark
		\\\\
		$\displaystyle\pdr{^2f(x,y)}{x^2}+f(x,y)=x^2$&\xmark&\cmark&\xmark
		\\\\
		$\displaystyle f(x)\pdr{^3f(x,y)}{x^3}+\left(\pdr{f(x,y)}{y}\right)^2=0$&\xmark&\xmark&\cmark
		\\\\
		$\displaystyle\pdr{^2f(x,y)}{x\partial y}+\sin(f(x,y))=0$&\xmark&\xmark&\xmark
	\end{tabular}
\end{table}

\chapter{Linear equations with constant coefficients}
\section{Linear mappings and kernels}
Formally, we could write down the most generic linear ordinary differential equation for the unknown function $f$ as
\be 
g\left(x,\rdr{}{x}\right)f(x)=h(x)
\ee 
for arbitrary known functions $g$ and $h$. Indeed, this is a linear equation in the function $f$, and it has only one kind of derivative, $\rdr{}{x}$, hence it is an ordinary differential equation.

Let's assume that we are given such an equation for known $g$ and $h$, and we are trying to solve for $f$. A naive attempt would be to write down
\be 
f(x)=\frac{1}{g\left(x,\rdr{}{x}\right)}h(x)
\ee 
which looks like a total nonsense! Nevertheless, we cannot help but realize that it does somewhat work in some cases; for instance, for
\be 
\rdr{}{x}f(x)=x^2
\ee 
we can write down
\be 
f(x)=\left(\rdr{}{x}\right)^{-1}x^2
\ee 
which we can rewrite as 
\be 
f(x)=\int dx x^2=\frac{x^3}{3}+\text{constant}
\ee 
by observing that integral is \emph{the inverse of derivative}.\footnote{Rigorously speaking, we are referring to indefinite integrals (also known as antiderivatives or Newton integrals).}

We need to be careful with such manipulations, but physicists \emph{tend to} define things \emph{formally}, which allows such expressions. For instance, we could say that \emph{the formal solution} to the differential equation
\be 
\left(\rdr{^2}{x^2}+c^2\right)f(x)=0
\ee 
is
\be 
f(x)=\left(\rdr{^2}{x^2}+c^2\right)^{-1}0
\ee 
For a physicist, there is nothing wrong with writing things like the equation above \emph{as long as we are careful with what we mean}! To spell out what we really mean with such an equation, we need to set up some terminology.

Remember how we defined the derivative higher order function (or its integer powers) in \equref{eq: type of nth order derivatives}:
\bea 
\rdr{{}^n}{x^n}::{}&\left(\texttt{A}\to\texttt{B}\right)\to\left(\texttt{A}\to\texttt{B}\right)\\
f::{}&\texttt{A}\to\texttt{B}\\
f^{(n)}::{}&\texttt{A}\to\texttt{B}
\eea  
The operation of taking derivatives is \emph{a map of functions to functions}; in fact, it is a \emph{linear map}!\footnote{We can easily see the linearity by noting the relation
	\be 
	\rdr{^n}{x^n}\left(c_1 f(x)+c_2 g(x)\right)=c_1\rdr{^n}{x^n}f(x)+c_2\rdr{^n}{x^n}g(x)
	\ee 
	for arbitrary coefficients $c_1$ and $c_2$.
} Linear maps are really useful when we work with vectors, but we will see below that an important notion called \emph{kernel} can be extended from vector spaces to the functions as well.\footnote{
	The analogy is as follows: functions are like vectors, and linear transformations due to derivatives are like matrix multiplications. Indeed, a matrix $M$ (say $\begin{pmatrix}
		1&1\\0&1
	\end{pmatrix}$) acting on a vector $v$ (say $\begin{pmatrix}
		2\\3
	\end{pmatrix}$) is a linear mapping, just as the derivative $\rdr{}{x}$ turning the function $x^2$ into $2x$.
	
	The analogy extends to the equations. We could solve $M\.w=v$ for the unknown vector $w$, similar to how we solve $\rdr{}{x}f(x)=x^2$ for the function $f$. In fact such analogies can be made more precise if we realize that a function $f$ is in some sense an infinite dimensional vector. Indeed, in a neighborhood containing the point $c$ in which the function $f$ is analytic, we can just do a Taylor expansion and rewrite $f(x)$ as 
	\be 
	f(x)=\sum\limits_{n=0}^{\infty}f_n x^n
	\ee  
	where $f_n$ can be viewed as an infinite-dimensional vector $f_n=\left(f_0,f_1,\dots\right)$.\footnotemark
}\footnotetext{
	If we take a step back, we can actually realize that the converse is also true (in fact, it is \emph{generically} true): \emph{any vector $v$ is simply a function from integers to the domain of the components of the vector}.
	
	What do we mean by that? Consider the vector $\vec{v}=\begin{pmatrix}
		1\\0\\-3
	\end{pmatrix}$. This vector is equivalent to the set of relations $v_1=1$, $v_2=0$, and $v_3=-3$. But that is simply a function
	\bea 
	v::{}&\N\to\R\\
	v={}&n\to\left\{\begin{aligned}
		1&\text{ if }n=1\\
		0&\text{ if }n=2\\
		-3&\text{ if }n=3\\
		\text{undefined}&\text{ otherwise}
	\end{aligned}\right.
	\eea 
	
	This process can be generalized to any finite or infinite dimensional vector.
} 

In vector spaces linear transformations are implemented by matrices; for instance, the transformation ``clockwise rotation by $\pi/4$'' on $2d$ vectors can be implemented by the matrix
\be 
R(-\pi/4)=\frac{1}{\sqrt{2}}\begin{pmatrix}
	1&1\\-1&1
\end{pmatrix}
\ee 
which indeed rotates any vector $\vec{v}=\begin{pmatrix}
	v_x\\v_y
\end{pmatrix}$ to its rotated version\linebreak \mbox{$R(-\pi/4)\.\vec{v}$}; for instance, the unit vector pointing to NorthEast direction on a map ---i.e. $\frac{1}{\sqrt{2}}\begin{pmatrix}
	1\\1
\end{pmatrix}$--- gets rotated to the vector pointing to the East by this $45$ degrees of clockwise rotation:
\be 
\frac{1}{\sqrt{2}}\begin{pmatrix}
	1&1\\-1&1
\end{pmatrix}\frac{1}{\sqrt{2}}\begin{pmatrix}
	1\\1
\end{pmatrix}=\begin{pmatrix}
	1\\0
\end{pmatrix}
\ee 
In fact \emph{a general counterclockwise rotation by an angle $\theta$} can be implemented by the matrix
\be 
R(-\pi/4)=\begin{pmatrix}
	\cos(\theta)&\sin(\theta)\\-\sin(\theta)&\cos(\theta)
\end{pmatrix}
\ee 

The \emph{kernel of a map} (or equivalently the kernel of the matrix that implements that map) is the set of vectors that are mapped to \emph{zero vector}; for instance, we can show that the only such vector for the rotation matrix is the zero vector itself; in other words,
\be 
\begin{pmatrix}
	\cos(\theta)&\sin(\theta)\\-\sin(\theta)&\cos(\theta)
\end{pmatrix}\begin{pmatrix}
	a\\b
\end{pmatrix}=\begin{pmatrix}
	0\\0
\end{pmatrix}
\ee 
is true only if $a=b=0$; thus, we write
\be 
\ker\left[R(\theta)\right]=\{\vec{0}\}
\ee 
which means \emph{the only vector that can be rotated to the zero vector is the zero vector itself}. When said this way, it clearly makes sense!

Let's look at another example: we define the matrix $S$ as
\be 
S=\begin{pmatrix}
	1&2\\2&4
\end{pmatrix}
\ee 
If we now look at the \emph{kernel of this linear transformation}, we find a non-trivial result; in fact, we can immediately write down
\be 
\ker [S]=\left\{\vec{0},\begin{pmatrix}
	2a\\-a
\end{pmatrix}\right\}
\ee 
which means not only the zero vector gets mapped to zero vector, but also any vector of the form $\begin{pmatrix}
	2a\\-a
\end{pmatrix}$ becomes zero under the action of this matrix. Indeed, we see that
\be 
\begin{pmatrix}
	1&2\\2&4
\end{pmatrix}\begin{pmatrix}
	2a\\-a
\end{pmatrix}=\begin{pmatrix}0\\0\end{pmatrix}
\ee 

What does this mean? And what is the action of this linear transformation? Just like the rotation matrix rotates any input vector, this $S$ matrix also transforms the input vectors, but it actually \emph{squeezes} them. Indeed, we see that for any vector pointing in any direction, the action of this transformation squeezes them into the $\begin{pmatrix}
	1\\2
\end{pmatrix}$ direction. We can see this explicitly:
\bea 
S\.\vec{v}_{\text{input}}={}&\vec{v}_{\text{output}}\\
\vec{v}_{\text{input}}={}&\begin{pmatrix}
	a\\b
\end{pmatrix}
\\
\vec{v}_{\text{output}}={}&(a+2b)\begin{pmatrix}
	1\\2\end{pmatrix}
\eea

\paragraph{Summary:} We have seen with examples that some linear transformations (such as rotation) has a \emph{trivial kernel},\footnote{We say that a kernel is trivial if it only includes the identity element ($\vec{0}$ vector in the case of vector spaces).} whereas other transformations (such as squeezing) may have a nontrivial kernel. 
\paragraph{Quick check in vector spaces:} Whether a linear transformation has a trivial kernel or not can immediately be checked in the case of vector spaces by computing the \emph{determinant} of the matrix that implements that transformation. If the determinant is zero (e.g. $\det S=0$), then the kernel is nontrivial; otherwise ($\det R = 1$) the kernel is trivial.
\paragraph{The importance of nontrivial kernel:} If the kernel is nontrivial, then the transformation is not uniquely invertible. For instance, if we have 
\be 
\begin{pmatrix}
	1&2\\2&4
\end{pmatrix}\begin{pmatrix}
	x\\y
\end{pmatrix}=\begin{pmatrix}
	1\\2
\end{pmatrix}
\ee 
Then $\begin{pmatrix}
	1\\0
\end{pmatrix}$ is a solution, but so is $\begin{pmatrix}
	3\\-1
\end{pmatrix}$ or $\begin{pmatrix}
	-1\\1
\end{pmatrix}$. In fact, the full family of solutions is given as 
\be 
\begin{pmatrix}
	x\\y
\end{pmatrix}=\begin{pmatrix}
	1\\0
\end{pmatrix}+\begin{pmatrix}
	2a\\-a
\end{pmatrix}
\ee 
On the other hand, the rotation having a trivial kernel makes sure that we have a unique answer; for instance, 
\be 
\frac{1}{\sqrt{2}}\begin{pmatrix}
	1&1\\-1&1
\end{pmatrix}\begin{pmatrix}
	x\\y
\end{pmatrix}=\begin{pmatrix}
	1\\2
\end{pmatrix}
\ee 
has the unique answer
\be 
\begin{pmatrix}
	x\\y
\end{pmatrix}=\frac{1}{\sqrt{2}}\begin{pmatrix}
	-1\\3
\end{pmatrix}
\ee 
\paragraph{Back to the differential equations:} The story with the matrices immediately carries over to the differential equations: the differential operators have nontrivial kernels, and these results are called \emph{homogeneous solutions}. For the general differential equation
\be 
g\left(x,\rdr{}{x}\right)f(x)=h(x)
\ee 
the solution then becomes
\be 
f(x)=p(x)+\ker \left[g\left(x,\rdr{}{x}\right)\right]
\ee 
where $p(x)$ is called the \emph{particular solution}, and the elements of the kernel are the homogeneous solutions.

\section{Homogeneous solutions}
\subsection{Basics}
We have discussed in \S~\ref{section: basics of diff eqn} that the solution to the differential equation $f'(x)=f(x)$ is given as\footnote{As we discussed in that section, this result is either a definition or a derived result depending our conventions.}
\be 
f(x)=e^x
\ee 
We can actually generalize this to\footnote{
	One way to show this is the judicious use of the chain rule as follows:
	\be 
	\rdr{}{x}e^{x}=e^{x}\xrightarrow{\text{define }x=\lambda y} \rdr{}{x}e^{\lambda y}=e^{\l y}\\
	\xrightarrow{\text{use chain rule}}\rdr{y}{x}\rdr{}{y}e^{\lambda y}=e^{\l y}\\\xrightarrow{\text{use }y=x/\l}\frac{1}{\l}\rdr{}{y}e^{\lambda y}=e^{\l y}
	\\\xrightarrow{\text{rewrite}}\left(\rdr{}{y}-\l\right)e^{\lambda y}=0
	\ee 
}
\be 
\left(\rdr{}{y}-\l\right)e^{\lambda y}=0
\ee 
which says the \emph{linear ordinary differential equation with constant coefficient}
\be 
\left(\rdr{}{x}-\l\right)f(x)=0
\ee 
has the \emph{solution}
\be 
f(x)=e^{\lambda x}
\ee 
In the fancy language, we can now write this result as 
\be 
\ker\left[\left(\rdr{}{x}-\l\right)\right]=\left\{0,e^{\lambda x}\right\}
\ee 
which means that
\be 
\left(\rdr{}{x}-\l\right)f(x)=h(x)\quad\Rightarrow\quad f(x)=p(x)+ce^{\l x}
\ee 
for the arbitrary variable $c$, where we will discuss the computation of particular solution $p(x)$ later.

One immediate observation we can make is that $e^{\l x}$ would still be a solution if there were more terms to the left of the equation; in other words,
\be 
g\left(x,\rdr{}{x}\right)\left(\rdr{}{x}-\l\right)f(x)=0
\ee 
is still satisfied for $f(x)=e^{\l x}$. This becomes particularly interesting if $g\left(x,\rdr{}{x}\right)$ is a product of $\left(\rdr{}{x}-a\right)$, i.e.
\be
\label{eq: product form diff eqn} 
\left(\rdr{}{x}-r_1\right)\left(\rdr{}{x}-r_2\right)\cdots\left(\rdr{}{x}-r_n\right)f(x)=0
\ee 
Clearly $e^{r_nx}$ is a solution, but as these terms commute with each other, we can immediately write down the full solution as\footnote{This follows from the principle of superposition, see footnote~\ref{footnote:superposition}.}
\bea 
f::{}&{}\C\to\C\\
f={}&{}x\to \sum\limits_{i=1}^n c_i e^{r_i x}
\eea
for arbitrary constants $c_i$.

Differential equations are usually given in the form
\be 
\left(a_0+a_1\rdr{}{x}+a_2\rdr{^2}{x^2}+\dots+a_n\rdr{^n}{x^n}\right)f(x)=0
\ee 
which can be brought to the form \equref{eq: product form diff eqn} by simply finding the roots of the equation\footnote{This equation is called \emph{characteristic equation} of the given system.}
\be 
a_0+a_1r+a_2r^2+\dots+a_nr^n=0
\ee 
If the coefficients $a_i$ are simply complex numbers (or real numbers as a special case of complex numbers), we can always find $n$ complex roots $r_i$!\footnote{
	The field of complex numbers is algebraically closed, hence such polynomials \emph{always} have solutions. In contrast, the field of real numbers is \emph{not} algebraically closed; for instance, $x^2+1=0$ has no real root. For more information on these, see \emph{fundamental theorem of algebra}.
}

\subsection{Repeated roots}
Consider the differential equation
\be
\left(\rdr{}{x}-r_1\right)\left(\rdr{}{x}-r_2\right)f(x)=0
\ee 
which has the solution $f(x)=c_1 e^{r_1x}+c_2e^{r_2x}$. If we now do a change of parameters as
\be 
\label{eq: repeated root transformation}
r_1=r\;,\quad r_2=r+\de\;,\quad c_1=a-\frac{b}{\de}\;,\quad c_2=\frac{b}{\de}
\ee 
our statement becomes
\be 
\left(\rdr{}{x}-r\right)\left(\rdr{}{x}-(r+\de)\right)f(x)=0\quad f(x)=a e^{r x}+b\frac{e^{(r+\de)x}-e^{rx}}{\de}
\ee 
If we now take the limit $\de\to 0$ and recognize the definition of derivative, we arrive at
\be 
\label{eq: repeated result}
\left(\rdr{}{x}-r\right)^2f(x)=0\quad\rightarrow\quad f(x)=a e^{r x}+bxe^{rx}
\ee 

The way we arrived at this curious result is not satisfactory: we did a particular transformation in \equref{eq: repeated root transformation} and we do not have a strong reason to choose that transformation. For instance, if we instead choose
\be 
\label{eq: repeated root transformation 2}
r_1=r\;,\quad r_2=r+\de\;,\quad c_1=a-b\;,\quad c_2=b
\ee 
and then take the limit $\de\to0$, we end up with
\be 
\label{eq: repeated result 2}
\left(\rdr{}{x}-r\right)^2f(x)=0\quad\xrightarrow{???}\quad f(x)=a e^{r x}
\ee 
We missed the second piece of $f(x)$ in \equref{eq: repeated result}.

What is the resolution of this discrepancy? We have two potential scenarios: \textbf{(a)} $x e^{ax}$ is \emph{a spurious solution},\footnote{Spurious solutions are fake results that emerge as solutions even though they actually do not solve the problem.} or \textbf{(b)} \equref{eq: repeated result 2} misses one of the solutions.

We can check it straightforwardly that the option \textbf{(b)} is the correct case,\footnote{
	We only need to check
	\be 
	\left(\rdr{}{x}-r\right)^2(xe^{rx})=0
	\ee 
} indicating that out choice of reparametrization of the variables in terms of infinitesimal variable $\de$ affects which solutions we obtain. This then begs the question: \emph{can we potentially have more solutions?}

We have mathematical arguments why a second order differential equation should have two solutions,\footnote{\draftnote{Maybe expand on this more.}} so we can already infer that \equref{eq: repeated result} is the full solution; however, let's see another method to derive why this is the case.

Define a new function $g$ such that $f(x)=g(x)e^{rx}$.\footnote{Note that we can do this \emph{without a loss of generality}!} If we insert this into the original differential equation, we immediately see that
\be 
\left(\rdr{}{x}-r\right)^2 f(x)=0\quad\xrightarrow{f(x)=g(x)e^{rx}}\quad \rdr{^2}{x^2}g(x)=0
\ee 
which tells us that \emph{the most general result} is $f(x)=(ax+b)e^{rx}$. In fact, this derivation generalizes, i.e.
\be 
\left(\rdr{}{x}-r\right)^n f(x)=0\quad\xrightarrow{f(x)=g(x)e^{rx}}\quad \rdr{^n}{x^n}g(x)=0
\ee
yielding $f(x)=(a_1+a_2 x+\dots a_n x^{n-1})e^{rx}$.

With the discussion above, we can now write down the most general homogeneous solution to a linear ordinary differential equation with constant coefficients:
\bea
\left(\rdr{}{x}-r_1\right)^{m_1+1} \left(\rdr{}{x}-r_2\right)^{m_2+1}\cdots \left(\rdr{}{x}-r_n\right)^{m_n+1}f(x)=0\\
\Rightarrow\qquad f(x)=\sum\limits_{i=1}^n\left[\left(\sum\limits_{k=0}^m c_{ik} x^k\right)e^{r_i x}\right]
\eea 
for arbitrary coefficients $c_{ij}$.


\subsection{Examples}
\draftnote{RLC circuits, damper-spring systems, traffic models, etc.}

\section{Laplace transform}
Consider the following higher order function:\footnote{\label{footnote:dummy variables}
	Note that the letters on the left hand side of an arrow are \emph{placeholders}, i.e. they do not inherently carry information. Such parameters are called dummy variables in math (or scooping variables in computer science) and they are ubiquitous in math and physics; for instance, the integrals $\int dx f(x)$ and $\int dy f(y)$ are the same expression as $x$ and $y$ are dummy variables. Similarly, the expressions $x\to f(x)$ and $y\to f(y)$ are equivalent.
	
	It gets complicated with the higher order functions as they include multiple arrows; in such cases, the left hand side of \emph{each arrow} contains only placeholders for \emph{the right hand side of that particular arrow}. For example, let us rewrite \equref{eq: general integral transform} in a colorful way:
	\bea
	\mathfrak{IT}::{}&(\C\to\C)\to (\C\to\C)\\
	\mathfrak{IT}={}&\bm{(\textcolor{orange}{x}\textcolor{orange}{\to} \textcolor{red}{f}(\textcolor{orange}{x}))\textcolor{red}{\to}\left(\textcolor{blue}{s}\textcolor{blue}{\to}\textcolor{magenta}{\int_{\textcolor{black}{\a}}^{\textcolor{black}{\b}}} K(\textcolor{magenta}{x},\textcolor{blue}{s})\textcolor{red}{f}(\textcolor{magenta}{x})d\textcolor{magenta}{x}\right)}
	\eea  
	Variables of the same color can be changed as they are dummy variables for the same color arrow (in the case of the color magenta, the variables are dummy variables of the integration operation). For instance, following expressions are all equivalent:
	\begin{equation*}
		\begin{aligned}
			\mathfrak{IT}={}&\bm{(\textcolor{orange}{x}\textcolor{orange}{\to} \textcolor{red}{f}(\textcolor{orange}{x}))\textcolor{red}{\to}\left(\textcolor{blue}{s}\textcolor{blue}{\to}\textcolor{magenta}{\int_{\textcolor{black}{\a}}^{\textcolor{black}{\b}}} K(\textcolor{magenta}{x},\textcolor{blue}{s})\textcolor{red}{f}(\textcolor{magenta}{x})d\textcolor{magenta}{x}\right)}
			\\
			\mathfrak{IT}={}&\bm{(\textcolor{orange}{y}\textcolor{orange}{\to} \textcolor{red}{f}(\textcolor{orange}{y}))\textcolor{red}{\to}\left(\textcolor{blue}{z}\textcolor{blue}{\to}\textcolor{magenta}{\int_{\textcolor{black}{\a}}^{\textcolor{black}{\b}}} K(\textcolor{magenta}{x},\textcolor{blue}{z})\textcolor{red}{f}(\textcolor{magenta}{x})d\textcolor{magenta}{x}\right)}
			\\
			\mathfrak{IT}={}&\bm{(\textcolor{orange}{y}\textcolor{orange}{\to} \textcolor{red}{g}(\textcolor{orange}{y}))\textcolor{red}{\to}\left(\textcolor{blue}{z}\textcolor{blue}{\to}\textcolor{magenta}{\int_{\textcolor{black}{\a}}^{\textcolor{black}{\b}}} K(\textcolor{magenta}{s},\textcolor{blue}{z})\textcolor{red}{g}(\textcolor{magenta}{s})d\textcolor{magenta}{s}\right)}
		\end{aligned}
	\end{equation*}
	Note that the letters $K$, $\a$, $\b$ are not dummy variables as they are externally fixed. Nevertheless, we \emph{can} turn them into dummy variables of the equal sign $=$ by defining them in the left hand side of $=$; e.g.
	\begin{equation*}
		\begin{aligned}
			\mathfrak{IT}_{\textcolor{cyan}{K},\textcolor{cyan}{\a},\textcolor{cyan}{\b}}\textcolor{cyan}{=}{}&\bm{(\textcolor{orange}{y}\textcolor{orange}{\to} \textcolor{red}{g}(\textcolor{orange}{y}))\textcolor{red}{\to}\left(\textcolor{blue}{z}\textcolor{blue}{\to}\textcolor{magenta}{\int_{\textcolor{cyan}{\a}}^{\textcolor{cyan}{\b}}} \textcolor{cyan}{K}(\textcolor{magenta}{s},\textcolor{blue}{z})\textcolor{red}{g}(\textcolor{magenta}{s})d\textcolor{magenta}{s}\right)}
			\\
			\mathfrak{IT}_{\textcolor{cyan}{T},\textcolor{cyan}{\g},\textcolor{cyan}{\l}}\textcolor{cyan}{=}{}&\bm{(\textcolor{orange}{y}\textcolor{orange}{\to} \textcolor{red}{g}(\textcolor{orange}{y})) \textcolor{red}{\to}\left(\textcolor{blue}{z}\textcolor{blue}{\to}\textcolor{magenta}{\int_{\textcolor{cyan}{\g}}^{\textcolor{cyan}{\l}}} \textcolor{cyan}{T}(\textcolor{magenta}{s},\textcolor{blue}{z})\textcolor{red}{g}(\textcolor{magenta}{s})d\textcolor{magenta}{s}\right)}
		\end{aligned}
	\end{equation*}
	are equivalent expressions ---just like \mbox{$f(x)=x^2$} and \mbox{$f(y)=y^2$} being equivalent expressions.
}
\bea[eq: general integral transform]
\mathfrak{IT}::{}&(\C\to\C)\to (\C\to\C)\\
\mathfrak{IT}={}&(x\to f(x))\to\left(s\to\int\limits_\a^\b K(x,s)f(x)dx\right)
\eea 
where $\mathfrak{IT}$ is an \emph{integral transform}, i.e. it maps a function to another one by using the integration operation. The function $K$ above is called \emph{the kernel of the tranformation}: different kernels (along with different integration ranges) lead to different integral transforms.

The Laplace transform is a special kind of an integral transformation:
\bea[eq: laplace transform]
\cL::{}&(\C\to\C)\to (\C\to\C)\\
\cL={}&(x\to f(x))\to\left(s\to\int\limits_0^\infty e^{-xs}f(x)dx\right)
\eea 
which plays a immense role in the analysis of linear ordinary differential equations with constant coefficients because such equations become algebraic under this transformation. To see this, consider how the laplace transform interacts with the derivative operation: replace $f$ with $g'$ above, and integrate by parts
\be 
\cL={}&(x\to g'(x))\to\left(s\to \left[s\int\limits_0^\infty e^{-xs}g(x)dx-g(0)+\lim\limits_{x\rightarrow\infty}e^{-xs}g(x)\right]\right)
\ee 
We will assume that the last piece is zero, which is a necessary condition for the Laplace transform to be well-defined in the first place.\footnote{Otherwise, the integral in the definition does not converge.} Thus
\be 
\cL\.g'={}&s\to \left(s(\cL\.g)(s)-g(0)\right)
\ee 
or in a more conventional notation, we state
\be 
\rdr{g(x)}{x}\xrightarrow{\text{Laplace transform}} s G(s)-g(0)
\ee 
where $G(s)$ is the laplace transform of $g(x)$.

One can repeat this process iteratively for higher numbers of derivative; in fact, we can immediately write down the Laplace transform of $n-$the derivative of a function:
\be 
\left(\cL\.g^{(n)}\right)(s)=s^n\left(\cL\.g\right)(s)-\sum\limits_{i=0}^{n-1} s^{n-i-1}g^{(i)}(0)
\ee 

We can now justify our previous statement of \emph{\textbf{Laplace transform converts  linear ordinary differential equations with constant coefficients into algebraic ones}}! Start with the most generic such differential equation:
\be 
\label{eq: general constant coefficient differential equation}
\sum\limits_{i=0}^n a_i f^{(i)}(x)=g(x)
\ee 
which is \emph{homogeneous} if $g(x)=0$ and nonhomogeneous otherwise. If we take the Laplace transform of this equation, we end up with
\be 
\sum\limits_{i=0}^n a_i \left[s^i F(s)-\sum\limits_{k=0}^{i-1}s^{i-k-1}f^{(k)}(0)\right]=G(s)
\ee 
where $F(s)\coloneqq (\cL\.f)(s)$ and $G(s)\coloneqq (\cL\.g)(s)$ are defined for brevity. By using algebra, we can rewrite this equation in the form
\be 
F(s)=\frac{\displaystyle\sum\limits_{i=0}^{n-1}f^{(i)}(0)\left[\sum\limits_{k=1+i}^na_ks^{k-i-1}\right]}{\displaystyle\sum\limits_{i=0}^na_is^i}+\frac{G(s)}{\displaystyle\sum\limits_{i=0}^na_is^i}
\ee 

Let us comment on this result a little bit. \textbf{Firstly}, we can immediately state that the solution $f(x)$ to the differential equation in \equref{eq: general constant coefficient differential equation} is simply the \emph{inverse Laplace transform} of $F(s)$. Even though this is a well-defined transformation that we can introduce, we actually do not need it: we will discuss other methods to obtain $f(x)$ from $F(s)$. \textbf{Secondly}, we can actually see that the first piece is the homogeneous solution to the differential equation, and the second piece is the particular solution: Laplace transform allowed us to solve both of them at once!

Consider the simple case of $n=2$:
\be 
F(s)=\frac{f(0)\left(a_1+a_2s\right)+f^{(1)}(0) a_2}{a_0+a_1s+a_2s^2}+\frac{G(s)}{a_0+a_1s+a_2s^2}
\ee 
If $r_1$ and $r_2$ are two distinct roots of $a_0+a_1s+a_2s^2=0$, we can simply write down this expression as \emph{bla bla bla bla}\draftnote{ to be written later, probably next year.}

\draftnote{
	We have covered several topics in class but I will not be able to type them in time. So I'm postponing that to next year; after all, all of those topics are already in the textbook --- chapter 6 of Elementary Differential Equations and Boundary Value Problems” by Boyce and Diprima (10th edition). The summary is as follows:
	\begin{enumerate}
		\item Derive laplace transforms of common functions
		\item Solving homogeneous differential equations using Laplace transformation, and "proof" of why characteristic equation method works
		\item Discussion of particular solutions in Laplace domain; example: RLC circuit with an AC input
		\item Discussion of how it is tiresome to repeat the computations for each nonhomogeneous piece and why we need a universal solution true for any non-homogeneous piece. For this, we need to find an operation that maps to multiplication in Laplace domain
		\item Derive $H(s)=F(s)G(s)$ means $h(x)= $convolution of $f(s)$ and $g(x)$, i.e. derive convolution
		\item Introduce impulse response: it is the solution to the same differential equation with nonhomogeneous part $H(s)=1$ in laplace domain (with $i(0)=i'(0)=\dots=0$).
		\item Introduce Dirac-delta distribution, discuss why it is a generalized function but not a well defined function. Show that it maps to $1$ as needed.
		\item Write down the most generic solution to a linear ordinary differential equation with constant coefficients. Show examples.
	\end{enumerate}
}

\chapter{Linear equations with functional coefficients}
\section{Homogeneous solution}
\draftnote{
	Summary of what we have discussed in class (to be typed later, maybe next year):
	\begin{enumerate}
		\item Differential equations with functional coefficients \emph{do not necessarily have generic analytic solutions}: we do not know how to solve them except the isolated cases!
		\item Explicit solutions of $\left(x\rdr{}{x}+a\right)f(x)=0$ and $\left(x^k\rdr{}{x}+a\right)f(x)=0$ for $k\ne 1$.
		\item How these differential operators can be chained, and how it leads to the following form:
		\be 
		\left(x^k\rdr{}{x}+a_1\right)\cdots\left(x^k\rdr{}{x}+a_n\right)f(x)=0
		\ee 
		for both $k=1$ \& $k\ne 1$.
		\item How these equations are secretly related to the differential equations with constants coefficients through a reparametrization, i.e. 
		\be 
		\rdr{}{y}=\frac{1}{1-n}x^n\rdr{}{x}\text{ for }y=x^{-n+1}\\
		\rdr{}{y}=x\rdr{}{x}\text{ for }y=\log(x)
		\ee 
		\item Lesson: The differential equations with constant coefficients are nice guys with straightforward general solutions: do your best to check if a given differential equation can be brought to that form! Usually, the physics of the problem gives us insight as to whether that would be possible.
		\item Discuss why $k=1$ case above is treated differently, show a little bit about scale invariance, and informally mention the Mellin transform: we do not need to know this for this course (but it is super relevant in modern physics)!
		
		\item Introduce Euler equations: show how this is solvable simple because of the scale invariance and how it is actually a subset of similar higher order equations.
		
		\item Discuss change of variables to make equations constant coefficients:
		\begin{multline}
			f''(x)+p(x)f'(x)+q(x)f(x)=0\\\Rightarrow\\
			(u')^2 f''(u)+(u''(x)+u'(x)p(x))y'(u)+q(x)y(u)=0
		\end{multline}
		for a change of parameter $x\rightarrow u(x)$. For this to be constant coefficient, we need $u(x)=\int \sqrt{q(x)}dx$ and $\frac{u''(x)+u'(x)p(x)}{(u')^2}=$constant. 
		\item An example equation where change of variables would work: $ty''+(t^2-1)y'+t^3y=0$.
		
		\item Reparametrization is harder to do generically for higher orders! For those equations (also for second order equations without reparametrization), check if the function $f(x)$ is missing. If that is missing, we can lower the order of differential equation by writing it in terms of a new function $g(x)=f'(x)$. If both $f(x)$ and $f'(x)$ are missing, use $g(x)=f''(x)$, and so on!
		\item Examples (page 135 of textbook): $xf''(x)+f'(x)=0$, $x^2f''(x)+2xf'(x)=2$. Solve these!
		\item Another thing to check is if the equation is \emph{exact}, i.e. if it can be rewritten as a total derivative:
		\begin{multline*}
			\left[p_n(x)\rdr{^n}{x^n}+\dots+p_1(x)\rdr{}{x}+p_0(x)\right]f(x)=0\\
			\xRightarrow{???}\\
			\rdr{}{x}\left(\left[q_{n-1}(x)\rdr{^{n-1}}{x^{n-1}}+\dots+q_1(x)\rdr{}{x}+q_0(x)\right]f(x)\right)=0
		\end{multline*}
		If that is the case, then we can turn an order-$n$ (non)homogeneous differential equation into an order-$(n-1)$ nonhomogeneous one.
		\item For a second order differential equation $p(x)f''(x)+q(x)f'(x)+r(x)f(x)=0$, the condition $p''(x)-q'(x)+r(x)=0$ is sufficient for it to be exact.
		\item Examples (page 157 of textbook): $f''(x)+xf'(x)+f(x)=0$, $x^2f''(x)+xf'(x)-f(x)=0$.
		\item \textbf{Reduction of order:} if we already know $k$ solutions of an order $n$ differential equation, we can use that information to transform the system into an order $n-k$ differential equation with no known solutions. This is rather useful as lower differential equations are easier to solve, and it becomes extremely useful if we know one solution of a second order differential equation as first order differential equations are always formally solvable (we will discuss this in more detail later).
		\item Examples (page 174 of textbook): $xf''(x)-f'(x)+4x^3f(x)=0$ with $f_1(x)=\sin(x^2)$, $(x-1)f''(x)-xf'(x)+f(x)=0$ with $f_1(x)=e^x$.
		\item We introduced the Levi-Civita symbol $\e::\{\Z^+,\dots,\Z^+\}\to\Z$ and discussed its properties.
		\bea 
		\e::{}&{}\{\Z^+,\dots,\Z^+\}\to\Z\\
		\e={}&{}\{a_1,\dots,a_n\}\to\left\{\begin{aligned}
			1&\quad\text{ if }(a_1a_2\dots a_n)\text{ is an even permutation of }(12\dots n)\\
			-1&\quad\text{ if }(a_1a_2\dots a_n)\text{ is an odd permutation of }(12\dots n)\\
			0&\quad\text{ otherwise}
		\end{aligned}\right.
		\eea 
		Example: $(132)\to(123)$: we need 1 permutation for $(132)$: $\e_{132}=-1$. $(2314)\to(2134)\to(1234)$: we need 2 permutations for $2314$: $\e_{2314}=1$.\\
		Properties: $\e_{\dots a\dots a\dots }=0$, $\e_{\dots a\dots b\dots}=- \e_{\dots b\dots a\dots}$
		\item We introduced the function $\det::\mathfrak{M}_{n\x n}(\C)\to\C$ in terms of Levi-Civita symbol $\e$ and discussed its properties.
		\bea 
		\det::{}&{}\mathfrak{M}_{n\x n}(\C)\to\C\\
		\det={}&{}\begin{pmatrix}
			a_{11}&a_{12}&\dots & a_{1n}\\
			a_{21}&a_{22}&\dots & a_{2n}\\
			\dots \\
			a_{n1}&a_{n2}&\dots & a_{nn}
		\end{pmatrix}\to\sum\limits_{i_1,\dots,i_n}\e_{i_1\dots i_n}a_{1i_1}\dots a_{ni_n}
		\eea 
		
		\item We discussed linear independence of solutions and introduced the Wronskian determinant to check if given set of solutions span the solution space.
		
		The summary of the discussion is as follows. Assume that we are given an order$-n$ linear ordinary differential equation $g\left(x,\rdr{}{x}\right)f(x)=h(x)$, and assume that we have found $n-$solutions $f_i(x)$. If these solutions are linearly independent, they span the solution space and can be used to match any initial condition uniquely, i.e.
		\bea 
		\sum\limits_{i=1}^n c_i f_i(x_0)=&f(x_0)\\
		\sum\limits_{i=1}^n c_i f_i'(x_0)=&f'(x_0)\\
		\dots\\
		\sum\limits_{i=1}^n c_i f_i^{(n-1)}(x_0)=&f^{(n-1)}(x_0)
		\eea  
		for the unique set of numbers $c_i$. As a matrix equation, this means
		\be 
		\begin{pmatrix}
			f_1(x_0)&f_2(x_0)&\dots &f_n(x_0)\\
			f_1'(x_0)&f_2'(x_0)&\dots &f_n'(x_0)\\
			\dots\\
			f_1^{(n-1)}(x_0)&f_2^{(n-1)}(x_0)&\dots &f_n^{(n-1)}(x_0)
		\end{pmatrix}\begin{pmatrix}
			c_1\\c_2\\\dots\\c_n
		\end{pmatrix}=\begin{pmatrix}
			f(x_0)\\f'(x_0)\\\dots\\f^{(n-1)}(x_0)
		\end{pmatrix}
		\ee  
		We can find out the unique $c_i$ only if we can invert the matrix, which is only possible if it is full rank, which requires its determinant to be nonzero. That determinant is called Wronskian determinant and its value tells us if the given set of solutions span the solution space or not.
		\item Started talking about Taylor series, how it can be interpreted as an expansion over an infinite dimensional vector space, and how series expansion can turn a differential equation into infinitely many algebraic equations. This is similar to turning vector equations into multiple scalar equations by expanding vectors on a basis and working with the components instead.
		\item Solved explicitly the differential equation  $f''(x)-x f(x)=0$. Note that 
		\begin{itemize}
			\item Not with constant coefficients
			\item not in $\cD_1\.\cD_2\.f=0$ form
			\item cannot do reparametrization as $\frac{q'+2pq}{2q^{3/2}}$ is not constant for $p=0$ and $q=x$
			\item $f(x)$ is not missing and we do not know one of the solutions (cannot reduce order)
			\item equation is not exact, hence cannot be rewritten as a nonhomogeneous lower-order equation
		\end{itemize}
		So we have to use series expansion!
		
		\item Discussed expansions around different points, i.e. $f(x)=\sum\limits_{n=0}^\infty a_n(x-x_0)^n$
		
		\item Introduced the classification of expansion points: given the differential equation 
		\be 
		\left[P_n(x)\rdr{^n}{x^n}+P_{n-1}(x)\rdr{^{n-1}}{x^{d-1}}+\dots+ P_0(x)\right]f(x)=0
		\ee 
		for the analytic functions $P_i(x)$, a point $x_0$ is called ``ordinary point'' if $\frac{P_i(x)}{P_n(x)}$ is analytic for all $i$, and is called ``singular point'' otherwise. For instance $x=0$ is an ordinary points of $\left(x\rdr{}{x}+\sin(x)\right)f(x)=0$.
		\item Around ordinary points, Taylor series expansion works and one can get all solutions correctly.
		\item  Rewrite the differential equation above as 
		\be 
		\left[\rdr{^n}{x^n}+Q_{n-1}(x)\rdr{^{n-1}}{x^{d-1}}+\dots+ Q_0(x)\right]f(x)=0
		\ee
		If $Q_{n-i}(x)$ has a pole of order at most $i$ at $x_0$ for all $i$, then the singular point is called ``regular singular point''. Otherwise, it is called ``essential singular point'' (or non-regular singular point).
		\item Around regular singular points, we can use Taylor series expansion with an unknown monomial as an overall factor; i.e. we can take $f(x)=(x-x_0)^r\sum\limits_{n=0}^\infty a_n (x-x_0)^n$ for the unknowns $a_n$ and $r$. This is called Frobenius method.
		\item We do not use series expansions around essential singular points; even if there is a way to do that, I do not know!
		\item In class, we solved explicitly the diff eqn.
		\be 
		x^2f''(x)-xf'(x)+(1+x)f(x)=0
		\ee 
		around $x=0$.
		
		\item Discussion of how series solutions are \emph{local solutions} in the complex plane: they have (usually) finite radius of convergence, and one needs to construct different series solutions to access values of the function in different locations of the complex plane. In some cases, we might get lucky as we can recognize the series series as a particular representation of a more general function such as Bessel function; in such cases, with one series solutions, we can discover more general properties of the solution for the given differential equation. However, if we are not lucky, we need to construct other series solutions for different points, and we cannot really see the whole picture.
		\item Another problem with the series solutions is that they do not make use of the global properties of the unknown function such as its symmetries. For instance, if I'm trying to solve a differential equation and I know that the solution function should have a symmetry (such as $f(x+a)=f(x)$ for some $a$), this information should in principle help me constraint the solution further. But as series expansions focus on local properties (such as analyticity in and around expansion point), they do not make use of such global information.
		\item If we know some global properties of a function (such as it being spherically symmetric), we may be better off with a different kind of expansion. In fact, there are infinitely many different expansions (in group theoretical language, this is because we can use unitary irreducible representation of any group as a basis). To understand that, we need to discuss the concept of unitarity.
		\item We started a new discussion: concept of unitarity. To understand that, we introduced the following definitions:
		\bea 
		*{}::{}&{}\C\to\C\\
		*{}={}&{}z\to z^*=\Re{z}-i\Im{z}\\
		T{}::{}&{}\mathfrak{M}_{n\x n}(\C)\to\mathfrak{M}_{n\x n}(\C)\\
		T{}={}&{}\begin{pmatrix}
			a_{11}&a_{12}&\dots & a_{1n}\\
			a_{21}&a_{22}&\dots & a_{2n}\\
			\dots \\
			a_{n1}&a_{n2}&\dots & a_{nn}
		\end{pmatrix}\to\begin{pmatrix}
			a_{11}&a_{21}&\dots & a_{n1}\\
			a_{12}&a_{22}&\dots & a_{n2}\\
			\dots \\
			a_{1n}&a_{2n}&\dots & a_{nn}
		\end{pmatrix}\\
		\dagger{}::{}&{}\mathfrak{M}_{n\x n}(\C)\to\mathfrak{M}_{n\x n}(\C)\\
		\dagger{}={}&{}\begin{pmatrix}
			a_{11}&a_{12}&\dots & a_{1n}\\
			a_{21}&a_{22}&\dots & a_{2n}\\
			\dots \\
			a_{n1}&a_{n2}&\dots & a_{nn}
		\end{pmatrix}\to\begin{pmatrix}
			a_{11}^*&a_{21}^*&\dots & a_{n1}^*\\
			a_{12}^*&a_{22}^*&\dots & a_{n2}^*\\
			\dots \\
			a_{1n}^*&a_{2n}^*&\dots & a_{nn}^*
		\end{pmatrix}
		\eea 
		where $*,T,\dagger$ are called to output \emph{the complex conjugate}, \emph{the transpose}, and \emph{the hermitian conjugate} of the input respectively; for illustration, $A^\dagger$ is called the hermitian conjugate of the matrix $A$.
		\item A \emph{unitary} ordinary number is a complex number with unit length, i.e. $z$ with $\abs{z}=z z^*=1$. To matrices, this can be generalized with the hermitian conjugation: \emph{a unitary matrix $A$ is a matrix such that $A\. A^\dagger=\mathbb{I}$ for the unit matrix $\mathbb{I}$.} 
		\item Unitarity can be generalized beyond those inputs: an invertible object $U$ is called \emph{unitary} if it satisfies the condition $U^\dagger=U^{-1}$. $U$ can be wilder objects in principle, such as an infinite dimensional matrix or a general operator; for instance, $U=\exp(i\rdr{}{x})$ is a unitary operator for $x\in\R$.
		
		\item Concept of unitarity is important, because there exists unitary matrices with functional entries which can be used as a basis to expand any given function. This is similar to Taylor series expansion: there, we used some sort of orthogonality of $x^m$ and $x^n$ for $m\ne n$ and used $\{x^i\}$ as a basis over which we expand $f(x)$. We state that similar basis (in fact, infinitely many of them) exist and we can expand any given function in terms of such basis consisting of an infinite set of particular unitary matrices of functional entries.
		
		\item The modern way to understand such expansions is through \emph{group theory}! Since we will not learn about these details, we only present a very important and general theorem: we will not dwell on its details and we will not try to do actual computations with that. We only present this to emphasize that Fourier transform (or spherical harmonics expansion, or Mellin transform, or many more) are special examples of a very general and fundamental branch of mathematics called \emph{harmonic analysis}!
		
		The main result of harmonic analysis is as follows:
		\bea 
		f(g)=&\int\limits_{\hat G} d\pi \tr(\hat\pi(g)^{-1}\hat f(\pi))\\
		\hat f(\pi)=&\int\limits_G dh \hat\pi(g)f(h)
		\eea 
		\item We will \emph{not} discuss the details of above formula, and no one needs to know the following for this course! What we need to know is that there exist such a general result, and most of the stuff we see around (such as Fourier transform) are special cases of this general result! Nevertheless, for completeness, I list the ingredients as follows:
		\begin{itemize}
			\item $G$: space of a group, with $dh$ being the measure in this space invariant under the action of the group (Haar measure)
			\item $\hat G$: space of the unitary irreducible representations of $G$, with $d\pi$ being the Plancheral measure
			\item $\hat\pi :: g\to\mathrm{End}(V_\pi)$ is a map from the group element $g$ to the space of endomorphisms on the vector space of the representations of the group. Such endomorphisms can be implemented as a matrix acting on this vector space, therefore $\hat\pi$ is a simple matrix on the representation space (which is why we take a trace in the first integral)
			\item $f(g)$ is a function on the group space
			\item $\hat f(\pi)$ is generalization of \emph{Fourier coefficients}: this is a matrix in the representation space
		\end{itemize}
		\item We can make the following analogy:\\
		\begin{tabular}{lllll}
			Object & Basis & Decomposition & Components& Extracting components\\ 
			$\vec{v}$ & $\{\hat i,\hat j,\hat k\}$ & $\vec{v}=v_x\hat i+v_y\hat j+v_z\hat k$ & $\{v_x,v_y,v_z\}$ (a finite set)& $v_x=\vec{v}\.\hat i$\\ 
			$f(x)$&$\{x^i\}$&$f(x)=\sum\limits_{n=0}^\infty a_n x^n$&$\{a_n\}$ (countable infinite set)& {\footnotesize Cauchy's integral formula,see 210}\\
			$f(g)$ &$\{\hat\pi(g)\}$& $f(g)=\int\limits_{\hat G} d\pi \tr(\hat\pi(g)^{-1}\hat f(\pi))$ & $\{\hat f(\pi)\}$ (uncountable infinite set)&$\hat f(\pi)=\int\limits_G dh \hat\pi(g)f(h)$
		\end{tabular}
		
		
		\item Simplest example of Harmonic analysis is the Fourier transform. In this case, $\hat\pi$ is a one-dimensional unitary function $\hat\pi= e^{-i k x}$, $\pi$ are spanned by one continuous real parameter $k$ (meaning $\hat G=\R$), and the original function space is one dimensional real line as well (hence $G=\R$ with $g=x$). One can derive that the Plancheral measure is $\frac{dk}{2\pi}$, hence we have
		\bea 
		f(x)=&\int\limits_{\R}\frac{dk}{2\pi} \left(e^{-ikx}\right)^{-1}\hat f(k)\\
		\hat f(k)=&\int\limits_\R dx e^{-i k x} f(x)
		\eea 
		hence we can specialize the table above for Fourier transform as 
		\begin{tabular}{llll}
			Object & Basis & Decomposition & Components\\ 
			$\vec{v}$ & $\{\hat i,\hat j,\hat k\}$ & $\vec{v}=v_x\hat i+v_y\hat j+v_z\hat k$ & $\{v_x,v_y,v_z\}$ (a finite set)\\ 
			$f(x)$&$\{x^i\}$&$f(x)=\sum\limits_{n=0}^\infty a_n x^n$&$\{a_n\}$ (countable infinite set)\\
			$f(x)$ &$\{e^{ikx}\}$& $f(x)=\int\limits_{-\infty}^\infty dk e^{ikx} \hat f(k)$ & $\left\{\hat f(k)=\int\limits_{-\infty}^\infty e^{-ikx}f(x)\right\}$ (uncountable infinite set)
		\end{tabular}
		\item We can now use periodicity information if we know that a function $f$ satisfies $f(x)=f(x+a)$: the identification $x\sim x+a$ means some of our basis components should be absent as we dictate $e^{ikx}\sim e^{ik(x+a)}$. We see that this is possible only if $k$ is actually discrete, i.e. 
		\be 
		k=\frac{2\pi}{a}n\text{ for }n\in\Z
		\ee
		This leads to discrete time fourier series:
		\bea 
		f(x)=&\frac{1}{a}\sum\limits_{n=-\infty}^{\infty}e^{i\frac{2\pi n}{a}x}\hat f(n)\\
		\hat f(n)&=\int\limits_0^a dx e^{-i\frac{2\pi n}{a} x} f(x)
		\eea  
		
		\item In class, we introduced four different Fourier analysis:
		\begin{itemize}
			\item Fourier transform:
			\bea 
			f{}::{}&{}\C\to\C\\
			\hat f{}::&{}\C\to\C\\ f(x)=&\int\limits_{-\infty}^\infty\frac{dk}{2\pi} e^{ikx}\hat f(k)\\
			\hat f(k)=&\int\limits_{-\infty}^\infty dx e^{-i k x} f(x)
			\eea 
			\item Fourier series:
			\bea 
			f{}::{}&{}[a,a+T]\to\C\\
			\hat f{}::&{}\Z\to\C\\ 	f(x)=&\frac{1}{T}\sum\limits_{n=-\infty}^\infty e^{i\frac{2\pi n}{T}x}\hat f(n)\\
			\hat f(n)=&\int\limits_{b}^{b+T} dx e^{-i\frac{2\pi n}{T}x} f(x)
			\eea 
			for arbitrary $a,b\in\R$. Note that this is also applicable for periodic functions: for any periodic function with a period $T$, we can use Fourier series as instructed above!
			
			\item Discrete-time Fourier transform:
			\bea 
			f{}::&{}\Z\to\C\\ 
			\hat f{}::{}&{}[a,a+T]\to\C\\
			f(n)=&\frac{1}{T}\int\limits_{b}^{b+T} dx e^{i\frac{2\pi n}{T}k} \hat f(k)\\
			\hat f(k)=&\sum\limits_{n=-\infty}^\infty e^{-i\frac{2\pi n}{T}k} f(n)\\
			\eea
			where $\Z_N$ denotes the set $\{0,1,\dots,N-1\}$.
			
			\item Discrete Fourier series:
			\bea 
			f{}::&{}\Z_N\to\Z_N\\ 
			\hat f{}::{}&{}\Z_N\to\Z_N\\ 
			f(n)=&\frac{1}{N}\sum\limits_{m=0}^{N-1} e^{i\frac{2\pi n m}{T}} \hat f(m)\\
			\hat f(m)=&\sum\limits_{n=0}^{N-1} e^{-i\frac{2\pi n m}{T}} f(n)\\
			\eea
			for arbitrary $a,b\in\R$.
		\end{itemize}
		\item Summary: finite range/periodic in one domain $\leftrightarrow$ discrete in the other domain
		\item Drew plots of these transformations for visualization in class.
		
		\item Introduced two higher order functions $E$ and $O$:
		\bea 
		E{}::&{}\left(\C\to\C\right)\to\left(\C\to\C\right)\\ 
		E={}&{}\left(x\to f(x)\right)\to\left(x\to f_E(x)=\frac{f(x)+f(-x)}{2}\right)\\
		O{}::&{}\left(\C\to\C\right)\to\left(\C\to\C\right)\\ 
		O={}&{}\left(x\to f(x)\right)\to\left(x\to f_E(x)=\frac{f(x)-f(-x)}{2}\right)
		\eea
		with which any single-argument function satisfies $f=E\.f+O\.f$, or with a more common notation, $f(x)=f_E(x)+f_O(x)$. As any function can also be decomposed into its real and imaginary part, we arrive at
		\be 
		f(x)=f_{RE}(x)+f_{RO}(x)+i f_{IE}(x)+i f_{IO}(x)
		\ee 
		
		\item Since Fourier transform is linear, we can reconstruct $(\mathrm{F.T.}\.f)(k)$ from $(\mathrm{F.T.}\.f_{RE})(k)$ and so on.
		
		\item $f_{RE}(k)$ etc. and their Fourier transforms satisfy nice properties. We derived a few of them in the class, for instance taking complex conjugation and changing the dummy variable $x\to -x$ leads to
		\be 
		\left[(\mathrm{F.T.}\.f_{RE})(k)\right]^*=(\mathrm{F.T.}\.f_{RE})(k)
		\ee 
		Likewise, taking $k\to -k$ and $x\to -x$ leads to
		\be 
		(\mathrm{F.T.}\.f_{RE})(-k)=(\mathrm{F.T.}\.f_{RE})(k)
		\ee 
		hence we conclude Fourier transform of $f_{RE}$ is itself even and real. On the contrary, Fourier transform of $F_{RO}$ is purely-imaginary and odd.
		\item For real functions,  $(\mathrm{F.T.}\.f_{R})(-k)=\left[(\mathrm{F.T.}\.f_{R})(k)\right]^*$, hence we only need to know positive frequencies, consistent with our everyday experience.
		\item Consider the Fourier transform of the function 
		\bea 
		f{}::&{}\left[-\frac{T}{2},\frac{T}{2}\right]\to\C\\
		f={}&{}x\to 1
		\eea 
		We computed Fourier series expansion of this in class, found $\hat f(n)=T\de_{n0}$ and checked consistency. We then considered the limit $T\to\infty$ and argued that Kronecker-delta function turns into Dirac-delta distribution.
		\item We also considered the function
		\bea 
		f{}::&{}\R\to\C\\
		f={}&{}x\to \left\{\begin{aligned}
			1\quad&\abs{x}\le \frac{T}{2}\\0\quad&\text{ otherwise}
		\end{aligned}\right.
		\eea 
		We computed the Fourier transformation of this, introduced the $\mathrm{sinc}$ function, discussed its properties, talked about its usage in single-slit experiment, optics, and signal process. We then also discussed how $T\to\infty$ takes $\mathrm{sinc}$ functio to the Dirac-delta distribution.
		
\end{enumerate}}


\section{Particular solution}
\label{chapter: Linear nonhomogeneous equations with functional coefficients}
\draftnote{
	Summary of what we have discussed in class (to be typed later, maybe next year):
	\begin{enumerate}
		\item The particular solution to a differential equation is unique: we haven't proven this but it is actually most easily seen if the unknown function is expanded in a basis where the action of the differential operator becomes \emph{diagonal}; i.e., the differential equation then becomes algebraic and there is a unique solution for an algebraic equation $ax=b$ for $x$.
		\item As particular solutions are unique, we can get away with guessing it if we can: the simplest way to find the particular solution is to guess it and then check that it satisfies the differential equation.
		\item The next simplest thing we can try is to guess the \emph{functional form} of the particular solution with some arbitrary coefficients and then fix them imposing the differential equation. This approach is called \emph{method of undetermined coefficients}. As an example, for
		\be 
		x f'(x)-2f(x)=6 x^4
		\ee 
		we can guess $f_p(x)= a x^b$: inserting it into the differential equation, we find that $f_p(x)=2x^4$.
		\item For differential equations with constant coefficients, we can find the particular solution more systematically as we have reviewed in the beginning of the semester: convolution of nonhomogeneous piece with the impulse response.
		\item A similar systematic approach exists for more general differential equations: it is called \emph{method of variation of parameters}. Let us consider a general linear ordinary differential equation
		\be 
		\label{temp-2}
		\left(\rdr{^n}{x^n}+a_{n-1}(x)\rdr{^{n-1}}{x^{n-1}}+a_0(x)\right)f(x)=h(x)
		\ee
		If we find the homogeneous solutions $f_1(x),\dots,f_n(x)$, then we know that the most general solution can be written as 
		\be 
		f(x)=\sum\limits_{i=1}^nc_i f_i(x)+f_p(x)
		\ee  
		for arbitrary coefficients $c_i$. Here, we have one unknown function $f_p(x)$ and we trade it for $n$ unknown functions by rewriting this equation as 
		\be
		\label{temp-1}
		f(x)=\sum\limits_{i=1}^nc_i(x) f_i(x) 
		\ee 
		for undetermined functions $c_i(x)$, i.e. we \emph{varied the parameters}. We can clearly do this, as we had 1 unknown functional degree of freedom and know we have $n$ unknown functional degrees of freedom. In fact, we can impose $n-1$ constraints, which we choose as 
		\be 
		\label{temp-3}
		\sum\limits_{i=1}^nc_i'(x)f_i^{(k-1)}(x)=0\qquad\text{ for }k=1,2,\dots,n-1
		\ee 
		If we know insert \eqref{temp-1} into \eqref{temp-2} and use these constraints and the fact that $f_i(x)$ are homogeneous solutions, we end up with
		\be 
		\label{temp-4}
		\sum\limits_{i=1}^nc_i'(x)f_i^{(n-1)}(x)=g(x)
		\ee 	
		The equations \eqref{temp-3} and \eqref{temp-4} can be combined to solve for $c_i'(x)$ as 
		\be 
		\begin{pmatrix}
			c_1'(x)\\c_2'(x)\\\dots\\c_n'(x)
		\end{pmatrix}=\begin{pmatrix}
			f_1(x)&f_2(x)&\dots&f_n(x)\\
			f_1'(x)&f_2'(x)&\dots&f_n'(x)\\
			\dots \\
			f_1^{(n-1)}(x)&f_2^{(n-1)}(x)&\dots&f_n^{(n-1)}(x)\\
		\end{pmatrix}^{-1}	\begin{pmatrix}
			0\\0\\\dots\\g(x)
		\end{pmatrix}
		\ee 
		\item In summary, inverse of the Wronskian matrix and the nonhomogeneous piece is sufficient to find $c_i'(x)$. By integrating these, we get both the homogeneous solution (through the integration constants) and particular solution.
		\item In practice, computer programs (such as Mathematica) are the best way to compute matrix inverses (avoid pen and paper if you can). However, it is better if we learn the math behind such implementations. To understand the computation of a matrix inverse, we need to define a new operation:
		\bea 
		\mathrm{adj}{}::&{}\cM_{n\x n}(\C)\to\cM_{n\x n}(\C)\\
		\mathrm{adj}={}&{}\begin{pmatrix}
			a_{11}&a_{12}&\dots & a_{1n} \\
			a_{21}&a_{22}&\dots & a_{2n} \\
			\dots \\
			a_{n1}&a_{n2}&\dots & a_{nn}
		\end{pmatrix}\to \begin{pmatrix}
			b_{11}&b_{12}&\dots & b_{1n} \\
			b_{21}&b_{22}&\dots & b_{2n} \\
			\dots \\
			b_{n1}&b_{n2}&\dots & b_{nn}
		\end{pmatrix}\quad\text{ where }\\
		&{}b_{i_n k_n}=\frac{1}{(n-1)!}\e_{i_1\dots i_n}\e_{k_1\dots k_n} a_{i_1k_1}\dots a_{i_{n-1}k_{n-1}}
		\eea 
		where $\mathrm{adj}$ yields the \emph{adjugate} of the given matrix. We can now give the inverse of a matrix as
		\be 
		A^{-1}=\frac{\mathrm{adj}(A)}{\det A}
		\ee 
		where we have defined and discussed determinant before:
		\bea 
		\det::{}&{}\mathfrak{M}_{n\x n}(\C)\to\C\\
		\det={}&{}\begin{pmatrix}
			a_{11}&a_{12}&\dots & a_{1n}\\
			a_{21}&a_{22}&\dots & a_{2n}\\
			\dots \\
			a_{n1}&a_{n2}&\dots & a_{nn}
		\end{pmatrix}\to\sum\limits_{i_1,\dots,i_n}\e_{i_1\dots i_n}a_{1i_1}\dots a_{ni_n}
		\eea 
		\item Computed adjugate and inverse of a $2\x 2$ matrix in the class.
	\end{enumerate}
}
\chapter{Systems of first order linear differential equations}
\draftnote{
	Summary of what we have discussed in class (to be typed later, maybe next year):
	\begin{enumerate}
		\item We reviewed in class that we have learned various methods to solve differential equations. In this chapter, we will see one last method: \emph{conversion of generic linear ordinary differential equations to first order system of differential equations}.
		\item Learning how to solve first order differential equations of matrices is important for multiple reasons:
		\begin{enumerate}
			\item Any linear ordinary differential equation can be rewritten as a first order differential equation of a column matrix.
			\item There exists coupled systems which can only be solved through differential equations of matrices
			\item There is a conceptual reason to consider such diffential equations of matrices: phase space, using momenta, etc.
		\end{enumerate}
		\item We introduced systems of linear ordinary differential equations: multiple unknown functions, one independent variable.
		\item As an example, we considered a one-dimensional system of two masses attached to a wall via two springs (something like \mbox{$|\sim\square\sim\square$}), and showed that this system is described by a system of two differential equations. The unknown functions are two positions of the masses, the independent variable is the time, and the key point is that this is a coupled system: the differential equations cannot be solved independently! Indeed, the system is best described as 
		\be 
		\rdr{^2}{t^2}\begin{pmatrix}
			x_1\\x_2
		\end{pmatrix}=\begin{pmatrix}
			-\frac{k_1+k_2}{m_1}&\frac{k_2}{m_1}\\\frac{k_2}{m_2}&-\frac{k_2}{m_2}
		\end{pmatrix}\begin{pmatrix}
			x_1\\x_2
		\end{pmatrix}
		\ee 
		\item One can solve this differential equation by various ways. One common method is to find the normal modes of this system by diagonalizing the square matrix: this will tell us the combinations of $x_1$ and $x_2$ which oscillate independently: for those variables, we have two independent differential equations that can be solved separately. We will not discuss this further in this class.
		\item Another approach to solve such a differential equation is to bring it to a first order form. We motivate this by using our physical intuition: momentum (in addition to position) also describes a current state of the system, so we should work with four variables instead of two! Indeed, we can rewrite the above differential equation as 
		\be 
		\rdr{}{t}\begin{pmatrix}
			x_1\\p_1\\x_2\\p_2
		\end{pmatrix}=\begin{pmatrix}
			0&\frac{1}{m_1}&0&0\\-(k_1+k_2)&0&k_2&0\\0&0&0&\frac{1}{m_2}\\k_2&0&-k_2&0
		\end{pmatrix}\begin{pmatrix}
			x_1\\p_1\\x_2\\p_2
		\end{pmatrix}
		\ee 
		\item This approach is generalizable to a set of differential equations for $m$ unknown functions where the highest order derivative for each unknown function is $n_1$, $n_2$,$\dots$, $n_m$. This system is equivalent to a first-order differential equation for a vector of $\sum\limits_{i=1}^m n_i$ components!
		
		\item As example, we solved $f''(x)+3f'(x)+2f(x)=0$ by converting it into the matrix equation
		\be 
		\rdr{}{x}V(x)+A\. V(x)=0
		\ee 
		for 
		\be 
		A=\begin{pmatrix}
			0&-1\\2&3
		\end{pmatrix}
		\ee
		and
		\be 
		V(x)=\begin{pmatrix}
			f(x)\\f'(x)
		\end{pmatrix}
		\ee 
		We immediately say that the solution is 
		\be 
		V(x)=(\exp{-A x})\. C
		\ee 
		for the arbitrary column matrix $C$.
		\item We computed $\exp(A x)$ explicitly in class using its definition:
		\be 
		\exp(A x)=\mathbb{I}+x A+\frac{x^2}{2!}A\.A+\frac{x^3}{3!}A\.A\.A+\dots
		\ee 
		\item In class, we discussed that if $A$ is $t-$independent, then we can use the generalization
		\be 
		\left[f'(t)=af(t)\rightarrow f(t)=\exp(at)c\right]\Rightarrow\left[V'(t)=A\.V(t)\rightarrow V(t)=\exp(At)\.C\right]
		\ee
		which means for any first-order differential equation with constant $A$ is immediately solvable this way. A similar generalization exists even if $A$ is $t-$dependent, but it is nontrivial: 
		\be 
		f'(t)=a(t)f(t)\rightarrow f(t)=\exp\left(\int a(t)dt\right)c
		\ee
		however
		\be 
		V'(t)=A(t)\.V(t)\rightarrow V(t)\ne \exp\left(\int A(t)dt\right)\.C
		\ee 
		The correct version is 
		\be 
		V'(t)=A(t)\.V(t)\rightarrow V(t)=\mathcal{T}\left\{\exp\left(\int A(t)dt\right)\right\}\.C
		\ee 
		where $\cT$ is called \emph{time-ordering operator}! There is a purely mathematical derivation of this via Volterra integral equation (we will see this), but the physical implications are rather important in its usage in quantum mechanics: the measurements at different times (by $A(t)$) should be time-ordered, i.e. causality has to be preserved! You will learn more about this when you solve Schrödinger's equation (a first order diff. equation of operators, which can be represented with matrices)!
		\item We solved in class $f''(x)+3f'(x)+2f(x)=0$ as an example. We already know that the characteristic equation is $(r+2)(r+1)=0$ hence the answer is $f(x)=c_1 e^{-x}+c_2 e^{-2x}$. Nevertheless, let's see how we can get this answer through matrix computation.
		
		We realize that this equation can be rewritten as
		\be 
		\rdr{}{x}\begin{pmatrix}
			f(x)\\f'(x)
		\end{pmatrix}=\begin{pmatrix}
			0&1\\-2&-3
		\end{pmatrix}\begin{pmatrix}
			f(x)\\f'(x)
		\end{pmatrix}
		\ee 
		hence we can immediately write down the answer as
		\be 
		\begin{pmatrix}
			f(x)\\f'(x)
		\end{pmatrix}=e^{\begin{pmatrix}
				0&1\\-2&-3
		\end{pmatrix}}\begin{pmatrix}
			c_1\\c_2
		\end{pmatrix}
		\ee 
		
		In class, we calculated the exponentiation of this matrix by computing first few terms in the Taylor expansion, finding the pattern, and then resumming all terms.
		
		\item We emphasize that this approach does not wotk if the matrix is not constant: as we mentioned above, we instead need the \emph{time-ordering}. In class, we schematically derived this as follows.
		
		Start with
		\be 
		\rdr{}{t}V(t)=A(t)\.V(t)
		\ee  
		Any non-homogeneous piece can be ignored as the particular solutions can always be found via variation of parameters method \emph{after} homogeneous solutions are computed. By integrating this equation, we get \emph{Volterra integral equation}
		\be 
		V(t)=V(0)+\int\limits_{0}^t A(t')\. V(t')dt'
		\ee 
		Now consider a modified version of this as 
		\be 
		V(\e,t)=V(0)+\e\int\limits_{0}^t A(t')\. V(\e,t')dt'
		\ee
		where $V(t)=V(1,t)$. Let us expand $V(\e,t)$ around $\e=0$ as follows:
		\be 
		V(\e,t)=V_{(0)}(t)+\e V_{(1)}(t)+\e^2 V_{(2)}(t)+\dots
		\ee 
		Note that $V(0,t)=V_{(0)}(t)=V(0)$ so this series is definitely well defined around $\e\sim 0$, but we do not yet know its radius of convergence so we might in principle not be able to take $\e\to 1$ to  obtain $V(t)$ at the end. Nevertheless, we will see that the radius of convergence is actually infinite as the expansion will turn out to be that of the exponential function. For now, let us proceed by inserting this into the modified Volterra equation:
		\be 
		\left(V(0)+\e V_{(1)}(t)+\e^2 V_{(2)}(t)+\dots\right)=V(0)+\e\int\limits_{0}^t A(t')\. \left(V(0)+\e V_{(1)}(t')+\e^2 V_{(2)}(t')+\dots\right)dt'
		\ee
		which leads to
		\be 
		V_{(n+1)}(t)=\int\limits_{0}^t A(t')\. V_{(n)}(t')dt'
		\ee
		if we match different orders of $\e$. But inserting this back, we obtain
		\be 
		V(\e,t)=V(0)+\left[\e\int\limits_{0}^t A(t')dt'\right]\. V(0)+\left[\e^2\int\limits_{0}^t dt' \int\limits_{0}^{t'} dt'' A(t')\. A(t'')\right]\. V(0)+\dots 
		\ee 
		which can be rewritten as 
		\be 
		\label{temp1}
		V(\e,t)=\left(\mathbb{I}+\left[\e\int\limits_{0}^t A(t')dt'\right]+\left[\e^2\int\limits_{0}^t dt' \int\limits_{0}^{t'} dt'' A(t')\. A(t'')\right]+\dots\right)\. V(0)
		\ee 
		
		One can show that the integration ranges $\int\limits_{0}^t dt' \int\limits_{0}^{t'} dt''$ and $\int\limits_{0}^t dt'' \int\limits_{0}^{t''}dt'$ actually cover two triangles that add upto a square in the $t'-t''$ plane, given by the integration range $\int\limits_{0}^t dt' \int\limits_{0}^{t}dt''$. Therefore, we see that 
		\be 
		\int\limits_{0}^t dt' \int\limits_{0}^{t}dt'' f(t',t'')=\int\limits_{0}^t dt' \int\limits_{0}^{t'} dt'' f(t',t'')+\int\limits_{0}^t dt'' \int\limits_{0}^{t''}dt' f(t',t'')
		\ee 
		which can be rewritten by changing the dummy variable as 
		\be 
		\int\limits_{0}^t dt' \int\limits_{0}^{t}dt'' f(t',t'')=\int\limits_{0}^t dt' \int\limits_{0}^{t'} dt''\left(f(t',t'')+f(t'',t')\right)
		\ee 
		Observe that if we define 
		\be 
		f(t,t')=\left\{\begin{aligned}
			g(t,t')\text{ if }t>t'\\
			g(t',t)\text{ if }t'>t
		\end{aligned}\right.
		\ee 
		then we get
		\be 
		\half\int\limits_{0}^t dt' \int\limits_{0}^{t}dt'' f(t',t'')=\int\limits_{0}^t dt' \int\limits_{0}^{t'} dt''g(t',t'')
		\ee 
		Note that $f$ is just the function $g$ in a way that its arguments are rearranged in the decreasing order. If we then define \emph{time-ordering} $\cT$ as 
		\be 
		\cT::{}&{} (\R^n\to \C)\to(\R^n\to \C)\\
		\cT={}&{} \Big((t_1,\dots,t_n)\to f(t_1,\dots,t_n)\Big)\to\Big((t_1,\dots,t_n)\to f\.\cO_>\.(t_1,\dots,t_n)\Big)
		\ee 
		where $\cO_>$ is the ordering function acting on \emph{list of real numbers}
		\be 
		\cO_>::{}&{} \R^n\to\R^n\\
		\cO_>={}&{}(t_1,\dots,t_n)\to (t_{i_1},\dots,t_{i_n}) \text{ such that } t_{i_a}\le t_{i_b}\text{ if }a<b
		\ee 
		we then see that 
		\be 
		\int\limits_{0}^t dt' \int\limits_{0}^{t'} dt''g(t',t'')=\half\int\limits_{0}^t dt' \int\limits_{0}^{t}dt'' \cT\.g(t',t'')
		\ee 
		One can actually observe the same pattern in higher order versions, hence
		\be 
		\int\limits_{0}^t dt_1 \int\limits_{0}^{t_1} dt_2  \int\limits_{0}^{t_{n-1}} dt_n g(t_1,\dots,t_n)=\frac{1}{n!}\int\limits_{0}^t dt_1\dots \int\limits_{0}^t dt_n \cT\.g(t_1\dots,t_n)
		\ee 
		
		One can use this result in \eqref{temp1} and after some manipulation, we obtain
		\be
		V(\e,t)=\cT\left(e^{\e\int\limits_0^t A(t') dt'}\right)\. V(0)
		\ee 
		where $\cT$ is understood to apply products of $A$ when exponential is expanded. Since the Taylor expansion of exponential function has infinite radius of convergence, we can take $\e\to 1$ and obtain the final result:
		\be
		V(t)=\cT\left(e^{\int\limits_0^t A(t') dt'}\right)\. V(0)
		\ee 
		In Physics, this is mostly known as Dyson series.
		\item In class, I emphasized that this is a hard topic and the students are not expected to understand it fully. No question was asked about this in any homework or examination.
\end{enumerate}}

\chapter{Eigensystems and Sturm-Lioville theory}
\draftnote{
	\begin{enumerate}
		\item Getting back to the simple case of constant $A(t)$, we solved some explicit examples in class. We considered an RLC circuit of resistor, capacitor, and inductor in parallel (and nothing else). We have shown in class that this system is coupled, hence one needs to work with matrices to find the, say, current over inductor as a function of time if there was initial energy in the system. Indeed, the system is described by
		\be 
		\rdr{}{t}\begin{pmatrix}
			I_L(t)\\ V(t)
		\end{pmatrix}= \begin{pmatrix}
			0&1/L\\-1/C&-1/(RC)
		\end{pmatrix}\begin{pmatrix}
			I_L(t)\\ V(t)
		\end{pmatrix}
		\ee 
		We solved in class for $C=1 mF$, $L=200 H$, and $R=1/6 kOhm$ via exponentiation of the matrix. As teasor, I also showed that the differential equations for the variables $S_1=I_L+10^{-3}V$ and $S_2=I_L+5 10^{-3}V$ are actually decoupled: these are called normal mods, and we will learn about how to get these (finding the eigenvalues and eigenfunctions of a system) below!
		
		\item How to compute exponential of a matrix? As an example, consider
		\be 
		e^{\begin{pmatrix}
				a&b\\c&d
		\end{pmatrix}}=&\begin{pmatrix}
			1&0\\0&1
		\end{pmatrix}+\begin{pmatrix}
			a&b\\c&d
		\end{pmatrix}+\half \begin{pmatrix}
			a&b\\c&d
		\end{pmatrix}\.\begin{pmatrix}
			a&b\\c&d
		\end{pmatrix}+\frac{1}{3!}\begin{pmatrix}
			a&b\\c&d
		\end{pmatrix}\.\begin{pmatrix}
			a&b\\c&d
		\end{pmatrix}\.\begin{pmatrix}
			a&b\\c&d
		\end{pmatrix}+\dots
		\\
		=&\begin{pmatrix}
			f_{11}(a,b,c,d)&f_{12}(a,b,c,d)\\f_{21}(a,b,c,d)&f_{22}(a,b,c,d)
		\end{pmatrix}
		\ee 
		This is tedious!
		
		Observe that 
		\be 
		A=U\.D\.U^{-1} \quad e^A=U\.e^D\.U^{-1}
		\ee 
		which one can check by expanding matrices. If $D$ is diagonal, then
		\be 
		e^A=U\.\begin{pmatrix}
			e^{\lambda_1}& 0& \dots & 0\\
			0& e^{\lambda_2}& \dots & 0\\
			\dots \\
			0&\dots & 0 & e^{\lambda_n}
		\end{pmatrix}\. U^{-1}
		\ee 
		where $\lambda_i$ are diagonal entries of the matrix $D$.
		\item We now try to solve the question: how do we find the matrix $U$ for a given $A$? If we write $U$ in terms of its column vectors as $U=\begin{pmatrix}
			\vec{u}_1&\vec{u}_2&\dots&\vec{u}_n
		\end{pmatrix}$, then $A\.U=U\.D$ implies
		\be 
		A\.\begin{pmatrix}
			\vec{u}_1&\vec{u}_2&\dots&\vec{u}_n
		\end{pmatrix}=\begin{pmatrix}
			\lambda_1\vec{u}_1&\l_2\vec{u}_2&\dots&\l_n\vec{u}_n
		\end{pmatrix}
		\ee 
		which implies 
		\be
		\label{temp2} 
		A\.\vec{u}_i=\lambda_i \vec{u}_i\quad \forall i
		\ee 
		$u_i$ (we'll drop the vector sign for simplicity) are called \emph{right eigenvectors} of $A$, and $\l_i$ are called \emph{eigenvalues} of $A$. We can find an analogous equation in terms of \emph{left eigenvectors} of $A$ ($\vec{v}_i\.A=\vec{v}_i\l_i$), where $\vec{v}$ are actually row vectors of $U$ (unlike $\vec{u}$, which are column vectors of $U$). We'll only work with right eigenvectors so we'll drop the adjective right from now on.
		
		\item Observe that finding the matrix $U$, which is also called \emph{modal matrix}, is equivalent to solving the equation \eqref{temp2}, which can be rewritten as
		\be 
		\left(A-\lambda_i\mathbb{I}\right)\. u_i=0
		\ee
		If $\left(A-\lambda_i\mathbb{I}\right)$ is invertible, then we contract the equation above with $\left(A-\lambda_i\mathbb{I}\right)^{-1}$ from the left, which gives $u_i=\left(A-\lambda_i\mathbb{I}\right)^{-1}\.0$ hence the only solution is the trivial solution $u_i=0$. Therefore, for nontrivial $u_i$, we need $\left(A-\lambda_i\mathbb{I}\right)$ to be non-invertible, which means its determinant is zero:
		\be 
		\det \left(A-\lambda_i\mathbb{I}\right) =0
		\ee 
		This gives us a polynomial in $\l_i$, which is called \emph{characteristic polynomial} of $A$. For the matrix $A$ that describes a single linear ordinary differential equation with constant coefficients, the characteristic polynomial reduces to the characteristic equation that we have learned earlier.
		
		\item In class, we have solved two examples: the RLC circuit earlier, and $f''(x)+3f'(x)+2f(x)=0$. We found their eigensystem (and hence normal modes).
		\item Discussed physical importance of this \emph{similarity transformation}, i.e. 
		\be 
		A=U\.D\.U^{-1}
		\ee 
		where $U$ is called a similarity transformation and the diagonal matrix $D$ is called the spectrum. This mathematical terminology is in line with our physical intuitions and usage of the term spectrum. Indeed, in optics, electromagnetic theory, atomic \& nuclear physics, and chemical physics, we use the term spectrum to refer to the different frequencies of light in the emission (or absorption) spectrum of  some system/material. These different frequencies are actually the eigenvalues of the system as monochromatic light with different frequencies are \emph{eigenfunctions} that form a basis on which any electromagnetic disturbance can be decomposed. This is most easily seen in a prism where light right gets decomposed into monochromotic light (remember Pink Floyd :D): this is simply expansion of white light in the basis of eigenfunctions of the light ($e^{ikx}$). In fact, the Harmonic analysis that we have learned before (Fourier transform being one example) are generally expansions over eigenfunctions of a differential operator! (that differential operator being Casimir operator and eigenfunctions being unitary irreducible representations of the relevant group, but this is a story for another time!)
		\item Extending this whole discussion of spectrum and matrix transformation to infinite dimensions is equivalent to extending our eigenvalue equation to functions from finite dimensional column vectors: we now have
		\be 
		\cD\. f(x)=\l f(x)
		\ee 
		for a differential operator $\cD$, where $f(x)$ is its eigenfunction and $\l$ is its eigenvalue.
		\item The analysis of eigensystem of operators is immensely important, but to make further progress, we need to define \emph{inner products}!
		\item Remind dot products over real vector spaces $(.)::(\R^n,\R^n)\to \R$, extend them to functions $\R\to\R$ by defining the inner product $\<,\>::(\R\to\R,\R\to\R)\to\R$ as $\<f,g\>=\int\limits_\R f(x)g(x)dx$, and then generalize this further to complex functions as 
		\bea
		\<,\>_\w::{}&{}(A\to\C,A\to\C)\to\C \\
		\<f,g\>_\w={}&{}\int\limits_A \Big(f(x)\Big)^*g(x)\w(x)dx
		\eea
		for $A\subseteq\R$.
		\item In class, we defined adjoint of an operator wrt an inner product, discussed its properties, showed that self-adjoint operators have a real spectrum and orthogonal eigenfunctions. Add further discussion regarding symmetric vs self-adjoint operators.
		\item Sturm-Liouville problem is basically analysis of the eigensystem of \emph{second order self adjoint differential operators}.
	\end{enumerate}
}

\chapter{Beyond linear ordinary differential equations}
\begin{enumerate}
	\item Talked about abstract concept of a system S, with input I, and with output O, where their types are $\cS$, $\cI$, and $\cO$. Talked about how systems can be represented in different frameworks, as differential equations, as algebraic equations, or etc. Discussed how certain transformations actually act like a map between different representations of a system, e.g. Fourier transform can take a differential representation in one domain to an algebraic representation in another domain.
	\item Discussed how systems can have properties (such as bounded input bounded output, casual, linear, etc.), and how we have been focused on linear systems with single input (hence can be represented by an linear ordinary differential equations), and how natural generalizations to nonlinear systems or multiple inputs lead to non-linear and/or partial differential equations.
\end{enumerate}
\section{Non-linear ordinary differential equations}
\begin{enumerate}
	\item Discussed exact equations, separable equations, Bernoulli equations, and solved examples.
\end{enumerate}
\section{Partial differential equations}
\label{sec: partial differential equations}
\begin{enumerate}
	\item Introduced \emph{tuples}, which are ordered lists:
	\be 
	(x_1,x_2,\dots,x_n)::T_1\x T_2\x \cdots \x T_n
	\ee 
	where $\x$ denotes a product, and $T_1\x T_2\x \cdots \x T_n$ is called a \emph{product type}. $\x$ creates a new product type from individual types; its inverse is \emph{projection} denoted by $\pi_i$ which works such that
	
	
	
	\bea 
	\pi_i ::{}&{} \left(T_1\x T_2\x \cdots \x T_n\right)\to T_i\\
	\pi_i={}&{}(x_1,x_2,\dots,x_n)\to x_i
	\eea
	
	What does it mean to take products of types? What do we really mean by, say, $\cM_{n\x n}(\C)\x \R$?
	\begin{itemize}
		\item In computer science, vast applications via so-called algebraic data types
		\item In logic, product of types are actually related to $\land$ (and) operation via Curry-Howard correspondence
		\item In abstract math, product of any family of objects is defined as \emph{``the most general thing''} from which individual objects can be extracted (category theory).
	\end{itemize}
	
	We are really used to taking products of types in Physics; for instance, Cartesian plane is $\R^2=\R\x\R$ is simply product of two real types. However, there is actually subtlety with such products in more general cases; for instance, the mathematical framework for quantum information theory is the so-called \texttt{Hilb} category which does not admit a Cartesian product (the only available products are non-cartesian), hence we do not actually have projection operations in this category. The well known physical result of this mathematical fact is that you cannot copy information in a quantum computer!
	
	\item If the elements of a tuple are of the same type, we use the easier notation $(x_1,\dots,x_n)::T^n$.
	
	\item Multi-variable functions are functions from tuples:
	\bea 
	f::{}&{} \R^3\to\R \\
	f={}&{} (x_1,x_2,x_3)\to f(x_1,x_2,x_3)
	\eea 
	
	\item Introduced D-notation:
	\bea 
	D::{}&{}\Z_+\to(\R^n\to\C)\to(\R^n\to\C)\\ 
	D={}&{}i\to\left((x_1,\dots,x_n)\to f(x_1,\dots,x_n)\right)\to \left((x_1,\dots,x_n)\to \frac{\partial f(x_1,\dots,x_n)}{\partial x_i}\right)
	\eea 
	e.g. $\displaystyle D_2f(x,y)=\frac{\partial f(x,y)}{\partial y}$
	
	\item Chain rule:
	\be 
	\rdr{}{x} f(g_1(x),\dots,g_k(x))=\sum\limits_{i=1}^k\left(\rdr{}{x}g_i(x)\right)D_if(g_1(x),\dots,g_k(x))
	\ee 
	
	\item Consider two functions $f$ and $g$:
	\be 
	f::{}&{}(X\x Y)\to Z\\
	g::{}&{} X\to (Y\to Z)
	\ee 
	We can \emph{choose} $g$ such that $g(x)(y)=f(x,y)$. This is called \emph{currying}, i.e. we curried $f$ into this new form as $g$. The reverse (rewriting a function like $g$ as a function like $f$) is called \emph{uncurrying}.
	
	\item Discussed several important examples of partial differential equations:
	\bea
	D_1 f(t,x)-\a^2 D_2^2 f(t,x)={}&{}0\qquad(\text{heat conductance equation})\\
	D_1^2 f(t,x)-c^2 D_2^2 f(t,x)={}&{}0\qquad(\text{wave equation})\\
	D_1^2 f(t,x)+ D_2^2 f(t,x)={}&{}0\qquad(\text{Laplace equation})
	\eea 
	
	\item We solved these equations both via method of seperation of variables, and also proposing solutions of the form $f(x,t)=g(x+a t)$.
	\item We discussed how partial differential equations can have undetermined functions in the answer just like how ordinary differential equations have undetermined coefficients in the answer!
\end{enumerate}
\part{Vector Calculus for Physicists}
\chapter{Definition and fundamentals of Vector Spaces}
\section{Preliminaries: some basic terminology}
\subsection{Primer: groups, fields, and algebras}
The vector spaces are some structures build on fields, which are themselves built on groups. The groups are built on sets. So we start with sets.

Our practical reason to start with sets is that we need to properly understand them to understand vectors, but one can even say that set theory is the basis of all mathematics! Indeed, historically, 20th century Mathematicians did try to build the foundations of math on sets, but Russell's paradox squeezed in and ruined the whole program, at least to the degree it was envisioned back then. Today, even more fundamental approaches (such as category theory or type theory) try to explain whole of math, but this is beyond our scope: sets are good enough to be the foundation of everything we will care in this book, so we will stick with them to prepare the background for vector spaces.

\paragraph{Sets:} A set is a collection of objects, and its elements determine what the set is. Importantly, we know all of the elements of a set!\footnote{This is actually a rather important point for deep mathematical reasons that we will not visit, so what follows in this footnote is beyond the scope of this book (rather disappointing for me).\\\\ The fact that we cannot know everything in a quantum system suggests that we should probably not use sets for quantum information, and this is actually true! The right mathematical object to use in quantum system does not have the type \texttt{Set} but has the type \texttt{Hilb}, or more explicitly, quantum systems belong to a different \emph{category} then the category of sets! An explicit discussion of these would require us to cover category theory, but an interested reader can consult here \draftnote{add sources}.}

Sets can be denoted as explicit list of elements within curly brackets;\footnote{Note that sets are orderless\footnotemark and that duplication of elements do not change a set, e.g.
\bea 
A,B,C::{}&\texttt{Set}\\
A={}&\{1,2,3,4\}\\
B={}&\{1,4,3,2\}\\
C={}&\{1,2,3,4,3,2,1\}
\eea 
implies $A=B=C$.
}
\footnotetext{
Ordered sets are denoted with \emph{round} brackets, e.g.
\bea 
A::{}&\texttt{Ordered Set}\\
A={}&(1,2,3)
\eea 
where $(1,2,3)\ne(1,3,2)$. Ordered sets can be constructed via usual sets, e.g.
\be 
(1,2,3)\leftrightarrow\{\{1\},\{1,2\},\{1,2,3\}\}\\
(1,3,2)\leftrightarrow\{\{1\},\{1,3\},\{1,2,3\}\}\\
\ee 
} for instance
\bea 
A::{}&\texttt{Set}\label{eq: set example type}\\
A={}&\{1,2,3\}\label{eq: set example}
\eea 
Equation~\eqref{eq: set example} shows that the set $A$ consists of the elements $1$, $2$, and $3$. Equation~\eqref{eq: set example type} on the other hand tells us that \emph{$A$ is of type Set}, i.e. $A$ is a set: for more details about this notation, please check out Section~\ref{sec: type notation and functions}.

Sets can have infinitely many elements and we can use dots to infer the rest from the given; for instance, the set $\{\dots,-1,1,3,5,\dots\}$ can denote the set of all odd integers. Although rather convenient, this method is inherently ambiguous; a more proper way to denote sets of infinite elements is through \emph{set comprehensions}.

A set comprehension is a way to build a set, and it is usually used with \emph{predicates}.\footnote{Predicates are functions with the codomain \texttt{Boolean}.\footnotemark In other words, a function is a predicate if it only yields two outputs: \textsf{True}, or \textsf{False}. For instance, the function ``isOddInteger'' defined as
\bea 
\text{isOddInteger}::{}&\texttt{Boolean}\\
\text{isOddInteger}={}&x\to \left(\frac{x+1}{2}\in\Z\right)
\eea 
is a \emph{predicate}, e.g. \mbox{$\text{isOddInteger}(1)=$\textsf{True}}, \mbox{$\text{isOddInteger}(2)=$\textsf{False}}, \mbox{$\text{isOddInteger}(5/2)=$\textsf{False}}, etc.
}
\footnotetext{This is the definition due to \emph{Gottlob Frege}, the father of axiomatic first-order logic. You may find other definitions in different branches of math: an interested reader can check Wikipedia for more.} There are two conventional ways to write down a set $A$ this way:
\begin{itemize}
	\item $A=\{x\in\text{ Domain}\;|\;\text{conditions}\}$
	\item $A=\{f(x)\;|\;\text{conditions}\}$ which is just an abbreviated version of the method above in the sense that this form actually means
\end{itemize}
\be 
A={}&{}\{y\in\text{ Domain inferred from the conditions}\\&\quad|\;\exists x\text{ s.t. }y=f(x)\text{ \& }\text{conditions}\}
\ee 
For instance, the rigorous way to denote the odd integers, compared to a convenient casual notation $\{\dots,-3,-1,1,3,\dots\}$, is\linebreak $\{a\in\Z\;|\;(\exists n\in\Z)\;a=2n+1\}$ by the first approach.\footnote{This would read as \emph{``the set of integer $a$'s for which there exists an integer $n$ such that $a=2n+1$ is true''}.\footnotemark}\footnotetext{
We can understand this more deeply as follows. Consider the \emph{predicate} $P$ defined as 
\be 
P::{}&{}\Z\to\texttt{Boolean}\\
P={}&{}a\to (\exists n\in\Z)\;a=2n+1
\ee 
The set of odd integers are then constructed by using this predicate with the list comprehension, i.e.
\be 
\text{odd integers}=\{a\in\Z\;|\; P(a)\}
\ee 
This also shows how the list comprehension works: $S=\{T\;|\; C\}$ means that the set $S$ consists of all elements of $T$ for which the condition $C$ is true.
}${}^{\;}$\footnote{We remind the reader that $\exists$ reads as \emph{``there exists$\dots$''}, and it is one of the \emph{quantifiers} in logic. The other two important ones that we will make use of are $\forall$ and $\exists !$, which denote \emph{``for all$\dots$''} and \emph{``there exists a unique$\dots$''} respectively. For instance, $(\exists!\;n\in\Z)\;n^2=n$ is the statement \emph{``there exists a unique integer $n$ such that $n^2=n$''}, and it is \textsf{False} as both $0$ and $1$ satisfy $n^2=n$: $\boxed{\left((\exists!\;n\in\Z)\;n^2=n\right)=\textsf{False}}$.\footnotemark\\\\
It is no coincidence that I put the dots to the right of the expressions for the quantifiers; one should read them as in a sentence, hence their order matters. For instance
\bea 
\Big((\forall a\in\Z)(\exists b\in\Z)\;b=2a\Big)=\textsf{True}\\
\Big((\exists b\in\Z)(\forall a\in\Z)\;b=2a\Big)=\textsf{False}
\eea 
where the first expression is \emph{``For all integers, there exists an integer which is twice the former''} and the second one is \emph{``There exists an integer which is twice all integers''}.
}\footnotetext{
It may seem weird at first that there is a variable in the left hand side of the equality $\left((\exists!\;n\in\Z)\;n^2=n\right)=\textsf{False}$, which does not appear on the right hand side. The variable $n$ here is a \emph{dummy/scooping/bound} variable hence the right hand side should be independent of it, just like the function $g(y)$ in $\int f(x,y)dx=g(y)$ being independent of $x$. For more details about this, see the comprehensive footnote~\ref{footnote:dummy variables}.} In comparison, the second approach makes it simpler: $\{2n+1\;|\; n\in\Z\}$.

I will assume that the readers are familiar with the comparisons of sets\footnote{$A\subseteq B$ (or $B\supseteq A$) means that all elements of $A$ are also elements of $B$ ($A$ is called a \emph{subset} of $B$), and $A\subset B$ (or $B\supset A$) means the same thing with the additional condition that $A\ne B$ ($A$ is called a \emph{proper subset} of $B$).} and miscellaneous details of sets.\footnote{For instance, $\abs{S}$ denotes the number of elements in the set $S$, e.g. $\abs{\{1,2,4\}}=3$. If the set $S$ is infinite, we write $\abs{S}=\infty$. As expected, $S\subseteq T$ implies $\abs{S}\le\abs{T}$. I would also like to introduce \emph{power sets}: the power set of a set $S$ is denoted by $\mathcal{P}(S)$ and it is the set whose elements are the subsets of the set $S$. For example,
\bea
S,\mathcal{P}(S)::{}&{}\texttt{Set}\\
S={}&{}\{a,b,1\}\\
\mathcal{P}(S)={}&{}\Big\{\{\},\{a\},\{b\},\{1\},\{a,b\}\\
{}&{}\{a,1\},\{b,1\},\{a,b,1\}\Big\}
\eea 
We note that if the set $S$ is finite, we have $\abs{\cP(S)}=2^{\abs{S}}$, which is indeed the case in the above example ($8=2^3$). This relation is why the power set of $S$ is also loosely denoted as $2^S$.
} I will also skip discussing the details of operations between sets.\footnote{Six important operations to know are \emph{union}, \emph{intersection}, \emph{difference} (also called \emph{complement}), \emph{Cartesian product}, \emph{disjoint union}, and \emph{quotient by an equivalence relation}. I expect the reader to know the first four operations: the last two will not be relevant for this book, but I advice the interested reader to learn more about them.} Lastly, I would like to mention the relevant concepts such as \emph{relations}, \emph{partitions}, and \emph{quotients}: one should know about them for a good general mathematical background, but they will unfortunately not be covered in this book.

\paragraph{Groups:} Consider a set $S$ and impose the existence of two functions $o$ and $i$ such that 
\bea 
S::{}&{}\texttt{Set}\\
o::{}&{}(S,S)\to S\\
i::{}&{}S\to S
\eea 
where $S$ is of type set, and the function $o$ has the domain ``tuple of $S$'' and the codomain $S$ ---check out section~\ref{sec: partial differential equations} for more details about tuples. In simpler terms, the function $o$ takes two inputs (both being elements of the set $S$) and yields an element of the set $S$; in comparison, the function $i$ takes an element of $S$ to another (not necessarily different) element of $S$.

The set $S$ with these two functions $o$ and $i$ form a group if the following statements are true:
\begin{enumerate}
	\item $\boxed{(\exists e\in S)(\forall s\in S)\; o(e,s)=o(s,e)=s}$. In words, this says that\linebreak \emph{``There exists an element $e$ of the set $S$ and $o(e,s)=o(s,e)=e$ for any element $s$ of this set''}. The element $e$ is called \emph{the identity element} with respect to the \emph{group operation} $o$.\footnote{Suppose for a second that there are two identity elements: $e_1,e_2$. Since $e_1$ is identity element, we need $o(e_1,e_2)=e_1$; but the same argument for $e_2$ dictates $o(e_1,e_2)=e_2$: we need $e_1=e_2$. In conclusion, if identity element exists, it is \emph{unique}!}
	\item  $\boxed{(\forall s \in S)\; o(s,i(s))=o(i(s),s)=e}$. In words, this says that \emph{``The group operation $o$ acting on the tuple $(s,i(s))$ or $(i(s),s)$ yields the identity element $e$, for any element $s$ of the set $S$''}. In practice, this condition specifies the action of $i$ on any element $s$, and $i(s)$ is called the \emph{inverse} of the element $s$.
	\item $\boxed{(\forall a,b,c \in S)\; o(a,o(b,c))=o(o(a,b),c)}$. This is called \emph{associative law}, and it is best visualized when we use \emph{infix notation}: if we denote $o(a,b)$ as $a*b$, then this condition simply reads as $a*(b*c)=(a*b)*c$. Therefore, the parentheses are not important, and we can even loosely denote without parentheses: $a*(b*c)=(a*b)*c=a*b*c$.
\end{enumerate}

We will \emph{define} a group to be the pair $(S,o)$, i.e.
\be 
\texttt{Group}=\Big(\texttt{Set},\;(\texttt{Set},\texttt{Set})\to\texttt{Set}\Big)
\ee 
such that the above statements are true (satisfaction of those rules will in turn determine the inverse function $i::S\to S$ as well). For instance, the pair $(\Z,+)$ is a group, where $+$ denotes the arithmetic addition, and the inverse function turns out to be multiplication by $-1$: we say that \emph{integers form a group under arithmetic addition}. We can indeed check that the above statements are true:
\begin{enumerate}
	\item $(\forall a\in \Z)\; a+0=0+a=a$ (addition by $0$ does not change any integer)
	\item $(\forall a\in \Z)\; a+(-a)=(-a)+a=0$ (addition by the inverse element takes an integer to zero)
	\item $(\forall a,b,c \in \Z)\; a+(b+c)=(a+b)+c$ (the order of addition does not change the result)
\end{enumerate}

As another example for groups, consider the set\linebreak$\textsf{classrooms}=\{P1,P2,P3\}$ and the group operation denoted as $<>$ in infix notation. If we are given the information
\be 
\label{ex: classrooms}
P1<>P1={}&{}P3\\
P2<>P2={}&{}P2\\
P3<>P3={}&{}P1\\
P1<>P2=P2<>P1={}&{}P1\\
P1<>P3=P3<>P1={}&{}P2\\
P2<>P3=P3<>P2={}&{}P3
\ee 
we can immediately deduce that $P2$ is the identity element, and $P1^{-1}=P3$, $P2^{-1}=P2$, \& $P3^{-1}=P1$ (inverse operations). One can check that the necessary conditions are satisfied, hence we state
\be 
(\textsf{classrooms},<>)::\texttt{Group}
\ee 
We will leave further details of groups to the interested reader to research! One important exception will be the concept of \emph{commutativity}: if the group operation is commutative, i.e. $a*b=b*a$, then that group is called a \emph{commutative group}. In commutative groups, one conventionally use the infix operator $+$ instead of $*$ for the group operation, $-s$ instead of $s^{-1}$ to denote inverse operation, and $0$ instead of $1$ to denote identity element. For instance, we can rewrite the example in \equref{ex: classrooms} as $\textsf{classrooms}=\{P1,P2,P3\}=\{P1,0,-P1\}$. Note that the minus ($-$) sign and the symbol $(0)$ has nothing to do with the arithmetic negation and the number $0$ in general: this is simply a conventional notation.

\paragraph{Rings:} Consider a set $S$ and two functions $+$ and $\.$ such that 
\bea 
S::{}&{}\texttt{Set}\\
+::{}&{}(S,S)\to S\\
\.::{}&{}(S,S)\to S
\eea 
where we will denote the functions in the infix form, e.g. $a+b$ instead of $+(a,b)$. We say that the triplet $(S,+,\.)$ is a ring, i.e.
\be 
(S,+,\.)::\texttt{Ring}
\ee
if the following statements are true:
\begin{enumerate}
	\item $\boxed{(S,+)\text{ is a commutative group}}$
	\item $\boxed{(\forall a,b,c\in S)\; a\.(b+c)=a\.b+a\.c}$
	\item $\boxed{(\forall a,b,c\in S)\; (b+c)\.a=b\.a+c\.a}$
\end{enumerate}
In other words, rings are simply commutative groups with an additional operation (call multiplication) that distributes over group addition.

The most familiar rings are those where the ring addition and multiplication operations are literally arithmetic addition and multiplication; for instance, $(\Z,+,\.)$ forms a ring and so does $(\R,+,\.)$. Likewise, rectangular matrices with complex entries form a ring under matrix addition and matrix multiplication, i.e. $(\cM_{n\x n}(\C),+,\.)::\texttt{Ring}$.

\paragraph{Skew field:} We say that a triplet $(S,+,\.)$ is a skew field, i.e.
\be 
(S,+,\.)::\texttt{Skew Field}
\ee
if the following statements are true:
\begin{enumerate}
	\item $\boxed{(S,+,\.)\text{ is a ring}}$
	\item $\boxed{(S\backslash\{0\},\.)\text{ is a group}}$
\end{enumerate}
In other words, skew fields are rings where multiplication operation also forms a group if we remove the identity element of the addition operation. The prototypical example of the skew fields is the \emph{quaternions}.\footnote{\draftnote{Put some discussion of quaternions}.}

\paragraph{Fields:}  We say that a triplet $(S,+,\.)$ is a field, i.e.
\be 
(S,+,\.)::\texttt{Field}
\ee
if the following statements are true:
\begin{enumerate}
	\item $\boxed{(S,+,\.)\text{ is a ring}}$
	\item $\boxed{(S\backslash\{0\},\.)\text{ is a commutative group}}$
\end{enumerate}

Fields are ubiquitous and super important for physicists as we use them everwhere! The rational, real, and complex numbers all form fields under ordinary arithmetic addition and multiplication:\footnote{\label{footnote: exotic example for a field}
For a more exotic example, consider the following set:
\be 
A=\Big\{(x\to x+a)\;\Big|\; a\in \R\Big\}
\ee 
for the ordinary arithmetic addition. This is a set of \emph{functions}, i.e. elements of the set are functions; for instance, one of the elements of this set is $x\to x+1$: this is a function that adds one to its input.
\\\\
Let us rewrite this set in a more familiar way. For this, we define the \emph{higher order function} ``\textsf{plus}'':
\be 
\textsf{plus}::{}&{}\R\to(\R\to\R)\\
\textsf{plus}={}&{}a\to\big(x\to(x+a)\big)
\ee 
This higher order function becomes an ordinary function if given an input, i.e. $\textsf{plus}(1)::\R\to\R$, which can then be given one more input that converts it into a real number: $\textsf{plus}(1)(5)=6::\R$. Indeed, this higher order function simply adds its inputs, i.e. $\textsf{plus}(a)(b)=a+b$. In terms of this higher order function, we can write down the set $A$ above as
\be 
A=\Big\{\textsf{plus}(a)\;\Big|\; a\in \R\Big\}
\ee 
Let us now introduce two higher order functions:
\be
\oplus,\otimes ::{}\big((\R\to\R),(\R\to\R)\big)\to (\R\to\R)
\ee
These functions take a pair of functions, and then give another function! Let us use these guys in infix form, e.g. $\textsf{plus}(a) \oplus\textsf{plus}(b)$.
\\\\
If we are given the information that 
\be 
\textsf{plus}(a) \oplus\textsf{plus}(b)=\textsf{plus}(a+b)
\ee 
we can immediately say $(A,\oplus)$ forms a commutative group, with the identity element $\textsf{plus}(0)$. If we are given the further information
\bea 
{}&\big(\textsf{plus}(a) \oplus\textsf{plus}(b)\big) \otimes\textsf{plus}(c)=\nn\\
{}&\big(\textsf{plus}(a)\otimes\textsf{plus}(c)\big) \oplus\big(\textsf{plus}(b)\otimes\textsf{plus}(c)\big)
\\
{}& \textsf{plus}(c)\otimes\big(\textsf{plus}(a) \oplus\textsf{plus}(b)\big)=\nn\\
{}&\big(\textsf{plus}(c)\otimes\textsf{plus}(a)\big) \oplus\big(\textsf{plus}(c)\otimes\textsf{plus}(b)\big)
\eea 
we can then say that $(A,\oplus,\otimes)$ forms a ring! Finally, if we are also given
\be 
\textsf{plus}(a) \otimes\textsf{plus}(b)=\textsf{plus}(a.b)
\ee 
then we can prove that $(A,\oplus,\otimes)$ is actually a field!
}
\be 
(\Q,+,\.)::\texttt{Field}\;,\qquad (\R,+,\.)::\texttt{Field}\;,\qquad (\C,+,\.)::\texttt{Field}
\ee  

\paragraph{Vector space (Linear space):} We \emph{define} \emph{``a vector space over a field $F$''} as the triple $(V,\oplus,\odot)$, i.e.
\bea
V::{}&{}\texttt{Set}\\
\Big(F=(S,+,\.)\Big)::{}&{}\texttt{Field}\\
\oplus::{}&{}(V,V)\to V\\
\odot::{}&{}(S,V)\to V\\
(V,\oplus,\odot)::{}&{}\texttt{Vector Space}
\eea
if the following statements are true
\begin{enumerate}
	\item $\boxed{(V,\oplus)::\text{Commutative Group}}$
	\item $\boxed{(\forall v\in V)\; 1\odot v=v\text{ where $1$ is the identity element of $\.$}}$
	\item $\boxed{(\forall v\in V)(\forall s\in S)\; s\odot v \in V}$
	\item $\boxed{(\forall v\in V)(\forall a,b\in S)\; (a\.b)\odot v=a\odot (b\odot v)}$
	\item $\boxed{(\forall v\in V)(\forall a,b\in S)\; (a+b)\odot v=(a\odot v)\oplus (b \odot v)}$
	\item $\boxed{(\forall v,w\in V)(\forall s\in S)\; s\odot(v\oplus w)=(s\odot v)\oplus (s\odot w)}$
\end{enumerate}
Here, the elements of the underlying field $F$ are called \emph{scalars} ($s\in S$) and the elements of the linear spaces are called \emph{vectors} ($v\in V$). indeed, $V$ above is the set of these vectors, $\oplus$ is the \emph{vector addition} operation, and $\odot$ is the \emph{scalar multiplication} operation. We will see more properties of vectors (such as basis, linear independence, etc.) in the next section.

The standard example of a linear space is this:
\be
\big(
\left(\cM_{n\x m}(S),\oplus\right),\left(S,+,\.\right)
\big)::\text{Linear Space}
\ee 
In words, we say that all $m\x n$ matrices from a field $\ka=\left(S,+,\.\right)$ form a linear space over $\ka$. In this example, matrices \emph{are} vectors of this linear space and the vector addition $\oplus$ is simply defined as the matrix addition. A more familiar example can be given as follows:
\be 
V={}&{}\{a\bx+b\by+c\bz \;|\; a,b,c\in\R\}\\
\left(a_1\bx+b_1\by+c_1\bz\right)\oplus\left(a_1\bx+b_1\by+c_1\bz\right)={}&{}(a_1+a_2)\bx+(b_1+b_2)\by+(c_1+c_2)\bz\\
s\odot\left(a\bx+b\by+c\bz\right)={}&{}(s\.a)\bx+(s\.b)\by+(s\.c)\bz
\ee 
where we defined the set $V$ and the explicit action of the functions $\oplus$ and $\odot$.\footnote{
Let us consider a more nontrivial example. Consider the field $(A,\oplus,\otimes)$ disucssed in footnote~\ref{footnote: exotic example for a field}. We will define a vector space $(W,\square,\odot)$ over this field. Note that, in this example, $\square$ will denote the vector addition whereas $\oplus$ is reserved for the scalar addition in the field $A$.
\\\\
We define the set $W$ as
\be 
W=\{(a,b)\;|\;a,b\in\R\}
\ee  
whereas we choose the vector addition and scalar multiplications as
\be 
(a,b)\square (c,d)=(a+c,\;b+d)
\ee 
and
\be 
\textsf{plus}(a)\odot (b,c)=(a\.b,\;a\.c)
\ee 
With these definitions, we can verify that all conditions for $(W,\square,\odot)$ to be a vector space over the field $A$ are satisfied.
}
\draftnote{
\begin{enumerate}
	\item (Linear algebra$=$linear space with vector multiplication) is introduced and discussed.
	\item (Lie algebras$=$linear algebra with multiplication being antisymmetric) is introduced and discussed.
	\item Made reference to the commutators and discussed how a linear algebra can be turned into a lie algebra via commutators.
	\item Cross products in three dimensional vector space is discussed from this perspective: $3d$ vectors form a lie algebra thanks to the cross product.
	\item Emphasized how vector spaces can be built on an arbitrary set $V$ as long as the necessary conditions are satisfied. For instance, we can choose
	\be 
	V=\Big\{a\pdr{}{x}+b\pdr{}{y}+c\pdr{}{z}\;\Big|\;a,b,c\in\R\Big\}
	\ee 
	With arithmatic addition and multiplication, these indeed form a vector space. In fact, this way of definition is far more useful than $a\bi+b\bj+c\bk$ as we will see later.
\end{enumerate}
}

\subsection{Properties of Linear spaces}
\draftnote{
\begin{enumerate}
	\item Basis for vector spaces is introduced: let $(B\supset V)::\texttt{Set}$. $B$ is called \emph{a basis} of $V$ if following statements are true:
\begin{itemize}
	\item $(\forall k\in\{1,2,\dim B\})(\forall e_1,\dots,e_k\in B)(\forall c_1,\dots,c_k\in S) [c_1=\dots=c_k=0]\lor[c_1e_1+\dots+c_ke_k\ne 0]$
	\item $(\forall v\in V)(\exists ! a_1,\dots,a_{\dim B}\in S) v=a_1e_1+\dots a_{\dim B}e_{\dim B}$
\end{itemize}
The first statement is the \emph{linear independence} of the basis vectors, whereas second statement is the requirement that the basis \emph{spans} the vector space, i.e. any vector can be decomposed in terms of the basis elements. The coefficients $a_i$ are called the \emph{components} of the vector in this basis.
\item If $\dim B<\infty$, it is called the dimension of the vector space!
\item If $\dim B=\infty$, the vector space is called \emph{infinite dimensional}!
\item Using a basis, a vector can be related to a list/array/tuple of scalars:
\be 
\left.\begin{aligned}
	v::&\texttt{Vector}\\
	a_1,\dots,a_n::&\texttt{Scalar}
\end{aligned}\right\}\Rightarrow v=\begin{pmatrix}
	a_1&\dots&a_n
\end{pmatrix}\begin{pmatrix}
	e_1\\\dots\\e_n
\end{pmatrix}
\ee 
For instance, $(3\bi+4\bk)::\texttt{Vector}$ whereas $(3,0,4)::(\texttt{Scalar},\texttt{Scalar},\texttt{Scalar})$; nevertheless, they represent the same information once the proper basis is given: $3\bi+4\bk=\begin{pmatrix}
	3&0&4
\end{pmatrix}\begin{pmatrix}
	\bi\\\bj\\bk
\end{pmatrix}$.
\item If multiple basis exist for a vector space, all have the same dimension and all can be converted to one another (proof omitted).
\item Index notation and Einstein's summation convention is discussed, i.e. $v=v^ie_i$. Note that index placement is important, i.e. $e^i\ne e_i$.
\item One can choose $a_i$ to represent a row or column matrix (and $a^i$ the other one). Note that $M_i^{\;j}$ or $M^i_{\; j}$ represent a square matrix (as they convert a row/column matrix to another row/column matrix), but $M_{ij}$ or $M^{ij}$ are not square matrices.
\end{enumerate}
}

\section{Inner product spaces}
\draftnote{
\begin{enumerate}
\item Normed vector spaces: in general, we do not need to have functions that takes vectors as inputs and spits out scalars as outputs, i.e. no $(\cdot)::(V,V)\to S$ or $\norm{.}::V\to S$ is necessary for the definition of the vector space.
\\\\
If we impose that a function \emph{\textsf{norm}} such that
\be 
\textsf{norm}::&V\to\R \\
\textsf{norm}=& x\to\norm{x}
\ee 
exists where
\begin{itemize}
	\item $(\forall v\in V) [\norm{v}\ne 0]\lor[v=0]$
	\item $(\forall v\in V)(\forall s\in S)\norm{s\odot v}=\abs{s}.\norm{v}$
	\item $(\forall v,w\in V)\norm{v\oplus w}\le\norm{v}+\norm{w}$
\end{itemize}
and where $V$ is a vector space over the field $\R$ or $\C$, then we call $(V,\norm{.})$ a \emph{normed vector space}.
\item With norm, we can define distance between two vectors: $d::(V,V)\to\R$ and $d=(x,y)\to\norm{x-y}$.
\item With norm, we can define angle between two vectors: $\theta::(V,V)\to\R$ and $\displaystyle \theta=(x,y)\to\acos(\frac{\norm{x}^2+\norm{y}^2-\norm{x-y}^2}{2\norm{x}\norm{y}})$.
\item Inner product spaces: If we have a vector space over the field $\R$ or $\C$, and if we impose the existence of the function $\<.,\.\>$ for 
\be 
F\in&\{\R,\C\}\\
V::&\texttt{Vector Space over }F\\
\<.,.\>::&(V,V)\to F
\ee 
where the following are true:
\begin{itemize}
	\item $(\forall v,w\in V)\;\<v,w\>=\<w,v\>^*$ (conjugate symmetry)
	\item $(\forall u,v,w\in V)(\forall a,b\in F)\; \<au+bv,w\>=a\<u,w\>+b\<v,w\>$
	\item $(\forall v\in V\backslash\{0\})\;\<v,v\>>0$
	\item $\<0,0\>=0$
\end{itemize}
\item Inner product vector spaces can immediately be turned to a normed vector space as well as we can  define $\norm{v}=\sqrt{\<v,v\>}$ which is consistent with all norm requirements. In this definition, the angle between two vectors become $\displaystyle\cos\theta=\frac{\Re(\<x,y\>)}{\norm{x}\norm{y}}$.
\end{enumerate}
}
\section{Dual vector spaces}
\draftnote{
\begin{enumerate}
\item Dual vector spaces: Let $V$ be a vector space over a field $F$. If we consider linear functions that take a vector to a scalar, those functions themselves form a vector space: that is called the \emph{dual vector space} $V^*$. So we have 
\bea 
(F=(S,+,.))::&\texttt{Field}\\
V::&\texttt{Set}\\
\oplus::&(V,V)\to V\\
\odot::&(S,V)\to V\\
(V^*=\{V\to S\})::&\texttt{Set}\\
\oplus^*::&(V^*,V^*)\to V^*\\
\odot^*::&(S^*,V^*)\to V^*\\
\eea 
where 
\bea 
(V,\oplus,\odot)=\text{ vector space over }F\\
(V^*,\oplus^*,\odot^*)=\text{ vector space over }F
\eea 
Gave several examples in class, for instance $V=\{a\bi+b\bj\;|\;a,b\in\R\}$ and $V^*=\{(x\bi+y\bj)\to(ax+by)\;|\;a,b\in\R\}$.
\item Conventionally, elements of $V$ are called \emph{vectors} and elements of $V^*$ are called \emph{covectors}.
\item For the vector space given in the notation
\be 
\label{eq: vector derivative notation}
V=\Big\{a\pdr{}{x}+b\pdr{}{y}+c\pdr{}{z}\;\Big|\;a,b,c,\in\R\Big\}
\ee 
the dual vector space is denoted as
\be 
V^*=\big\{adx+bdy+cdz\;\big|\;a,b,c\in\R\}
\ee 
where $dx\left(\pdr{}{x}\right)=1$, $dx\left(\pdr{}{y}\right)=0$, and similarly.

Note that this convention is more than just a notation: the vector spaces indeed transform functions to functions, and integrations are indeed summations of covectors in a loose sense. We will make these much more concrete later.
\item Discussed in detail and derived that if a basis of the vector space $e_i$ and the basis of the dual vector space $e^i$ transform as 
\be 
e_i\to& M_i^{\;k}e_k\\
e^i\to& e^kN_k^{\;\;i}
\ee 
then the components of the basis vector $v^i$ and the components of the covector $w_i$ transform as 
\be 
v^i\to& v^k(M^{-1})_k^{\;\;i}\\
w_i\to& (N^{-1})_i^{\;k}w_k
\ee 
If we want the action of the covector basis on vector basis to remain unchanged, i.e.
\be 
e^i(e_k)\to e^i(e_k)
\ee 
then we need $N=M^{-1}$. With this, we see that
\be 
e_i\to& M_i^{\;k}e_k\\
w_i\to& M_i^{\;k}w_k\\
e^i\to& e^k(M^{-1})_k^{\;\;i}\\
v^i\to& v^k(M^{-1})_k^{\;\;i}
\ee
We see that the array $w_i$ transforms in the same way as the basis vectors of $V$, i.e. it transforms \emph{covariantly}. We then loosely state that $w_i$ is a \emph{covariant vector}!\footnote{Note that $v^i$ ($w_i$) is actually only the components of the (co)vector, hence it is only an array of the underlying field , i.e. $w_i::(S,S,\dots,S)$ and $v^i::(S,S,\dots,S)$. In contrast, the combination $(w_ie^i)::V^*$ is a covector and the combination $(v^ie_i)::V$ is a vector.} In contrast, $v^i$ transforms in an inverse way compared to $e_i$, and hence it is loosely called a \emph{contravariant vector}!
\end{enumerate}
}
\draftnote{We did not discuss topological vector spaces and Hilbert spaces this year: we may discuss it next year!}

\section{Algebras over fields}
\subsection{Tensor algebras}
\draftnote{
\begin{enumerate}
	\item Consider a linear space over a field $F$. We call \emph{``type $(r,s)$ tensor on a vector space $V$''} an element of 
	\be 
	\underbrace{V\otimes V\otimes\cdot\otimes V}_{r}\otimes\underbrace{V^*\otimes V^*\otimes\cdot\otimes V^*}_{s}
	\ee 
	where $\otimes$ is an associative bilinear map, i.e.
	\be 
	&A=\{V,V^*\}\\
	&\otimes::(A,A)\to A\otimes A\\
	&(\forall a,b,c\in A)\;(a\otimes b)\otimes c=a\otimes(b\otimes c)
	\ee
\item We conventionally denote
\be 
T^kV=\underbrace{V\otimes\cdots\otimes V}_{k}
\ee 
Since the bilinear map $\otimes$ takes the product of $v:: T^kV$ and $\w:: T^l V$ to $(v\otimes\w)::T^{k+l}V$, we need the consider the space of all possible $T^kV$ to have an algebra: this is denoted as $T(V)$ (tensor algebra over $V$) and is defined as
\be 
T(V)=\bigoplus\limits_{k=0}^\infty T^kV=F\oplus V\oplus(V\otimes V)\oplus\cdots
\ee 
A similar definition can be made for $T(V^*)$; the product $T(V)\otimes T(V^*)$ is the algebra over vector field that includes all type of tensors (i.e. $(\forall r,s\in\N)\;(r,s)-$tensor).
\item In Physics, we usually work with components. Indeed, consider the following example:
\be 
a::&V\otimes V\otimes V^*\\
a=&a^{ij}_{\;\;k}e_i\otimes e_j\otimes e^k
\ee 
We usually call $a^{ij}_{\;\;k}$ a tensor: in reality, it is only a collection of scalars (underlying field elements of the vector space); nevertheless, once a basis is specified, there is indeed a unique map between $a$ and $a^{ij}_{\;\;k}$ hence this abuse of notation is justified.
\item $\otimes$ provides an \emph{outer} product, i.e. $a\otimes b$ is not part of the types of $a$ or $b$. For instance, if both $a$ and $b$ are of the type $(1,0)$, $a\otimes b$ is of the type $(2,0)$. In general the value $r+s$ for a type $(r,s)$ tensor is called \emph{the rank of the tensor}, and the rank of the tensor $a\otimes b$ is the sum of the ranks of the tensors $a$ and $b$. For instance:
\be 
\label{eqn: tensor example}
a::&V\otimes V\otimes V^*\\
b::&V^*\otimes V^*\\
(c=a\otimes b)::&V\otimes V\otimes V^*\otimes V^*\otimes V^*\\
a=&a^{ij}_{\;\;k}e_i\otimes e_j\otimes e^k\\
b=&b_{lm}e^l\otimes e^m\\
c=&c^{ij}_{\;\;klm}e_i\otimes e_j\otimes e^k\otimes e^l\otimes e^m\\
c^{ij}_{\;\;klm}=&a^{ij}_{\;\;k}b_{lm}
\ee 
\item As stated earlier, we usually work with components in practice (at least in the Physics), so the product of two tensors $a^{i_1\dots i_r}_{\quad\;\; k_1\dots k_s}$ and $b^{l_1\dots l_t}_{\quad\;\; m_1\dots m_u}$ just becomes $c^{i_1\dots i_r l_1\dots l_t}_{\qquad\quad k_1\dots k_s m_1\dots m_u}=a^{i_1\dots i_r}_{\quad\;\; k_1\dots k_s}b^{l_1\dots l_t}_{\quad\;\; m_1\dots m_u}$.
\item There is a way to change the type and rank of a tensor by an operation called \emph{contraction}. What it does is basically act the basis covector on the basis vector; to see this, consider the tensor $c$ in the equation~\ref{eqn: tensor example}. We can \emph{choose} to contract some of its basis elements to reduce its rank, for instance if we choose to act $e^k$ on $e_i$, we get $e^k(e_i)=\de^k_{\;i}$ for the kronecker-delta $\de$. But since $c^{ij}_{\;\;klm}\de^k_{\;i}=c^{ij}_{\;\;ilm}$, we arrive at
\be 
c'::&V\otimes V^*\otimes V^*\\
c'=&c^{ij}_{\;\;ilm} e_j\otimes e^l\otimes e^m
\ee 
If we instead acted $e^k$ on $e_j$, we would get a different tensor $c''$:
\be 
c''::&V\otimes V^*\otimes V^*\\
c''=&c^{ij}_{\;\;jlm} e_i\otimes e^l\otimes e^m
\ee
Or we could act $e^k$ on $e_i$ and $e^l$ on $e_j$: this would give us an entirely different tensor $c'''$:
\be 
c'''::& V^*\\
c'''=&c^{ij}_{\;\;ijm}e^m
\ee
\item Consider the rotational version of Newton's second law, i.e. \emph{``torque is equal to moment of inertial times angular acceleration''}. We conventionally write it as a matrix equation, i.e. $\tau^i=I^i_{\;j}\a^j$. We see that we could write them as follows:
\be 
\tau,\alpha::&V\\
I::&V\otimes V^*\\
(I\otimes \alpha)=& I^i_{\;j}\alpha^k e_i\otimes e_k\otimes e^j\\
\tau=&I^i_{\;j}\a^j e_i
\ee 
Clearly, we got $\tau$ by first taking the outer product of $I$ and $\a$, and then doing a contraction between two of the indices (equivalently, applying the covector $e^j$ on the vector $e_k$).\footnote{
This line of reasoning is valid to a degree for $3d-$vectors, but in higher dimensions, it is not true. Firstly, $\tau$ and $\a$ are not vectors but \emph{bivectors} in general dimension (we'll cover bivectors below), hence $I$ is actually not a matrix (a rank-2 tensor) but a linear map of bivectors to bivectors (hence a rank-4 tensor): $I::V\otimes V\otimes V^*\otimes V^*$.
}
\end{enumerate}
}
\subsection{Exterior algebras}
\draftnote{\begin{enumerate}
\item Tensor product $\otimes$ is only defined as an associative bilinear mapping. If we define a similar product with the additional property that it is antisymmetric, we get \emph{exterior algebra}.
\item The relevant antisymmetric operator is called \emph{Wedge product} and is denoted as $\wedge$:
\begin{itemize}
	\item $(\forall u,v\in V)\; u\wedge v=-v\wedge u$
	\item $(\forall u,v,w\in V)\; u\wedge(v\wedge w)=(u\wedge v)\wedge w$
\end{itemize}
\item As the product is fully antisymmetric, any wedge product of $k-$many vectors is zero if $k>\dim(V)$, thus:
\be 
\Lambda(V)=\bigoplus\limits_{k=0}^{\dim(V)} \Lambda^kV=F\oplus V\oplus(V\wedge V)\oplus\cdots
\ee 
for
\be 
\Lambda^kV=\underbrace{V\wedge\cdots\wedge V}_{k}
\ee 
forms the exterior algebra.
\item Elements of $\Lambda^kV$ are called \emph{k-vectors} (or multivectors in general).
\item In $3d$, the exterior product of two vectors would almost like the cross product of the vectors:
\be 
\left.\begin{aligned}
u=u_1e_1+u_2e_2+u_3e_3\\
v=v_1e_1+v_2e_2+v_3e_3
\end{aligned}\right\}\rightarrow u\wedge v=(u_1v_2-v_2u_1)(e_1\wedge e_2)+\dots
\ee 
Similarly, $u\wedge v\wedge w$ will be almost same as $\vec{u}\.(\vec{v}\x\vec{w})$. These are no coincidences; in fact, we will see that many vector operations can be understood in terms of multivectors.
\item We have seen several area and volume integrations in first year courses: we know that the integrations (such as use of Gauss law in electrodynamics) require \emph{oriented areas}; in practice, we write integrals like $\int f(x,y)dxdy$; however, obviously, the infinitesimal area element $dxdy$ is \emph{not} oriented when written like that. The proper way to write it is actually as $dx\wedge dy$: this is simply a \emph{2-covector}; likewise, the oriented volume element is a \emph{3-covector} $dx\wedge dy\wedge dz$.

\item Hodge duality: Any two vector space with same dimension are isomorphic to each other, i.e. they can be converted to one another. Since $\Lambda^p{V}$ is a vector space with dimensionality $\frac{d!}{p!(d-p)!}$,\footnote{This is the number to choose $p$ elements out of $d$, as the basis vectors of $\Lambda^p(V)$ are ordered combinations (e.g. $e_1\wedge e_2$ and $e_2\wedge e_1$ are not linearly independent).} there is an isomorphism between $p-$vectors and $(d-p)-$vectors: a natural way to achieve this is the \emph{Hodge duality}. If $v$ is a $p-$vector, then its Hodge dual (denoted as $\star v$) is a $(d-p)-$vector!

\item We physicists mostly work only with components (components of the vector, multivector, tensor, etc.) instead of the actual object, so for us it is sufficient to know how to relate the components of a $p-$vector and its Hodge dual.\footnote{
More abstractly, we can understand Hodge duality as follows.
\\\\
Let $V::\texttt{Inner Product Vector Space}$ have the inner product $\<\.,\.\>::(V,V)\to\C$. This inner product \emph{induces} and inner product in the vector space $(\star\a)_{i_{k+1}\dots i_d}$ as follows:
\be
\<\.,\.\>_k::(\Lambda^kV,\Lambda^k V)\to\C
\ee 
for 
\begin{multline}
\<\a_1\wedge\dots\wedge\a_k,\b_1\wedge\dots\wedge\b_k\>=\\\det\begin{pmatrix}
	\<\a_1,\b_1\> &\dots & \<\a_1,\b_k\>
	\\\vdots\\
	\<\a_k,\b_1\> &\dots & \<\a_k,\b_k\>
\end{pmatrix}
\end{multline} 
The Hodge-dual of a $k-$vector $\b$ is then defined in terms of this induced inner product as
\be 
(\forall \a\in \Lambda^k(V))\quad \a\wedge(\star\b)=\<\a,\b\>e_1\wedge\dots \wedge e_n
\ee 
where $e_i$ form the orthonormal basis of $V$.
} In the convention
\begin{subequations}
\label{eq: Hodge dual}
\be 
\a=\a_{i_1\dots i_k}e^{i_1}\wedge\cdots\wedge e^{i_k}\\
\star\a=(\star\a)_{i_{k+1}\dots i_d}e^{i_{k+1}}\wedge\cdots\wedge e^{i_d}
\ee 
the relation reads as
\be 
(\star\a)_{i_{k+1}\dots i_d}=\frac{1}{(d-k)!}\a_{i_1\dots i_k}\e^{i_1\dots i_kl_{k+1}\dots l_d}\de_{i_{k+1}l_{k+1}}\cdots \de_{i_dl_d}
\ee 
\end{subequations}
if we are working in Euclidean spaces, which will be the focus of this course! Here $\de$ is the Kronecker delta symbol.
\item We discussed examples of Hodge dualities in $2d$ and $3d$, e.g.
\be 
\star 1 =dx\wedge dy\;,\quad \star dx=dy\;,\quad \star dy=-dx\text{ in }2d
\ee 
\item We discussed how Hodge duality helps us define the cross product in $3d$: $\star \left(u\wedge v\right)=u\times v$.
\end{enumerate}}

\subsection{Division algebras}
\draftnote{
Following is not covered in class or in homeworks; may be covered next years:\\\\	
Normed division algebras and Hurwitz's theorem; finite-dimensional associative division algebras over the real numbers and Frobenius theorem; more about quaternions, their usage in spatial rotation (hence applications in engineering), and Feza Gursey's work with quaternions}

\chapter{Differentiation in Vector Calculus}
\section{Vector fields and forms on Euclidean Spaces}
\draftnote{
\begin{enumerate}
	\item In class, we discussed 2d and 3d pendulums: at different positions, the velocity at that position belongs to the vector space \emph{at that position}. Even though these vectors spaces are isomorphic to each other,\footnote{Any two finite dimensional vector spaces with the same dimension are actually isomorphic to each other.} i.e. they can be transformed to each other, they are actually different vector spaces! This is not only true for pendulums, but for any motion! 
	\item Let us assume that we are going to describe the kinematics of a particle. We need to specify the position and the velocity of the particle at that position: this is the complete information to understand the trajectory of the particle. In other words, if we know all possible $(r,\vec{v}(r))$ pairs where $r$ parametrizes the particle's position in the system and $\vec{v}(r)$ is the velocity vector in that position, then we can describe the motion completely!
	\\\\
	Consider a two dimensional pendulum. We can parametrize the object's position by an angle $\theta$ as the trajectory forms a circle. At any position, its velocity lies in the \emph{tangent line} at the pendulum's position to this circle: note that these lines are \emph{different} at different values of $\theta$, so we indeed have infinitely many vector spaces, one for each value of $\theta$. We could choose $\pdr{}{\theta}$ as the basis vector; then the relevant pairs would look like $\left(\frac{\pi}{6},2\pdr{}{\theta}\right)$ or $\left(-\frac{\pi}{3},-\pdr{}{\theta}\right)$. The first one says that at $\pi=\pi/6$, the velocity is $2\pdr{}{\theta}$ at the tangent space to the circle at $\theta=\pi/6$. We will see later that this basically means the particle is moving such that at $\theta=\pi/6$, it will move to increase its $\theta$. On the contrary, at $\theta=-\pi/3$, it will move to decrease its $\theta$ (as the velocity at that point is $-\pdr{}{\theta}$). If we are to combine all such $(\theta,v(\theta)\pdr{}{\theta})$ pairs, we have the complete behavior of the pendulum!
	\\\\
	To understand these better, we need to introduce some terminology.
	\item We call a space \emph{Manifold} if it resembles the Euclidean space near each point.\footnote{We actually need to know \emph{topological spaces} to define manifolds more appropriately; nevertheless, this definition is good enough for this course.} For instance, $\R^k$ is trivially a Manifold, it is Euclidean space itself. A more nontrivial example is a circle (denoted as $S^1$) or a sphere (denoted as $S^2$): near any point on a circle or a sphere, the space \emph{looks like} a Euclidean space!\footnote{This is why flat Earth believers are so confused: the surface of our planet is a Manifold, and at around any point, we see it approximately as a flat plane $\R^2$.}
	\item At any point on a Manifold $M$, we have the \emph{tangent space} to the manifold denoted as $T_xM$. For instance, $T_xS$ denotes the tangent like to the circle $S$ at the position $x$; likewise, $T_xS^2$ denotes the tangent plane to the sphere $S^2$ at the position $x$. Tangent spaces are necessarily flat, i.e. they are simply $\R^d$ where $d$ is the dimension of the Manifold.
	\item Clearly, $T_xM$ and $M$ are different spaces: this is most easy when the Manifold is not $\R^d$. For instance, if we consider a pendulum, the position of the object is an element of the Manifold $S$, and its velocity at any point $x$ is an element of the tangent space $T_xS$. Clearly, the tangent space (simply the tangent line $\R$ in this case) is a linear space, hence its elements are vectors (called tangent vector), which makes sense as we expect the velocity to be a vector! 
	\\\\
	In other examples, it might look tricky. For instance, for a car moving on a plane, the manifold is $\R^2$, and the tangent space at any point is also $\R^2$: even though these two spaces are \emph{isomorphic}, they are actually different: position is an element of the manifold $\R^2$, whereas the velocity at any position is an element of the tangent space $\R^2$ at that position.
	
	\item Consider a manifold $M$ and an element $x\in M$. Clearly we have infinitely many tangent spaces $T_xM$ for all $x\in M$. We can then define so-called \emph{tangent bundle} of this manifold as the collection of all such vector spaces! Tangent bundle of the manifold $M$ is denoted as $TM$ and we will define it\footnote{\label{footnote: tangent bundle}A more approriate definition of tangent bundle would involve \emph{disjoint union} of individual tangent spaces. As there is no unique way to define disjoint unions (they are only defined up to isomorphisms), we can in fact define $TM$ in different ways as well. We will not be using the explicit definition of $TM$ in this course, so our definition is correct and good enough!} as 
	\be 
	\label{eq: bundle}
	TM=\left\{(x,y)\;|\;x\in M,\; y\in T_xM\right\}
	\ee 
	\item At any point on the Manifold, we can also define the dual vector space $T_x^*M$, which is the vector space whose elements are linear functions from $T_xM$ to the underlying field. $T_x^*M$ is called \emph{cotangent space}, and by combining all of them for a given Manifold $M$, we obtain the \emph{cotangent bundle} denoted as $T^*M$.
	\item In physics, we tentatively call a function \emph{field} if its domain is the Manifold.\footnote{Note that this has nothing to do with the mathematical term \emph{field}, which refers to a set with two operations satisfying certain conditions as detailed in \S~\ref{}.} For instance, the function $f::M\to \R$ is a real scalar field on the manifold $M$; for instance, \emph{the temperature field} is a function which takes an element of the manifold (the position in the space) and spits out a real number, i.e. the temperature at that position. Likewise the \emph{electric field} is a function which takes an element of the manifold as input and spits out a vector, i.e. the electric field vector at that position.
	
	\item In the partial derivative notation of vectors (see 	\eqref{eq: vector derivative notation} for an example), \emph{the vector fields} become a simple generalization of that; for instance,
	\be 
	V::{}&{}\R^3\to T\R^3\\
	V={}&{}(x,y,z)\to\Big(x^2\pdr{}{x}+xz\pdr{}{y}+\frac{1}{x+y}\pdr{}{z}\Big)
	\ee 
	is a $3d$ vector field.\footnote{We are being very schematic with our definitions and types of objects when it comes to bundles. Indeed, the way we defined in \eqref{eq: bundle}, $V$ can have the codomain $T\R^3$ only if its output is actually $\left((x,y,z)\;,\;\Big(x^2\pdr{}{x}+xz\pdr{}{y}+\frac{1}{x+y}\pdr{}{z}\Big)\right)$. In fact, even this is not true: $(x,y,z)$ does not really represent the element of $M$ but rather an element of a \emph{chart} on $M$, which we do not cover in these notes. Furthermore, as we stated earlier in footnote~(\ref{footnote: tangent bundle}), the tangent bundle is defined only upto isomorphisms, hence even with charts introduced, we need to be extra careful with our definitions if we want rigor. Instead, we will only use types as \emph{schematic reminders} in this chapter and do not pursue the topic any further.
	}
	
	\item Multivector fields on the cotangent space, or equivalently covector fields on the tangent space, are called \emph{differential forms} (forms for short). For instance,
	\be 
w::{}&{}\R^3\to \Lambda^2(T_x\R^3)\\
w={}&{}(x,y,z)\to\Big(x^2dx\wedge dy+xzdx\wedge dz+\frac{1}{x+y}dy\wedge dz\Big)
\ee 
is a 2-form on the manifold $\R^3$.

	\item Tensor fields are also straightforward generalizations of tensors to mappings from manifolds; for example,
	\be 
	T::{}&{}S^2\to (T_x\R^2)\otimes(T_x\R^2)\otimes(T_x^*\R^2)\\
	T={}&{}(\f,\psi)\to\left(\sin(\psi)\pdr{}{\f}\otimes\pdr{}{\f}\otimes d\f+\cos(\psi+\f)\pdr{}{\psi}\otimes\pdr{}{\f}\otimes d\psi\right)
	\ee 
	is a type (2,1) tensor field on the manifold $S^2$.
\item Musical isomorphism: We define two operations that convert back and forth between tangent and cotangent bundles:
\bea 
\Flat ::{}&{} TM\to T^*M\\
\Sharp ::{}&{} T^*M\to TM
\eea 
which in practice amounts to raising and lowering the indices on the components of the tensors, i.e.
\be 
(x^ie_i)^\Flat=(x_i e^i)\;,\quad (x_ie^i)^\Sharp=(x^i e_i)
\ee 
In practice, we physicists do not use musical isomorphism terminology: the manifolds we work with are usually something called \emph{a metric space}, meaning that they are endowed with a \emph{metric} which is type $(0,2)$ tensor field (and the inverse metric which is type $(2,0)$ tensor field). The metric is then used to go back and forth between the dual spaces. In this course, we stick to Euclidean spaces, hence we do not need to introduce the metric tensor and its properties, and hence we won't: the discussion of musical isomorphism is simpler and sufficient enough!\footnote{The musical isomorphism is actually broader in the sense that it represent the abstract mapping between the bundles, which may not be actuated with a metric. For instance, \emph{fiber derivative} also induces a map from tangent to cotangent bundle, which is rather relevant in the study of \emph{Legendre transformations}.}
\item The application of musical isomorphism can be extended to $k-$vectors and forms as well, e.g. $(x_{ij}e^i\wedge e^j)^\Sharp=(x^{ij}e_i\wedge e_j)$
\end{enumerate}
}
\section{Grad, div, curl, and Laplacian}
\subsection{Exterior derivative and hodge star operator}
\draftnote{
\begin{enumerate}
	\item We define \emph{exterior derivative} as a map from $p-$form to a $(p+1)-$form, and denote it as $d\w$ where $\w$ is the $p-$form. With the differentials as the basis vectors of the cotangent bundle, it explicitly reads as 
	\be 
	\w=&\w^{i_1\dots i_p}dx_{i_1}\wedge\dots\wedge dx_{i_p}\\
	d\w=&\frac{\partial \w^{i_1\dots i_p}}{\partial x_k}dx_k\wedge dx_{i_1}\wedge\dots\wedge dx_{i_p}
	\ee 
	\item We have discussed several examples of exterior derivative in class. For instance, for the electric potential $V(x,y)=x^2+y^2$ given in a $2d$ plane, the exterior derivative of it reads as $dV=2xdx+2ydy$: if we were to plot them, they are actually vectors radially outward, precisely orthogonal to the equipotential lines (i.e. $V(x,y)=$constant); in fact, the electric field is actually proportional to this exterior derivative, i.e. $\bm{E}\sim dV$.
	\item The importance of the exterior derivative is that it is \emph{basis-independent}. Physically, this means that exterior derivative of an observer-covariant quantity is itself observer-covariant, which is precisely what we need! In contrast, a vector field defined as $v=\frac{\partial f}{\partial x}dx-\frac{\partial f}{\partial y}dy$ is \emph{not} observer-covariant even if the function $f$ is so. In other words, it cannot represent something physical!\footnote{Note to myself: we discussed these points in detail in class, so I better include a nice discussion here as well, along with some external references.}
	\item The Hodge duality introduced for multivectors naturally generalizes to the forms; in fact, we can immediately use the same formula given in \equref{eq: Hodge dual} for forms as follows:
	\begin{subequations}
		\be 
		\a(x_1,\dots,x_d)=\a_{i_1\dots i_k}(x_1,\dots,x_d)e^{i_1}\wedge\cdots\wedge e^{i_k}\\
		(\star\a)(x_1,\dots,x_d)=(\star\a)_{i_{k+1}\dots i_d}(x_1,\dots,x_d)e^{i_{k+1}}\wedge\cdots\wedge e^{i_d}
		\ee 
		and
		\be 
		(\star\a)_{i_{k+1}\dots i_d}(x_1,\dots,x_d)=\frac{1}{(d-k)!}\a_{i_1\dots i_k}(x_1,\dots,x_d)\e^{i_1\dots i_kl_{k+1}\dots l_d}\de_{i_{k+1}l_{k+1}}\cdots \de_{i_dl_d}
		\ee 
	\end{subequations}
	\item Schematically speaking, \emph{musical isomorphism}, \emph{Hodge-duality}, and \emph{exterior derivative} are our necessary and sufficient tools to construct \emph{differential maps} from scalar \& vector fields to scalar \& vector fields. In the next section, we will then use these to define various vector differentiation operations.
\end{enumerate}
}

\subsection{Invariant formulation of differentiation in vector calculus}
\draftnote{
\begin{enumerate}
	\item We define the gradient (denoted as \textsf{grad}) as follows:
	\be 
	\textsf{grad} ::{}&{} \texttt{Scalar Field}\to\texttt{Vector Field}\\
	\textsf{grad}= {}&{}f\to (df)^\Sharp\\
	\textsf{grad}= {}&{}\big((x_1,\dots,x_d)\to f(x_1,\dots,x_d)\big)\\
	&\quad\to \left((x_1,\dots,x_d)\to \pdr{f(x_1,\dots,x_d)}{ x_i}\hat{x_i}\right)\quad(\text{in Cartesian coordinates})
	\ee 
	\item It is also commonly denoted as $\grad$; for instance, for the scalar field $f(x,y)=x^2+y$, we write the vector field as $\grad{f}(x,y)=2x\pdr{}{x}+\pdr{}{y}$. In the freshman notation, this simply is $\grad{f}(x,y)=2x\hat{i}+\hat{j}$. 
	
	\item It is easiest to understand the gradient visually: For any scalar field $f$ whose value at any point gives the scalar $S$ (temperature, electric potential, pressure, etc.), draw the equi-$S$ lines (equitemperature, equipotential, etc.) which correspond to $S=$constant lines. The vector field $\grad{f}$ then is the collection of vectors that are \emph{orthogonal} to these equi-$S$ lines. In the case of electric potential scalar field $f$, $\grad{f}$ is proportional to the electric field; likewise, in the case of temperature field $f$, $\grad{f}$ is related to the heat flow vector field.
	\item We define the divergence (denoted as \textsf{div}) as follows:
	\be 
	\textsf{div} ::{}&{} \texttt{Vector Field}\to\texttt{Scalar Field}\\
	\textsf{div}= {}&{}v\to \big(\star d\star v^\Flat\big)\\
	\textsf{div}= {}&{}\big((x_1,\dots,x_d)\to v^i(x_1,\dots,x_d)\hat{x_i}\big)\\
	&\quad\to \left((x_1,\dots,x_d)\to \pdr{v_i(x_1,\dots,x_d)}{ x_i}\right)\quad(\text{in Cartesian coordinates\footnotemark})
	\ee 
	\footnotetext{
	It may look counter-intuitive to see that such a simple definition in Cartesian derivative involve so many operations in its abstract definition. Let us do the explicit derivation to see that it makes sense:
	\be 
	v=&v^i(x_1,\dots,x_d)\pdr{}{x^i}\\
	v^\Flat=&v_i(x_1,\dots,x_d) dx^i\\
	\star v^\Flat=&\frac{1}{(d-1)!}v_i(x_1,\dots,x_d)\e^{ik_2\dots k_d}\de_{k_2l_2}\cdots\de_{k_dl_d}\\&\quad\x dx^{l_2}\wedge \dots\wedge dx^{l_d}\\
	d\star v^\Flat=&\frac{1}{(d-1)!}\frac{\partial v_i(x_1,\dots,x_d)}{\partial x^m}\e^{ik_2\dots k_d}\de_{k_2l_2}\cdots\de_{k_dl_d}\\&\quad\x dx^m\wedge dx^{l_2}\wedge \dots\wedge dx^{l_d}\\
	\star d\star v^\Flat=&\frac{1}{(d-1)!}\frac{\partial v_i(x_1,\dots,x_d)}{\partial x^m}\e^{ik_2\dots k_d}\de_{k_2l_2}\cdots\de_{k_dl_d}\e^{ml_2\dots l_d}\\
	\star d\star v^\Flat=&\frac{\partial v_i(x_1,\dots,x_d)}{\partial x^m}\de^{im}\\
	\star d\star v^\Flat=&\frac{\partial v_i(x_1,\dots,x_d)}{\partial x^i}
	\ee 
	where we used $\e^{ia_1\dots a_n}	\e_{ka_1\dots a_n}=n!\de^i_k$.\\
}
	\item It is also commonly denoted as $\div{}$; for instance, for the vector field $v(x,y)=xy\pdr{}{x}-y^2\pdr{}{y}$, we write the scalar field as \mbox{$\div v(x,y)=-y$}.\footnote{This notation is no coincidence: the divergence can be seen as the dot product between the vector field and the gradient operator $\grad$.}
	\item Divergence measures \emph{the scalar source} of the vector field: we will learn more about this when we cover the Helmholtz decomposition.
	\item We define the curl (denoted as \textsf{curl}) as follows:
	\be 
	\textsf{curl} ::{}&{} \texttt{Vector Field}\to\texttt{Scalar Field}\quad (d=2)\\
	\textsf{curl} ::{}&{} \texttt{Vector Field}\to\texttt{Vector Field}\quad (d=3)\\
	\textsf{curl} ::{}&{} \texttt{Vector Field}\to(d-2)-\texttt{Vector Field}\quad (d>3)\\
	\textsf{curl}= {}&{}v\to \big(\star d v^\Flat\big)^\Sharp
	\ee 
	\item The curl operator reduces to the cross product with the gradient operation in $3d$, i.e. $\textsf{curl } v=\curl{v}$; in Cartesian coordinates, it then yields
	\be 
	\curl{v}=\left(\pdr{v_z}{y}-\pdr{v_y}{z}\right)\hat{x}+\left(\pdr{v_x}{z}-\pdr{v_z}{x}\right)\hat{y}+\left(\pdr{v_y}{x}-\pdr{v_x}{y}\right)\hat{z}
	\ee 
	\item Curl measures \emph{the vector source} of the given vector field: we will understand this better with the Helmholtz decomposition.
	\item Discussed the psudovector nature of curl and did a few examples in class.
	\item We now turn to our last operator: the Laplacian (denoted as $\Delta$). In principle, it can take any scalar or multivector field to itself; however, we will only define it for scalar and vector fields:\footnote{Note to myself: maybe discuss Laplace beltrami operator, or at least a few citations?}
	\be 
	\Delta ::{}&{} \texttt{Scalar Field}\to\texttt{Scalar Field}\\
	\Delta= {}&{}f\to \big(\star d\star d f\big)\\
	\Delta ::{}&{} \texttt{Vector Field}\to\texttt{Vector Field}\\
	\Delta= {}&{}v\to \big(d\star d\star v^\Flat-\star d\star d v^\Flat\big)^\Sharp
	\ee 
We only need this abstract definition to ensure that this operator is observer-covariant, i.e. we can do science with it!\footnote{Note to myself: Again, I discussed this whole observer-independence in class, but these notes lack a proper discussion.} In practice, we can also define this same operator as simply \emph{the gradient of divergence} and denote it as $\Delta=\div{\grad}=\grad^2$, which can take any tensor field to another tensor field: in Cartesian coordinates, it simply reads as follows:
\be 
R::{}&{}TM\otimes \dots\otimes TM\otimes T^*M\otimes \dots\otimes T^*M\\
R={}&{}R^{i_1\dots i_r}_{\quad\;\; k_1\dots k_s}\pdr{}{x^{i_1}}\otimes\dots\otimes\pdr{}{x^{i_r}}\otimes dx^{k_1}\otimes\dots\otimes dx^{k_s}\\
(\Delta R)::{}&{}TM\otimes \dots\otimes TM\otimes T^*M\otimes \dots\otimes T^*M\\
\Delta R={}&{}\frac{\partial^2 R^{i_1\dots i_r}_{\quad\;\; k_1\dots k_s}}{\partial x^m \partial x_m}\pdr{}{x^{i_1}}\otimes\dots\otimes\pdr{}{x^{i_r}}\otimes dx^{k_1}\otimes\dots\otimes dx^{k_s}
\ee 
\item In class, derived several identities such as $\div{\curl v}=0$ using their abstract definitions and the identity $d^2=0$.
\item In class, we discussed why we use these generalized definitions even though Cartesian component based ones are actually far more simple! The key point is that this way the result is guaranteed to be observer covariant/invariant! To emphasize, we defined a new operator in class (call it $\cD$) which acts on $2d-$vectors as
\be 
\cD\.A=\pdr{A_x}{y}+\pdr{A_y}{x}
\ee 
Even though this operator does generate a scalar field from a vector field, and even though there is nothing wrong with it mathematically, it is actually observer-dependent. In other words, the scalar that it generates changes when we change the coordinate system. Physically, we do not want scalars to be observer dependent (the temperature at a point should not change when I rotate clockwise 90 degrees). In this example, it is easy to see this as we can straightforwardly check it; but in general, it can become quite tedious for longer operators; and it does not instructs us how we should rather construct our operator. When we work with forms, we are guaranteed that any operator out of exterior derivative, hodge duality, and musical isomorphism is observer independent as it does not refer to a particular basis but is instead defined abstractly.
\end{enumerate}
}

\section{Helmholtz decomposition of vector fields}
Helmholtz decomposition and a brief mention of its generalization Hodge decomposition

\chapter{Integration in Vector Calculus}
\section{Curves and Line Integrals}
Curves, parametrization, arc length, Frenet–Serret formulas, line integral
\section{Generalized approach to integration}
Chains, boundaries of chains, Poincare lemma
\section{Integral theorems}
Discussion of generalized stokes theorem in forms, and its implication as \emph{Gradient theorem}, \emph{Divergence theorem}, \emph{Curl theorem}, and \emph{Green's theorem}.

\chapter{Curvilinear Coordinate Transformations}
Concept of coordinate charts, transformation of tensor fields under coordinate transformations, common orthogonal curvilinear coordinates (polar, cylindrical, spherical)

\part{Complex Analysis for Physicists}
\chapter{Basics of complex calculus}
\section{Structures of complex numbers}
\subsection{Complex numbers as a Euclidean vector space}
With $\{1,i\}$ as the standard basis, complex numbers are simply a real vector space of dimension two: complex plane as cartesian plane, geometric interpretation of complex numbers, complex absolute value as Euclidean norm, complex cojugation, polar form
\subsection{Further properties of complex numbers}
Complex numbers as an algebraically closed field, complex numbers as a commutative algebra over the reals, complex numbers (endowed with the metric $d(z_1,z_2)=\abs{z_1-z_2}$) as a complete metric space
\section{Complex differentiation}
Cauchy–Riemann equations, holomorphicity, Cauchy's integral formula

\chapter{Riemann Surfaces}
\section{Riemann sphere}
Extended complex numbers, projective maps
\section{Analysis of elementary functions via Riemann surfaces}
The qualitative analysis of the analytic structure of elementary functions via their Riemann surfaces (square-root, log, etc.),  

\chapter{Analytic Structure of Multi-valued ``functions''}
\section{Branch points, branch cuts, and singularities}
Branch analysis, different sheets, essential singularities, etc.
\section{Series expansion of complex functions}
Taylor and Laurent series, relations to Cauchy's integral formula, poles and residues, residue theorem
\section{Analytic continuation and monodromy theorem}

\appendix 
\addtocontents{toc}{\protect\setcounter{tocdepth}{1}}
\input{appendix}
\backmatter
\input{backmatter}
%%% Bibliography
\let\chapter\chapterTemporary
\checkoddpage\ifoddpage\strictpagecheck
	\begin{adjustwidth*}{0pt}{-\fullwidthlen}%
	\bibliography{\bibliographyfile}{}
\end{adjustwidth*}%
\else\strictpagecheck
\begin{adjustwidth*}{-\fullwidthlen}{0pt}%
	\bibliography{\bibliographyfile}{}
\end{adjustwidth*}%
\fi
\end{document}